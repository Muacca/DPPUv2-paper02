%==============================================================================
% Section 8: Topology Comparison under Weyl Extension
%==============================================================================
\section{Topology Comparison under Weyl Extension}
\label{sec:topology}

We now verify that the topology-selection principle of paper~I---the
energetic dominance of $\Sthree$---is preserved under the Weyl
extension. The three topologies $\Sthree$, $\Tthree$, and $\Nilthree$
are compared systematically.

\subsection{\texorpdfstring{$\alpha\le 0$}{α≤0}: Preservation of
  \texorpdfstring{$\Sthree$}{S³} Dominance}
\label{sec:topo-neg-alpha}

For $\alpha\le 0$, the lowest-energy vacua of the three topologies are:

\begin{table}[ht]
\centering
\caption{Topology comparison for $\alpha\le 0$ (reference parameters).}
\label{tab:topo-neg}
\begin{tabularx}{\textwidth}{@{}lrrlX@{}}
\toprule
Topology & $V_{\min}$ & $r^*$ & $\varepsilon^*$ & Status \\
\midrule
$\Sthree$   & $-421$   & $2.000$ & $0$
  & Deep stable vacuum (lowest energy) \\
$\Tthree$   & $\approx 0$ ($5.5\times 10^{-12}$) & $1.500$ & $-0.642$
  & Effectively flat \\
$\Nilthree$ & $\approx 5$ (flat-limit asymptote) & $1.500$
  & $\to\infty$ (flat limit) & Energetically unfavourable \\
\bottomrule
\end{tabularx}
\end{table}

$\Sthree$ achieves $V_{\min}=-421$, significantly the lowest energy,
confirming that the topology-selection principle is preserved under the
Weyl extension.

By Theorem~\ref{thm:stability}, $V_{\min}$ of $\Sthree$ is constant
(independent of~$\alpha$) for all $\alpha\le 0$. $\Tthree$ also has
$V_{\min}\approx 0$ independent of $\alpha$ (confirmed by the null
test below). $\Nilthree$ asymptotes to $V_{\min}\approx 5$ with weak
$\alpha$-dependence (see Appendix~\ref{app:nil3}).

\subsection{\texorpdfstring{$\alpha>0$}{α>0}: Loss of the Dominance
  Concept}
\label{sec:topo-pos-alpha}

For $\alpha>0$, the situation changes qualitatively:

\begin{table}[ht]
\centering
\caption{Topology comparison for $\alpha>0$.}
\label{tab:topo-pos}
\begin{tabularx}{\textwidth}{@{}llX@{}}
\toprule
Topology & $V_{\min}$ ($\alpha>0$) & Behaviour \\
\midrule
$\Sthree$ & $\to -\infty$ (linear divergence) & Unstable ($r\to 0$,
  $\varepsilon\to -1$) \\
$\Tthree$ & $\approx 0$ ($\alpha$-independent) & Stable but
  $V_{\min}\approx 0$ ($C^2=0$ $\Rightarrow$ Weyl term vanishes) \\
$\Nilthree$ & $\to -\infty$ (slower divergence than $\Sthree$)
  & Unstable \\
\bottomrule
\end{tabularx}
\end{table}

Both $\Sthree$ and $\Nilthree$ become unstable ($\Veff\to -\infty$).
$\Tthree$ remains stable but with $V_{\min}\approx 0$, offering no
deep stable vacuum. For $\alpha>0$ the concept of ``topology
selection'' itself becomes vacuous: no topology provides a physically
viable stable vacuum.

\subsection{\texorpdfstring{$\Tthree$}{T³} Null Test}
\label{sec:null-test}

$\Tthree\times\Sone$ provides a null test for all values of $\alpha$.
Since $\Tthree$ is flat, $C^2=0$ identically, and the Weyl term
$\alpha\,C^2$ vanishes; consequently, the effective potential should
exhibit no $\alpha$-dependence whatsoever.

\paragraph{Numerical verification.}
At all 201 points ($\alpha\in[-1,1]$, step~0.01):

\begin{table}[ht]
\centering
\caption{$\Tthree$ null-test results.}
\label{tab:null-test}
\begin{tabular}{@{}llc@{}}
\toprule
Quantity & Value & $\alpha$-dependence \\
\midrule
$r^*$ & $1.500$ & None (identical at all points) \\
$\varepsilon^*$ & $-0.642$ & None (identical at all points) \\
$V_{\min}$ & $5.54\times 10^{-12}$ & None (machine precision) \\
\bottomrule
\end{tabular}
\end{table}

The maximum deviation
$\max|\Veff(\alpha)-\Veff(0)| = 0.00$ (machine precision) confirms the
null test. This serves a dual purpose:
\begin{enumerate}
  \item \textbf{Engine validation}: the $\alpha$ implementation is
    correct---$C^2=0$ topologies produce exactly zero Weyl contribution.
  \item \textbf{Theoretical consistency}: the flatness of $\Tthree$ and
    the conformal invariance of the Weyl term are correctly coupled.
\end{enumerate}

\begin{figure}[ht]
\centering
\includegraphics[width=0.8\textwidth]{figures/fig07_t3_null_test.png}
\caption{$\Tthree$ null-test visualisation, confirming
  $\alpha$-independence of the $\Tthree$ effective potential at machine
  precision.}
\label{fig:null-test}
\end{figure}

\paragraph{On the non-zero $\varepsilon^*$ of $\Tthree$.}
The optimal anisotropy $\varepsilon^*\approx -0.642$ is non-isotropic,
but the potential landscape is extremely flat
($V_{\min}\approx 0$), so this ``minimum'' carries limited physical
significance. The near-zero energy cost for deformation reflects
the structural flexibility of~$\Tthree$.

\subsection{\texorpdfstring{$\Nilthree$}{Nil³} Behaviour: Asymptotic
  Approach to the Flat Limit}
\label{sec:nil3-behaviour}

$\Nilthree$ exhibits qualitatively distinct behaviour from the other
two topologies. An extended search over $\varepsilon\in[-0.95,5.0]$
(see Appendix~\ref{app:nil3}) reveals:
\begin{itemize}
  \item \textbf{No global minimum at finite $\varepsilon^*$}:
    For each fixed $\varepsilon$, $\Veff(r)$ possesses a local minimum
    in $r$ ($\Tthree$-like profile), but the minimised value
    $V_{\min}(\varepsilon)\equiv\min_r\Veff(r,\varepsilon)$ decreases
    monotonically in $\varepsilon$ with no interior minimum at finite
    $\varepsilon$; it asymptotes to the flat limit ($\Tthree$-like
    behaviour) as $\varepsilon\to\infty$.

  \item \textbf{Asymptotic $V_{\min}\approx 5 > 0$}: weakly
    $\alpha$-dependent, and energetically unfavourable compared to
    $\Sthree$ ($V_{\min}=-421$).

  \item \textbf{Physical interpretation}: the $\Nilthree$ structure
    constants contain the squashing factor $(1{+}\varepsilon)^{-4/3}$,
    which vanishes as $\varepsilon\to\infty$. The effective potential
    minimises the curvature cost ($C^2>0$) arising from the Heisenberg
    group structure by driving the structure constants to zero.
\end{itemize}

This result is consistent with the central theme of this paper:
conformally flat configurations ($C^2=0$) are energetically favoured.
$\Sthree$ achieves $C^2=0$ exactly at the finite deformation
$\varepsilon=0$, while $\Nilthree$ can only approach $C^2\to 0$
asymptotically in the $\varepsilon\to\infty$ limit---a fundamental
geometric asymmetry.

The universality of the $\alpha=0$ stability boundary for $\Nilthree$
is confirmed by high-resolution scans (Appendix~\ref{app:nil3-alpha}).

\subsection{Summary of the Topology Comparison}
\label{sec:topo-summary}

\begin{table}[ht]
\centering
\caption{Three-topology comparison summary (representative values at
  $\alpha\le 0$).}
\label{tab:topo-summary}
\begin{tabularx}{\textwidth}{@{}lllX@{}}
\toprule
Quantity & $\Sthree$ & $\Tthree$ & $\Nilthree$ \\
\midrule
$V_{\min}$ & $-421$ & $\approx 0$ & $\approx 5$ (flat-limit asymptote) \\
$r^*$ & $2.000$ & $1.500$ & $1.500$ \\
$\varepsilon^*$ & $0$ & $-0.642$ & $\to\infty$ (flat limit) \\
$C^2(\varepsilon^*)$ & $0$ & $0$ & $\to 0$ (asymptotically) \\
$\alpha$-dependence & None (Thm~1) & None ($C^2=0$) & Weak \\
Paper~I dominance & \textbf{Lowest energy} & Neutral & Unfavourable \\
Weyl-extended dominance & \textbf{Preserved} & Unchanged
  & No global minimum at finite $\varepsilon^*$ (asymptotes to flat limit) \\
\bottomrule
\end{tabularx}
\end{table}

The topology-selection principle of paper~I is preserved under the Weyl
extension for $\alpha\le 0$. $\Sthree$ achieves $C^2=0$ exactly at
$\varepsilon^*=0$ and forms a stable isotropic vacuum, whereas
$\Nilthree$ has no global minimum at finite $\varepsilon^*$ and can
only approach the flat limit asymptotically. $\Tthree$ is unaffected by the Weyl term ($C^2=0$
identically) but has $V_{\min}\approx 0$.

The mathematical root $\varepsilon=-2$ of $C^2=0$ lies beyond the
physical singularity at $\varepsilon=-1$ and is therefore excluded from
discussion as physically inaccessible. Note that $\Tthree$ has
$C^2=0$ identically everywhere, and $\Nilthree$ approaches $C^2\to 0$
only in the asymptotic limit $\varepsilon\to\infty$; hence the existence
of a finite root of $C^2=0$ in the region $\varepsilon<-1$ is a
peculiarity of the parametric representation of~$\Sthree$.
