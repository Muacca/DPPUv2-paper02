%==============================================================================
% Section 2: Framework
%==============================================================================
\section{Framework}
\label{sec:framework}

This section formulates the extended Lagrangian obtained by adding a
Weyl-squared term to the EC+NY theory of paper~I, and introduces the
volume-preserving squashed ansatz.

\subsection{EC+NY+Weyl Lagrangian}
\label{sec:lagrangian}

The extended Lagrangian studied in this paper is
\begin{equation}
  \mathcal{L}
  = \frac{R_{\EC}}{2\kappa^2}
  + \theta_{\NY}\,N
  + \alpha\,C^2\,,
  \label{eq:lagrangian}
\end{equation}
where each term is defined as follows:

\begin{itemize}
  \item $R_{\EC}/(2\kappa^2)$: the Ricci scalar computed from the
    Einstein--Cartan connection (including torsion), with gravitational
    coupling constant~$\kappa$.

  \item $\theta_{\NY}\,N$: the Nieh--Yan term, where
    $N = \dd(e^a\wedge T_a)$ is the Nieh--Yan density (a 4-form)
    and $\theta_{\NY}$ its coupling constant, following the notation
    of paper~I~\cite{paper1}.

  \item $\alpha\,C^2$: the Weyl-squared term, where
    $C^2 = C_{abcd}\,C^{abcd}$ is the Kretschner-type Weyl scalar
    computed from the Levi-Civita connection, and $\alpha$ is a
    dimensionless coupling constant.
\end{itemize}

The Weyl tensor is defined with respect to the Levi-Civita connection.
This choice avoids torsion contamination that would generically break
conformal invariance if the EC connection were used instead.
The quantities $R_{\EC}$ and $N$ are computed from the EC connection.

\paragraph{Sign conventions.}
We adopt the same conventions as paper~I:
\begin{itemize}
  \item Frame metric:
    $\eta_{ab}=\mathrm{diag}(+1,+1,+1,+1)$ (Euclidean signature).
  \item Riemann tensor:
    $R^{a}{}_{bcd} = \partial_c\Gamma^{a}{}_{bd}
    - \partial_d\Gamma^{a}{}_{bc}
    + \Gamma^{a}{}_{ec}\Gamma^{e}{}_{bd}
    - \Gamma^{a}{}_{ed}\Gamma^{e}{}_{bc}$.
  \item Contortion:
    $K_{abc} = \tfrac{1}{2}(T_{abc}+T_{bca}-T_{cab})$
    (Hehl et al.~\cite{Hehl1976}).
  \item Levi-Civita symbol: $\varepsilon_{0123}=+1$.
\end{itemize}

\subsection{Review of the $M^3\times\Sone$ Minisuperspace Ansatz}
\label{sec:ansatz-review}

Following paper~I, we decompose the four-dimensional Euclidean manifold as
$\mathcal{M}_4 = \mathcal{M}_3\times\Sone$, where $\mathcal{M}_3$ is a
compact quotient of a three-dimensional Lie group admitting a left-invariant
coframe $\{\sigma^i\}$ ($i=0,1,2$), and $\Sone$ has circumference~$L$.

In paper~I the isotropic coframe $e^a = r\,\sigma^a$ ($a=0,1,2$),
$e^3 = L\,\dd\tau$ was employed, with the scale variable~$r$ serving as
the principal argument of the effective potential $\Veff(r)$.

\subsection{Squashed Ansatz}
\label{sec:squashed}

To diagnose the effect of the Weyl term, a deformation parameter away
from isotropy is required. We introduce an axisymmetric,
volume-preserving squashing on~$\Sthree$:
\begin{equation}
  e^0 = r\,(1{+}\varepsilon)^{1/3}\,\sigma^0,\quad
  e^1 = r\,(1{+}\varepsilon)^{1/3}\,\sigma^1,\quad
  e^2 = r\,(1{+}\varepsilon)^{-2/3}\,\sigma^2,\quad
  e^3 = L\,\dd\tau\,,
  \label{eq:squashed-coframe}
\end{equation}
where
\begin{itemize}
  \item $r>0$ is the scale variable (identical to paper~I),
  \item $\varepsilon$ is the anisotropy parameter; $\varepsilon=0$ is the
    isotropic point and $\varepsilon>-1$ is required for physical
    significance:
    \begin{itemize}
      \item $\varepsilon>0$: oblate deformation ($e^0,e^1$ directions
        expand, $e^2$ contracts),
      \item $\varepsilon<0$: prolate deformation ($e^2$ expands, $e^0,e^1$
        contract),
      \item $\varepsilon=-1$: singular point (complete collapse of one
        direction).
    \end{itemize}
\end{itemize}

\paragraph{Volume preservation.}
The squashing factors satisfy
$(1{+}\varepsilon)^{1/3}\times(1{+}\varepsilon)^{1/3}
\times(1{+}\varepsilon)^{-2/3}=1$,
so that the four-volume is $\varepsilon$-independent:
\begin{equation}
  \Vol(\mathcal{M}_4) = 2\pi^2 L\,r^3
  \qquad (\text{independent of }\varepsilon).
  \label{eq:volume}
\end{equation}
This ensures that $\varepsilon$ modifies the \emph{shape} without
changing the \emph{size}~($r$).

\paragraph{Squashed structure constants.}
The squashing modifies the $\Sthree$ structure constants to
\begin{equation}
  C^i{}_{jk}(\varepsilon)
  = \frac{4}{r}\,\varepsilon_{ijk}\times f_i(\varepsilon),
  \label{eq:squashed-structure}
\end{equation}
with
\begin{equation}
  f_0(\varepsilon) = f_1(\varepsilon) = (1{+}\varepsilon)^{2/3},
  \qquad
  f_2(\varepsilon) = (1{+}\varepsilon)^{-4/3}.
  \label{eq:squash-factors}
\end{equation}
At the isotropic point $\varepsilon=0$, $f_0=f_1=f_2=1$ and one
recovers the paper~I structure constants.

\subsection{Notation Summary}
\label{sec:notation}

Table~\ref{tab:notation} collects the principal symbols used throughout
the paper.

\begin{table}[ht]
\centering
\caption{Principal symbols and their numerical scan ranges.}
\label{tab:notation}
\begin{tabular}{@{}lll@{}}
\toprule
Symbol & Meaning & Scan range \\
\midrule
$r$ & Scale variable & $[0.01,\,10]$ \\
$\varepsilon$ & Anisotropy parameter & $[-0.95,\,5.0]$ \\
$\alpha$ & Weyl coupling constant & $[-1,\,1]$ \\
$V$ & Vector-torsion amplitude & Fixed ($V=4$) \\
$\eta$ & Axial-torsion amplitude & Fixed ($\eta=-2$, varied in \S\ref{sec:universality}) \\
$\theta_{\NY}$ & Nieh--Yan coupling & Fixed ($\theta_{\NY}=1$) \\
$\kappa$ & Gravitational coupling & Fixed ($\kappa=1$) \\
$L$ & $\Sone$ circumference & Fixed ($L=1$) \\
\bottomrule
\end{tabular}
\end{table}

The paper~I parameters $(V,\eta,\theta_{\NY})$ are scanned in
\S\ref{sec:universality}; \S\ref{sec:weyl}--\ref{sec:instability} use
the reference values listed above.

\subsection{Separation of the Effective Potential}
\label{sec:separation}

Under the squashed ansatz, the effective potential separates
\emph{exactly linearly} in~$\alpha$:
\begin{equation}
  \Veff(r,\varepsilon;\alpha)
  = \VEC(r,\varepsilon)
  - \alpha\,C^2(r,\varepsilon)\cdot\Vol(r)\,.
  \label{eq:Veff-separation}
\end{equation}

Note the sign: since the Euclidean effective potential is defined as
$\Veff = -\mathcal{L}\times\Vol$ (see Appendix~\ref{app:engine},
\S\ref{sec:app-flow}), the $+\alpha\,C^2$ term in the
Lagrangian~\eqref{eq:lagrangian} enters as $-\alpha\,C^2\cdot\Vol$ in
the effective potential.

The two constituents are:
\begin{itemize}
  \item $\VEC(r,\varepsilon)$: the EC+NY effective potential, independent
    of~$\alpha$.
    \begin{equation}
      \VEC = -\biggl(\frac{R_{\EC}}{2\kappa^2}
      + \theta_{\NY}\,N\biggr)\times\Vol\,.
    \end{equation}

  \item $C^2(r,\varepsilon)\cdot\Vol(r)$: a purely geometric quantity,
    independent of the paper~I parameters $(V,\eta,\theta_{\NY})$.
\end{itemize}

This separation is the foundation of our analysis:
\begin{enumerate}
  \item $\VEC$ inherits the full content of paper~I.
  \item $C^2\cdot\Vol$ is a geometric invariant decoupled from the
    torsion parameters.
  \item The stability classification reduces to the sign of~$\alpha$.
\end{enumerate}

The computational engine (DPPUv2 Engine~v4)---including the derivation of
the Levi-Civita connection, construction of the EC connection, and
evaluation of the Weyl tensor---is documented in
Appendix~\ref{app:engine}.
