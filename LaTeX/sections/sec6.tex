%==============================================================================
% Section 6: Instability (alpha > 0)
%==============================================================================
\section{Instability for \texorpdfstring{$\alpha>0$}{α>0}}
\label{sec:instability}

We now prove that the effective potential is unbounded below for
$\alpha>0$ and establish that $\alpha=0$ is the sharp stability
boundary.

\subsection{Theorem~2 (Unbounded Instability)}
\label{sec:thm2}

\begin{theorem}[Unbounded instability]
\label{thm:instability}
For the EC+NY+Weyl effective potential on $\Sthree\times\Sone$:

\begin{enumerate}[label=(\alph*)]
  \item If $\alpha>0$, then $\Veff$ is unbounded below:
    \begin{equation}
      \inf_{r>0,\;\varepsilon>-1}\Veff(r,\varepsilon;\alpha) = -\infty\,.
    \end{equation}

  \item If $\alpha\le 0$ and $\VEC$ is bounded below, then $\Veff$ is
    also bounded below:
    \begin{equation}
      \inf_{r>0,\;\varepsilon>-1}\Veff(r,\varepsilon;\alpha) > -\infty\,.
    \end{equation}
\end{enumerate}

Hence $\alpha=0$ is the \emph{sharp boundary} between stability and
instability.
\end{theorem}

\subsection{Proof}
\label{sec:proof-thm2}

\paragraph{Part (a): Unboundedness for $\alpha>0$.}
Let $\alpha>0$ and fix any $\varepsilon_0\neq 0$ (e.g.\
$\varepsilon_0=-1/2$). The effective potential reads
\begin{equation}
  \Veff(r,\varepsilon_0;\alpha)
  = \VEC(r,\varepsilon_0)
  - \alpha\cdot
    \frac{2048\pi^2 L\,\varepsilon_0^2(\varepsilon_0{+}2)^2}
         {3\,r\,(1{+}\varepsilon_0)^{16/3}}\,.
\end{equation}
The asymptotic behaviour as $r\to 0^+$:
\begin{itemize}
  \item EC part: $\VEC(r,\varepsilon_0)\sim r \to 0$.
  \item Weyl part:
    $-\alpha\times\text{const}/r \to -\infty$.
\end{itemize}
Therefore
\begin{equation}
  \lim_{r\to 0^+}\Veff(r,\varepsilon_0;\alpha) = -\infty\,.
  \qquad\qed
\end{equation}

\paragraph{Part (b): Boundedness for $\alpha\le 0$.}
For $\alpha\le 0$:
\begin{equation}
  \Veff(r,\varepsilon;\alpha)
  = \VEC(r,\varepsilon) + |\alpha|\,C^2\cdot\Vol
  \ge \VEC(r,\varepsilon)\,.
\end{equation}
If $\VEC$ is bounded below ($\inf\VEC>-\infty$), then
\begin{equation}
  \inf\Veff \ge \inf\VEC > -\infty\,.
  \qquad\qed
\end{equation}

\subsection{Analytic Confirmation of the \texorpdfstring{$\VEC$}{V\_EC}
  Asymptotics}
\label{sec:VEC-asymptotics}

The key asymptotic estimate ``$\VEC\sim r$ as $r\to 0$'' used in part~(a)
is confirmed from the exact symbolic expression at the isotropic point
(paper~I, \S3.1.3):
\begin{equation}
  \VEC(r,0)
  = \frac{2\pi^2 Lr}{3\kappa^2}
    \bigl(V^2 r^2 + 6V\eta\kappa^2 r\,\theta_{\NY}
    + 9\eta^2 - 36\bigr)\,.
  \label{eq:VEC-exact}
\end{equation}
The individual $r$-scalings are:

\begin{table}[ht]
\centering
\caption{$r$-scaling of each term in $\VEC(r,0)$.}
\label{tab:r-scaling}
\begin{tabular}{@{}llc@{}}
\toprule
Term & Origin & Power of $r$ \\
\midrule
$V^2 r^2$ & Vector-torsion self-interaction & $r^3$ \\
$6V\eta\kappa^2 r\,\theta_{\NY}$ & Nieh--Yan cross term & $r^2$ \\
$9\eta^2-36$ & Curvature + axial torsion & $r^1$ \\
\bottomrule
\end{tabular}
\end{table}

The lowest power is $r^1$, so $\VEC\to 0$ as $r\to 0$ (bounded below).
The Weyl contribution scales as
\begin{equation}
  \VWeyl = -\alpha\,C^2\cdot\Vol \sim -\frac{\alpha}{r}
  \qquad (r\to 0)\,,
\end{equation}
giving a dominance ratio
\begin{equation}
  \frac{|\VWeyl|}{|\VEC|}
  \sim \frac{1/r}{r} = \frac{1}{r^2} \to \infty
  \qquad (r\to 0)\,.
\end{equation}
The Weyl term asymptotically dominates the EC part, and the
two-order gap in $r$-scaling is not reversed within the present
parameterisation.

\subsection{Analytic Linear Coefficient and Numerical Agreement}
\label{sec:linear-coeff}

The numerically observed linear behaviour $V_{\min}(\alpha)\approx
-K\times\alpha$ for $\alpha>0$ has its coefficient~$K$ predicted
analytically. Numerical optimisation converges to the search boundary
$(r_{\min},\varepsilon_{\min})=(0.01,-0.95)$, at which
\begin{equation}
  K(0.01,-0.95)
  = C^2(0.01,-0.95)\times\Vol(0.01)
  = 5.823\times 10^{12}\,.
\end{equation}

\begin{table}[ht]
\centering
\caption{Analytic prediction versus numerical result for the linear
  coefficient~$K$.}
\label{tab:linear-coeff}
\begin{tabular}{@{}lll@{}}
\toprule
Quantity & Analytic prediction & Numerical result \\
\midrule
$K$ & $5.823\times 10^{12}$ & $5.82\times 10^{12}$ (0.05\% accuracy) \\
\bottomrule
\end{tabular}
\end{table}

\textbf{Important caveat.} This linear coefficient $K$ is not a
physical quantity; it depends on the search boundary and is an
artefact of the finite scan range. The true physical conclusion is
$\inf\Veff=-\infty$ for $\alpha>0$; the finite $V_{\min}$ is merely
the minimum within the explored region.

\subsection{Direction of Instability}
\label{sec:instability-direction}

For $\alpha>0$, the instability drives the system simultaneously in two
directions:

\begin{table}[ht]
\centering
\caption{Instability directions for $\alpha>0$.}
\label{tab:instability-dir}
\begin{tabularx}{\textwidth}{@{}llXX@{}}
\toprule
Direction & Scaling & Geometric meaning & Physical meaning \\
\midrule
$r\to 0$
  & $C^2\cdot\Vol\sim 1/r$
  & Volume collapse of $\Sthree$
  & Contraction / singularity \\
$\varepsilon\to -1$
  & $C^2\cdot\Vol\sim 1/\delta^{16/3}$
  & Extreme uniaxial 
  & Complete breakdown \\
  \quad & \quad & \quad deformation & \quad of 3d isotropy \\[4pt]
\bottomrule
\end{tabularx}
\end{table}

\noindent
Here $\delta=1+\varepsilon\to 0^+$. Both directions proceed
simultaneously, and no lower bound on $\Veff$ exists.

\subsection{Why \texorpdfstring{$\alpha=0$}{α=0} Is the Sharp Boundary}
\label{sec:sharp-boundary}

The mathematical reason that $\alpha=0$ is \emph{exactly} the boundary
is the \emph{unboundedness} of $C^2\cdot\Vol$.

If $C^2\cdot\Vol$ were bounded ($\sup C^2\cdot\Vol = M<\infty$), there
would be room for stability up to some critical $\alpha_c>0$
(satisfying $\alpha_c M\lesssim|\inf\VEC|$). However, since
$C^2\cdot\Vol\sim 1/r$ diverges as $r\to 0$, any $\alpha>0$---no
matter how small---is sufficient to make $\Veff\to -\infty$. The
transition therefore occurs precisely and sharply at $\alpha=0$.

\subsection{Relation to Ghost Instability of Weyl Gravity}
\label{sec:ghost}

The $\alpha>0$ instability is the minisuperspace manifestation of the
well-known ghost instability of conformal (Weyl) gravity. The generic
Weyl-gravity action
\begin{equation}
  S_{\text{Weyl}}
  = \int\dd^4 x\,\sqrt{g}\;\alpha\,C_{\mu\nu\rho\sigma}\,
    C^{\mu\nu\rho\sigma}
\end{equation}
yields fourth-order equations of motion, which by Ostrogradsky's
theorem~\cite{Woodard2015} generically lead to an energy unbounded
below (ghost instability).

In our minisuperspace calculation, this instability manifests concretely
as collapse towards $r\to 0$ and maximal anisotropy $\varepsilon\to -1$.
The condition $\alpha\le 0$ for stability is the minisuperspace
counterpart of the ghost-avoidance condition in Weyl gravity.

\subsection{Three-Topology Comparison:
  \texorpdfstring{$V_{\min}(\alpha)$}{Vmin(α)}}
\label{sec:Vmin-comparison}

\begin{figure}[ht]
\centering
\includegraphics[width=0.85\textwidth]{figures/fig05_topology_comparison.png}
\caption{$V_{\min}(\alpha)$ for the three topologies $\Sthree$, $\Tthree$,
  and $\Nilthree$. The stability transition at $\alpha=0$ is most
  dramatic for $\Sthree$, while $\Tthree$ provides a null test.
  The topology comparison is discussed in detail in
  \S\ref{sec:topology}.}
\label{fig:topology-comparison}
\end{figure}
