%==============================================================================
% Section 4: Weyl Extension
%==============================================================================
\section{Weyl Extension: Squashed Ansatz}
\label{sec:weyl}

In this section we derive the closed-form expression for the Weyl scalar
$C^2(r,\varepsilon)$ under the squashed ansatz of \S\ref{sec:squashed}
and elucidate the structure of the effective potential.

\subsection{Computation of the Weyl Tensor}
\label{sec:weyl-computation}

The Weyl tensor is computed from the Levi-Civita connection via the
following standard procedure:

\begin{enumerate}
  \item \textbf{Levi-Civita connection} in the orthonormal frame
    (generalised Koszul formula):
    \begin{equation}
      \Gamma^{a}{}_{bc}
      = \tfrac{1}{2}\bigl(C^{a}{}_{bc} + C^{c}{}_{ba}
      - C^{b}{}_{ac}\bigr)\,.
    \end{equation}

  \item \textbf{Riemann tensor} in the frame basis:
    \begin{equation}
      R^{a}{}_{bcd}
      = \Gamma^{a}{}_{ec}\,\Gamma^{e}{}_{bd}
      - \Gamma^{a}{}_{ed}\,\Gamma^{e}{}_{bc}
      + \Gamma^{a}{}_{be}\,C^{e}{}_{cd}\,.
    \end{equation}

  \item \textbf{Ricci tensor and scalar}:
    $R_{bd} = R^a{}_{bad}$, $\;R = R^a{}_a$.

  \item \textbf{Weyl tensor} (4-dimensional definition):
    \begin{equation}
      C_{abcd} = R_{abcd}
      - \tfrac{1}{2}\bigl(g_{ac}R_{bd} - g_{ad}R_{bc}
      - g_{bc}R_{ad} + g_{bd}R_{ac}\bigr)
      + \tfrac{R}{6}\bigl(g_{ac}g_{bd} - g_{ad}g_{bc}\bigr)\,.
      \label{eq:weyl-def}
    \end{equation}

  \item \textbf{Weyl scalar}:
    $C^2 = C_{abcd}\,C^{abcd}$ (with $g^{ab}=\delta^{ab}$ in the
    orthonormal frame).
\end{enumerate}

\subsection{Lemma~1: Closed-Form Weyl Scalar}
\label{sec:lemma1}

\begin{lemma}
\label{lem:weyl-scalar}
The four-dimensional Weyl scalar of the squashed $\Sthree\times\Sone$
metric, computed from the Levi-Civita connection, is
\begin{equation}
  C^2(r,\varepsilon)
  = \frac{1024\,\varepsilon^2(\varepsilon{+}2)^2}
         {3\,r^4\,(1{+}\varepsilon)^{16/3}}\,.
  \label{eq:C2}
\end{equation}
\end{lemma}

\begin{proof}
By symbolic algebraic computation following the procedure of
\S\ref{sec:weyl-computation}, with the squashed structure
constants~\eqref{eq:squashed-structure} as input. All components of the
Weyl tensor are computed, and the sum of squares is simplified and
factored to yield~\eqref{eq:C2}. Full details are given in
Appendix~\ref{app:symbolic}.
\end{proof}

\subsection{Key Properties of Lemma~1}
\label{sec:C2-properties}

\paragraph{(i) Vanishing at the isotropic point: $C^2(r,0)=0$.}
The factor $\varepsilon^2$ in the numerator ensures $C^2=0$ at
$\varepsilon=0$. Physically, the round $\Sthree$ is a space of
constant curvature and hence conformally flat; the product metric
$\Sthree\times\Sone$ is therefore conformally flat in four dimensions,
with identically vanishing Weyl tensor.

\paragraph{(ii) Non-negativity: $C^2(r,\varepsilon)\ge 0$
  $(\varepsilon>-1)$.}
As a sum of squares in the orthonormal frame, $C^2\ge 0$ by
definition. The only zero (for $\varepsilon>-1$) is $\varepsilon=0$;
the other root $\varepsilon=-2$ lies beyond the singularity at
$\varepsilon=-1$.

\paragraph{(iii) Quadratic behaviour at the isotropic point.}
\begin{equation}
  \frac{\partial C^2}{\partial\varepsilon}\bigg|_{\varepsilon=0}=0,
  \qquad
  \frac{\partial^2 C^2}{\partial\varepsilon^2}\bigg|_{\varepsilon=0}
  = \frac{8192}{3r^4} > 0\,.
\end{equation}
Hence $C^2$ has a quadratic minimum at the isotropic point:
\begin{equation}
  C^2(r,\varepsilon)
  \approx \frac{8192}{3r^4}\,\varepsilon^2
  \qquad (|\varepsilon|\ll 1)\,.
\end{equation}

\paragraph{(iv) Divergence as $r\to 0$.}
For fixed $\varepsilon\neq 0$, $C^2\sim 1/r^4$. The product with the
volume~\eqref{eq:volume} gives
\begin{equation}
  C^2\cdot\Vol
  = \frac{2048\pi^2 L\,\varepsilon^2(\varepsilon{+}2)^2}
         {3\,r\,(1{+}\varepsilon)^{16/3}}
  \sim \frac{1}{r}
  \qquad (r\to 0)\,.
  \label{eq:C2Vol}
\end{equation}
This unboundedness is the key to the instability proof in
\S\ref{sec:instability}.

\subsection{Structure of Non-Vanishing Weyl Components}
\label{sec:weyl-components}

For general $\varepsilon$, there are 24 non-vanishing Weyl-tensor
components, all sharing the common factor
\begin{equation}
  C_{abcd} \propto
  \frac{\varepsilon(\varepsilon{+}2)}{r^2\,(1{+}\varepsilon)^{8/3}}\,.
\end{equation}
Representative independent components:

\begin{table}[ht]
\centering
\caption{Representative non-vanishing Weyl-tensor components.}
\label{tab:weyl-components}
\begin{tabular}{@{}ll@{}}
\toprule
Component & Value \\
\midrule
$C_{0101}$
  & $+\dfrac{16\,\varepsilon(\varepsilon{+}2)}
            {3r^2(1{+}\varepsilon)^{8/3}}$ \\[8pt]
$C_{0202}$, $C_{0303}$
  & $-\dfrac{8\,\varepsilon(\varepsilon{+}2)}
            {3r^2(1{+}\varepsilon)^{8/3}}$ \\[8pt]
$C_{2323}$
  & $+\dfrac{16\,\varepsilon(\varepsilon{+}2)}
            {3r^2(1{+}\varepsilon)^{8/3}}$ \\[8pt]
$C_{1212}$, $C_{1313}$
  & $-\dfrac{8\,\varepsilon(\varepsilon{+}2)}
            {3r^2(1{+}\varepsilon)^{8/3}}$ \\
\bottomrule
\end{tabular}
\end{table}

All components contain $\varepsilon(\varepsilon{+}2)$ as a factor and
vanish at $\varepsilon=0$. A complete listing is provided in
Appendix~\ref{app:symbolic}.

\subsection{Effective Potential with Explicit Weyl Contribution}
\label{sec:Veff-explicit}

Substituting the result of Lemma~1 into the
separation~\eqref{eq:Veff-separation} yields
\begin{equation}
  \Veff(r,\varepsilon;\alpha)
  = \VEC(r,\varepsilon)
  - \alpha\cdot
    \frac{2048\pi^2 L\,\varepsilon^2(\varepsilon{+}2)^2}
         {3\,r\,(1{+}\varepsilon)^{16/3}}\,.
  \label{eq:Veff-explicit}
\end{equation}

The effect of the Weyl term depends on the sign of~$\alpha$:

\begin{table}[ht]
\centering
\caption{Effect of the Weyl term $-\alpha\,C^2\cdot\Vol$ on
  $\varepsilon\neq 0$ configurations.}
\label{tab:alpha-sign}
\begin{tabular}{@{}lll@{}}
\toprule
Sign of $\alpha$ & Weyl contribution $-\alpha\,C^2\cdot\Vol$
  & Effect on $\varepsilon\neq 0$ \\
\midrule
$\alpha=0$ & $0$ & None (paper~I recovered) \\
$\alpha<0$ & $+|\alpha|\,C^2\cdot\Vol > 0$ & Positive penalty (stabilises isotropy) \\
$\alpha>0$ & $-\alpha\,C^2\cdot\Vol < 0$ & Negative reward (promotes anisotropy) \\
\bottomrule
\end{tabular}
\end{table}

\subsection{Effective Potential Contour Maps}
\label{sec:contour}

\begin{figure}[ht]
\centering
\includegraphics[width=\textwidth]{figures/fig02_landscape.png}
\caption{Contour maps of $\Veff$ in the $(r,\varepsilon)$ plane for
  $\Sthree$ at $\alpha=-10$, $0$, and $+10$. For $\alpha\le 0$ the
  isotropic vacuum ($\varepsilon=0$) forms a stable valley, while for
  $\alpha>0$ the valley is destroyed and the potential landscape
  collapses. The analytic foundations of this qualitative change are
  established in \S\ref{sec:stability} and \S\ref{sec:instability}.}
\label{fig:landscape}
\end{figure}
