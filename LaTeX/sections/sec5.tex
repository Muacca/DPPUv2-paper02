%==============================================================================
% Section 5: Stability (alpha <= 0)
%==============================================================================
\section{Stability of the Isotropic Vacuum (\texorpdfstring{$\alpha\le 0$}{α≤0})}
\label{sec:stability}

We now prove that for $\alpha\le 0$ the isotropic vacuum of
$\Sthree\times\Sone$ is analytically protected from the Weyl extension.

\subsection{Theorem~1 (Weyl Stability of the Isotropic Vacuum)}
\label{sec:thm1}

\begin{theorem}[Weyl stability]
\label{thm:stability}
Under the EC+NY+Weyl Lagrangian~\eqref{eq:lagrangian} on
$\Sthree\times\Sone$, the isotropic vacuum ($\varepsilon=0$) satisfies:

\begin{enumerate}[label=(\alph*)]
  \item The Weyl scalar vanishes identically at the isotropic point:
    \begin{equation}
      C^2(r,\varepsilon{=}0) = 0
      \qquad (\forall\,r>0)\,.
    \end{equation}

  \item For $\alpha\le 0$, the Weyl term reinforces (or leaves
    unaffected) the isotropic minimum of the effective potential:
    \begin{equation}
      \Veff(r,\varepsilon;\alpha)
      = \VEC(r,\varepsilon)
      + |\alpha|\,C^2(r,\varepsilon)\cdot\Vol(r)\,,
    \end{equation}
    where $|\alpha|\,C^2\cdot\Vol\ge 0$, with equality if and only if
    $\varepsilon=0$.

  \item As a corollary, if $\VEC$ attains a global minimum at
    $(r_0,0)$, then $\Veff$ also attains its global minimum at
    $(r_0,0)$ with the same minimum value $\VEC(r_0,0)$, for all
    $\alpha\le 0$.
\end{enumerate}
\end{theorem}

\subsection{Proof}
\label{sec:proof-thm1}

\paragraph{Part (a).}
Setting $\varepsilon=0$ in Lemma~\ref{lem:weyl-scalar}, the factor
$\varepsilon^2$ in the numerator of~\eqref{eq:C2} gives
$C^2(r,0)=0$. \qed

\paragraph{Part (b).}
For $\alpha\le 0$, write $-\alpha=|\alpha|\ge 0$. Then
\begin{equation}
  \Veff(r,\varepsilon;\alpha)
  = \VEC(r,\varepsilon) + |\alpha|\,C^2(r,\varepsilon)\cdot\Vol(r)\,.
\end{equation}
By Lemma~\ref{lem:weyl-scalar}~(ii), $C^2\ge 0$. Since
$\Vol(r)=2\pi^2 Lr^3>0$, it follows that
$|\alpha|\,C^2\cdot\Vol\ge 0$, with equality if and only if $C^2=0$,
i.e.\ $\varepsilon=0$. Hence:
\begin{itemize}
  \item $\Veff(r,\varepsilon;\alpha)\ge \VEC(r,\varepsilon)$
    for all $r,\varepsilon$;
  \item $\Veff(r,0;\alpha) = \VEC(r,0)$ for all $r$.
\end{itemize}
The Weyl term adds a positive penalty only in the $\varepsilon\neq 0$
directions and leaves the isotropic slice unchanged. \qed

\paragraph{Part (c).}
Suppose $\VEC$ has a global minimum at $(r_0,0)$:
\begin{equation}
  \VEC(r_0,0) \le \VEC(r,\varepsilon)
  \qquad \forall\,r,\varepsilon\,.
\end{equation}
For any $(r,\varepsilon)$:
\begin{equation}
  \Veff(r,\varepsilon;\alpha)
  = \VEC(r,\varepsilon) + |\alpha|\,C^2\cdot\Vol
  \ge \VEC(r,\varepsilon)
  \ge \VEC(r_0,0)\,.
\end{equation}
On the other hand:
\begin{equation}
  \Veff(r_0,0;\alpha)
  = \VEC(r_0,0) + |\alpha|\cdot 0\cdot\Vol
  = \VEC(r_0,0)\,.
\end{equation}
Hence $(r_0,0)$ is a global minimum of $\Veff$ with value
$\VEC(r_0,0)$.

Moreover, if the minimum of $\VEC$ is strict (attained only at
$(r_0,0)$), the Weyl penalty $|\alpha|\,C^2\cdot\Vol>0$ for
$\varepsilon\neq 0$ makes the minimum of $\Veff$ \emph{even more}
strictly isolated. \qed

\subsection{Assumptions of Theorem~1}
\label{sec:thm1-assumptions}

The conclusion of part~(c) relies on the premise that $\VEC$ has a
global minimum at $\varepsilon=0$. This premise is:
\begin{itemize}
  \item \textbf{Numerically verified}: the Stage~2D optimisation with
    reference parameters $(V{=}4,\,\eta{=}{-}2,\,\theta_{\NY}{=}1,\,
    \kappa{=}1,\,L{=}1)$ yields a minimum at $(r_0,\varepsilon) =
    (2.000,\,0)$ with $V_{\min}=-421.103$.
  \item \textbf{Partially supported analytically}: the positive
    definiteness of the $\varepsilon$-direction Hessian of $\VEC$ at
    the isotropic point provides a local-minimum proof. A complete
    analytic proof of global minimality remains an open problem.
\end{itemize}

\subsection{Numerical Verification}
\label{sec:thm1-numerics}

The predictions of Theorem~1 are compared with Stage~2D numerical
results:

\begin{table}[ht]
\centering
\caption{Theorem~1 predictions versus numerical results.}
\label{tab:thm1-verification}
\begin{tabularx}{\textwidth}{@{}lXl@{}}
\toprule
Prediction (from Theorem~1) & Numerical result & Agreement \\
\midrule
$V_{\min}$ constant for $\alpha\le 0$
  & $-421.103$ at all 201 points ($\alpha\in[-1,1]$), 11-digit match
  & $\checkmark$ \\
$r^*$ independent of $\alpha$
  & $r^*=2.000\pm 10^{-7}$ for all $\alpha\le 0$
  & $\checkmark$ \\
$\varepsilon^*$ independent of $\alpha$
  & $\varepsilon^*\approx 0$ ($10^{-8}$ precision) for all $\alpha\le 0$
  & $\checkmark$ \\
\bottomrule
\end{tabularx}
\end{table}

The 11-digit agreement provides strong numerical evidence for the
correctness of the theorem.

\subsection{Bifurcation Diagrams}
\label{sec:bifurcation}

\begin{figure}[ht]
\centering
\includegraphics[width=0.8\textwidth]{figures/fig03_bifurcation.png}
\caption{Bifurcation diagram $\varepsilon^*(\alpha)$ for $\Sthree$.
  For $\alpha\le 0$ the optimal anisotropy remains at
  $\varepsilon^*=0$; for $\alpha>0$ a bifurcation to
  $\varepsilon^*\neq 0$ occurs.}
\label{fig:bifurcation}
\end{figure}

\begin{figure}[ht]
\centering
\includegraphics[width=0.8\textwidth]{figures/fig04_size_stability.png}
\caption{Optimal scale $r^*(\alpha)$ for $\Sthree$. For $\alpha\le 0$,
  $r^*$ remains constant at $r^*=2.000$; for $\alpha>0$, $r^*$
  collapses to the search boundary.}
\label{fig:size-stability}
\end{figure}

\subsection{Physical Interpretation: Weyl Penalty Stabilises Isotropy}
\label{sec:weyl-penalty}

For $\alpha<0$, the Weyl term $|\alpha|\,C^2\cdot\Vol$ acts as a
\emph{penalty} on anisotropy ($\varepsilon\neq 0$). Near the isotropic
point, the Taylor expansion of \S\ref{sec:C2-properties}~(iii) gives
\begin{equation}
  \Delta\Veff
  \approx \frac{16384\pi^2\,|\alpha|\,L}{3r}\,\varepsilon^2\,.
\end{equation}
At $r=r^*=2$, $L=1$:
\begin{equation}
  \Delta\Veff \approx 26947\,|\alpha|\,\varepsilon^2\,.
\end{equation}
Even for $|\alpha|=1$ and a small anisotropy $\varepsilon=0.01$,
the penalty is $\Delta V\approx 2.7$. Larger $|\alpha|$ deepens the
isotropic valley and more strongly suppresses anisotropic fluctuations.

This stabilisation mechanism provides a clear physical picture: the
$\alpha<0$ Weyl term penalises conformal curvature ($C^2$), thereby
dynamically protecting spatial isotropy.
