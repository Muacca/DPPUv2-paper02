%==============================================================================
% Section 3: Chiral Equilibrium
%==============================================================================
\section{Chiral Equilibrium: \texorpdfstring{$P=0$}{P=0}}
\label{sec:chiral}

In this section we prove that the Pontryagin density
$P=\innerprod{R}{*R}$ vanishes identically under the
$M^3\times\Sone$ minisuperspace ansatz with EC connection. This
eliminates the topological threat of vacuum decay through self-dual
instanton tunnelling within the present framework.

\subsection{Pontryagin Density and Self-Duality}
\label{sec:pontryagin}

On a four-dimensional Euclidean manifold, the Pontryagin density is
defined as the inner product of the curvature 2-form $R^{ab}$ with its
Hodge dual~$*R^{ab}$:
\begin{equation}
  P = \innerprod{R}{*R}
  = \tfrac{1}{4}\,\varepsilon^{abcd}\,R_{abef}\,R_{cd}{}^{ef}\,.
  \label{eq:pontryagin}
\end{equation}
$P$ is the density of the Pontryagin class and is related to
self-duality as follows:
\begin{itemize}
  \item $R=*R$ (self-dual) $\;\Leftrightarrow\;$ $P = E > 0$,
  \item $R=-*R$ (anti-self-dual) $\;\Leftrightarrow\;$ $P = -E < 0$,
\end{itemize}
where $E = \innerprod{R}{R} = R_{abcd}\,R^{abcd}$ is the scalar
related to the Euler (Gauss--Bonnet) density.

When (anti-)self-dual instanton solutions exist, they furnish
finite-action extrema that can mediate vacuum decay via quantum
tunnelling. Since $P\neq 0$ is a necessary condition for self-duality,
the identity $P=0$ excludes such solutions.

\subsection{Proposition~1 (Chiral Equilibrium)}
\label{sec:prop1}

\begin{proposition}[Chiral equilibrium]
\label{prop:chiral}
Under the $M^3\times\Sone$ minisuperspace ansatz with Einstein--Cartan
connection, the Pontryagin density vanishes identically:
\begin{equation}
  P = \innerprod{R}{*R} = 0\,.
  \label{eq:P-zero}
\end{equation}
This holds as an algebraic identity for:
\begin{itemize}
  \item all choices of $M^3$ ($\Sthree$, $\Tthree$, $\Nilthree$),
  \item all torsion modes (MX, AX, VT),
  \item all Nieh--Yan variants (FULL, TT, REE),
  \item all parameter values $(r,L,\eta,V,\kappa,\theta_{\NY})$.
\end{itemize}
\end{proposition}

\subsection{Proof}
\label{sec:proof-prop1}

\paragraph{Step~1: Orthogonal decomposition of 2-forms.}
Let $(x^0,x^1,x^2,\tau)$ denote coordinates on
$M^4 = M^3\times\Sone$. The space of 2-forms decomposes into two
mutually orthogonal blocks:
\begin{equation}
  \Lambda^2(M^4)
  = \underbrace{\Lambda^2(M^3)}_{\text{Spatial (S)}}
  \;\oplus\;
  \underbrace{\Lambda^1(M^3)\wedge\dd\tau}_{\text{Mixed (M)}}\,.
  \label{eq:2form-decomp}
\end{equation}
Explicitly, the spatial block~(S) consists of 2-forms with index pairs
$\{(01),(02),(12)\}$, and the mixed block~(M) of
$\{(03),(13),(23)\}$.

\paragraph{Step~2: The Hodge dual exchanges blocks.}
The four-dimensional Hodge dual (with $\varepsilon_{0123}=+1$) acts as
\begin{equation}
\begin{aligned}
  *(01) &= +(23), & *(03) &= +(12), \\
  *(02) &= -(13), & *(13) &= -(02), \\
  *(12) &= +(03), & *(23) &= +(01).
\end{aligned}
\label{eq:hodge-exchange}
\end{equation}
That is, the Hodge dual maps $\Lambda^2(M^3)$ to
$\Lambda^1(M^3)\wedge\dd\tau$ and vice versa.

\paragraph{Step~3: Curvature lies in the spatial block.}
Under the minisuperspace ansatz, the $\Sone$ direction~$\tau$ is a
Killing direction and the connection is spatially homogeneous. As a
consequence, the curvature 2-form $R^{ab}{}_{cd}$ has non-vanishing
lower-index pairs $(c,d)$ only in the spatial block:
$(c,d)\in\{(01),(02),(12)\}$. Components with $(c,d)$ in the mixed
block vanish identically.

This property has been verified by symbolic computation for all three
topologies (see \S\ref{sec:symbolic-verification} and
Appendix~\ref{app:symbolic}).

\paragraph{Step~4: Orthogonality implies $P=0$.}
Since $R$ has lower indices in the spatial block only, $*R$ has lower
indices in the mixed block only (Step~2). The two blocks are orthogonal,
hence
\begin{equation}
  \innerprod{R}{*R} = 0\,.
  \qquad\qed
\end{equation}

\subsection{One-Line Summary}
\label{sec:one-line}

The proof may be condensed to a single chain of implications:
\begin{equation}
  R \in \Lambda^2(M^3)
  \;\Rightarrow\;
  *R \in \Lambda^1(M^3)\wedge\dd\tau
  \;\Rightarrow\;
  \innerprod{R}{*R} = 0\,.
\end{equation}

\subsection{Symbolic Verification}
\label{sec:symbolic-verification}

Algebraic verification using symbolic computation software confirms
$P=0$ for all three topologies (full details in
Appendix~\ref{app:symbolic}):

\begin{table}[ht]
\centering
\caption{Symbolic verification of $P=0$ across topologies.}
\label{tab:P-zero}
\begin{tabular}{@{}lcccc@{}}
\toprule
Topology & Spatial components & Mixed components
  & $P=\innerprod{R}{*R}$ & Status \\
\midrule
$\Sthree\times\Sone$ & 6 & 0 & $\mathbf{0}$ & $\checkmark$ \\
$\Tthree\times\Sone$  & 6 & 0 & $\mathbf{0}$ & $\checkmark$ \\
$\Nilthree\times\Sone$ & 6 & 0 & $\mathbf{0}$ & $\checkmark$ \\
\bottomrule
\end{tabular}
\end{table}

Representative non-vanishing curvature components for
$\Sthree\times\Sone$:
\begin{equation}
  R^{01}{}_{01}
  = \frac{-V^2 r^2/9 - \eta^2 - 8\eta - 12}{r^2}\,,
  \qquad
  R^{03}{}_{12}
  = \frac{-2V(\eta+4)}{3r}\,.
  \label{eq:curvature-examples}
\end{equation}
All non-vanishing lower-index pairs $(c,d)$ belong to the spatial block
$\{01,02,12\}$, with no mixed-block components.

\subsection{Physical Interpretation}
\label{sec:chiral-interpretation}

\subsubsection{Chiral equilibrium}

The identity $P=0$ means that the self-dual and anti-self-dual
components of the curvature are balanced in norm. Decomposing
\begin{equation}
  R = R^+ + R^-\,,\qquad
  R^{\pm} = \tfrac{1}{2}(R \pm *R)\,,
\end{equation}
one finds
$P = \innerprod{R}{*R} = \|R^+\|^2 - \|R^-\|^2$, so that $P=0$
implies
\begin{equation}
  \|R^+\| = \|R^-\|\,.
  \label{eq:chiral-balance}
\end{equation}
Both chiralities are present in precisely equal measure; we call this
\emph{chiral equilibrium}.

\subsubsection{Structural prohibition of self-dual instantons}

Self-duality $R=*R$ requires $R^-=0$, i.e.\ $\|R^-\|=0$. By chiral
equilibrium, $\|R^-\|=0$ forces $\|R^+\|=0$ simultaneously, hence
$R=0$. Therefore, \emph{as long as the curvature is non-vanishing},
neither self-dual nor anti-self-dual solutions can exist on
$M^3\times\Sone$ within the minisuperspace ansatz.

The $\Sthree\times\Sone$ isotropic vacuum is thus protected from
instanton-mediated tunnelling decay. This protection is a
\emph{geometric consequence} of the product structure
$M^3\times\Sone$ and the minisuperspace ansatz, independent of
specific parameter values or torsion configurations.

\subsection{Conditions for Breakdown of \texorpdfstring{$P=0$}{P=0}}
\label{sec:P-breakdown}

Achieving $P\neq 0$ and thereby admitting self-dual solutions requires
extending beyond the present geometric setting:

\begin{table}[ht]
\centering
\caption{Extensions that can break $P=0$.}
\label{tab:P-breakdown}
\begin{tabularx}{\textwidth}{@{}llX@{}}
\toprule
Direction of extension & Modification & Why $P=0$ breaks \\
\midrule
Inhomogeneous torsion & Position-dependent $T^a(x)$
  & Breaks homogeneity; mixed-block curvature appears \\
Non-product manifold & Beyond $M^3\times\Sone$
  & Product structure underlies the block restriction \\
Lorentzian signature & $(-,+,+,+)$
  & Eigenspace structure of Hodge $*$ changes \\
Non-compact $M^3$ & Inhomogeneous geometry
  & Curvature structure may differ \\
\bottomrule
\end{tabularx}
\end{table}

These extensions are discussed as future directions in
\S\ref{sec:discussion}.
