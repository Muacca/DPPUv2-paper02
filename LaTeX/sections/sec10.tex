%==============================================================================
% Section 10: Conclusion
%==============================================================================
\section{Conclusion}
\label{sec:conclusion}

We have subjected the $\Sthree\times\Sone$ isotropic vacuum of the
Einstein--Cartan + Nieh--Yan minisuperspace (paper~I) to two
independent extensions---topological and dynamical---and demonstrated
its structural robustness both analytically and numerically.

\subsection{Summary of Main Results}

The principal results are encapsulated in four statements:

\paragraph{(1) Proposition~1 (Chiral equilibrium, $P=0$).}
Under the $M^3\times\Sone$ minisuperspace ansatz with EC connection,
the Pontryagin density $P=\innerprod{R}{*R}=0$ vanishes identically.
This follows from the orthogonal decomposition
$\Lambda^2(M^4)=\Lambda^2(M^3)\oplus\Lambda^1(M^3)\wedge\dd\tau$ and
the block-exchange property of the Hodge dual: when the curvature lies
entirely in the spatial block, its dual lies in the mixed block, and
orthogonality gives $P=0$.

This identity means that self-dual instantons are forbidden within this
framework, eliminating the threat of instanton-mediated vacuum decay.

\paragraph{(2) Theorem~1 (Weyl stability of the isotropic vacuum).}
For $\alpha\le 0$, the conformal flatness $C^2=0$ of isotropic
$\Sthree\times\Sone$ shields the vacuum from the Weyl term, and the
global minimum of paper~I is analytically protected. Numerically, at
all 201 scan points ($\alpha\in[-1,1]$) the minimum value
$V_{\min}=-421.103$ (11-digit precision), $r^*=2.000$, and
$\varepsilon^*=0$ are maintained independently of~$\alpha$.

\paragraph{(3) Theorem~2 (Unbounded instability for $\alpha>0$).}
For $\alpha>0$, the asymptotic dominance of the Weyl term
($\VWeyl\sim -\alpha/r$ versus $\VEC\sim r$ as $r\to 0$) renders the
effective potential unbounded below ($\inf\Veff=-\infty$). The boundary
$\alpha=0$ is sharp, owing to the unboundedness of
$C^2\cdot\Vol\sim 1/r$.

\paragraph{(4) Theorem~3 (Parameter independence of the stability
  boundary).}
The stability boundary $\alpha=0$ is independent of the paper~I
parameters $(V,\eta,\theta_{\NY})$, a consequence of the geometric
decoupling of $C^2$ from torsion. This is confirmed analytically and
verified numerically at four representative points spanning Type~I
centre to II/III boundary.

\subsection{Sign Constraint on the Weyl Coupling Constant}

Theorems~2 and~3 jointly imply the sign constraint $\alpha\le 0$ for
the EC+NY+Weyl theory. $\alpha>0$ triggers ghost instability (the
minisuperspace manifestation of Ostrogradsky's theorem), while
$\alpha<0$ provides isotropy stabilisation and a repulsive core
(regularisation) at $r\to 0$. This constraint is independent of
paper~I parameter tuning.

\subsection{Preservation of the Topology-Selection Principle}

A systematic comparison of three topologies ($\Sthree$, $\Tthree$,
$\Nilthree$) confirms that the topology-selection principle of
paper~I---$\Sthree$ forming the lowest-energy vacuum---is preserved for
$\alpha\le 0$. $\Sthree$ attains $C^2=0$ exactly at $\varepsilon^*=0$,
forming a stable isotropic vacuum. $\Nilthree$ lacks a stable
anisotropic vacuum and asymptotes to a flat limit
($\varepsilon\to\infty$) with $V_{\min}\approx 5>0$, which is
energetically unfavourable relative to $\Sthree$.
