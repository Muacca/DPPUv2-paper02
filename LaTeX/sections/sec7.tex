%==============================================================================
% Section 7: Universality across paper I Parameters
%==============================================================================
\section{Universality across Paper~I Parameters}
\label{sec:universality}

We show that the stability boundary $\alpha=0$ is independent of the
paper~I parameters $(V,\eta,\theta_{\NY})$ and verify this numerically.

\subsection{Theorem~3 (Parameter Independence of the Stability
  Boundary)}
\label{sec:thm3}

\begin{theorem}[Parameter independence]
\label{thm:universality}
For any paper~I parameters $(V,\eta,\theta_{\NY})$ for which $\VEC$ is
bounded below, $\alpha=0$ is the stability boundary:
\begin{itemize}
  \item $\alpha>0$: $\inf\Veff = -\infty$ (unbounded below);
  \item $\alpha\le 0$: $\inf\Veff > -\infty$ (bounded below).
\end{itemize}
This boundary does not depend on the values of
$(V,\eta,\theta_{\NY})$.
\end{theorem}

\subsection{Proof: Geometric Decoupling}
\label{sec:proof-thm3}

Theorem~3 is a structural corollary of the proof of Theorem~2. The
argument rests on two key properties:

\paragraph{(i) Geometric decoupling.}
The Weyl scalar $C^2(r,\varepsilon)$ is computed entirely from the
Levi-Civita connection and is a purely geometric quantity independent
of the paper~I parameters $(V,\eta,\theta_{\NY})$:
\begin{equation}
  C^2(r,\varepsilon)
  = \frac{1024\,\varepsilon^2(\varepsilon{+}2)^2}
         {3\,r^4\,(1{+}\varepsilon)^{16/3}}\,.
\end{equation}
No torsion amplitude $\eta$, $V$, no Nieh--Yan coupling
$\theta_{\NY}$, and no gravitational coupling $\kappa$ appears. $C^2$
is a function of $(r,\varepsilon)$ alone.

\paragraph{(ii) Asymptotic dominance.}
The $r\to 0$ scaling comparison:
\begin{itemize}
  \item $\VEC(r,\varepsilon)\sim r$ ($r\to 0$, $\varepsilon$ fixed):
    the coefficient depends on paper~I parameters, but the leading
    power is uniformly $r^1$.
  \item $\VWeyl = -\alpha\,C^2\cdot\Vol \sim -\alpha/r$
    ($r\to 0$, $\varepsilon\neq 0$ fixed): independent of paper~I
    parameters.
\end{itemize}
The dominance ratio $|\VWeyl|/|\VEC|\sim 1/r^2\to\infty$ diverges
regardless of the values of $(V,\eta,\theta_{\NY})$.

\paragraph{Completion of the proof.}
The proof of Theorem~\ref{thm:instability}(a) does not depend on the
specific form of $\VEC$ (i.e.\ on the paper~I parameters). The
divergence $C^2\cdot\Vol\sim 1/r$ is geometrically guaranteed, so for
any $\alpha>0$, $\Veff\to -\infty$. Part~(b) requires only that
$\VEC$ be bounded below, without reference to its specific lower
bound.

Therefore, under the condition that $\VEC$ admits a stable vacuum
(Type~I or Type~II in the paper~I classification), the stability
boundary $\alpha=0$ holds irrespective of the paper~I parameters.
\qed

\subsection{Numerical Verification}
\label{sec:thm3-numerics}

The parameter independence is verified at four representative points
spanning the paper~I parameter space. In all cases $V=4$,
$\theta_{\NY}=1$, $\kappa=1$, $L=1$ are fixed, and $\eta$ is varied
to traverse different paper~I phases.

\begin{table}[ht]
\centering
\caption{Numerical verification of Theorem~3 across paper~I parameter
  space.}
\label{tab:thm3-verification}
\begin{tabular}{@{}lllrrl@{}}
\toprule
Parameter set & $\eta$ & Paper~I class & $V_{\min}$ ($\alpha\le 0$) & $r^*$
  & $\alpha=0$ boundary \\
\midrule
Type~I centre  & $-3.0$ & Type~I          & $-583$  & $2.65$ & Sharp \\
I/II boundary  & $-2.0$ & I/II boundary   & $-421$  & $2.00$ & Sharp \\
Type~II centre & $\phantom{-}0.0$ & Type~II & $-137$  & $0.87$ & Sharp \\
II/III boundary & $+2.0$ & II/III boundary & $+0.03$ & $0.01$ & Sharp \\
\bottomrule
\end{tabular}
\end{table}

\subsubsection{Type~I ($\eta=-3.0$)}
A deep stable vacuum at $V_{\min}\approx -583$, $r^*\approx 2.65$.
Immediate destabilisation upon $\alpha>0$. Even the most stable vacuum
in the paper~I landscape is susceptible to the Weyl instability for
$\alpha>0$.

\subsubsection{I/II boundary ($\eta=-2.0$)}
The reference parameter set, consistent with the detailed analysis of
\S\ref{sec:stability}--\ref{sec:instability}.

\subsubsection{Type~II ($\eta=0.0$)}
A clear stable vacuum at $r^*\approx 0.866$,
$V_{\min}\approx -137$, with a sharp stability boundary at
$\alpha=0$. Note that $\eta=0$ corresponds to zero axial torsion;
stability is maintained by vector torsion~$V$ and the Nieh--Yan
coupling~$\theta_{\NY}$ alone.

\subsubsection{II/III boundary ($\eta=2.0$)}
$V_{\min}\approx +0.03$ (positive), with $r^*$ at the search boundary.
This lies near the paper~I Type~III transition, where the premise
``$\VEC$ is bounded below with a stable vacuum'' barely holds.
Nevertheless, the sharp transition at $\alpha=0$ is observed.

\subsection{Visual Evidence of Parameter Independence}
\label{sec:facet-plot}

\begin{figure}[ht]
\centering
\includegraphics[width=\textwidth]{figures/fig06_gamma2_facet.png}
\caption{$V_{\min}(\alpha)$ for four paper~I parameter sets spanning
  the full range from Type~I centre to II/III boundary. In all cases
  $\alpha=0$ acts as the bifurcation point, providing direct visual
  evidence for Theorem~3.}
\label{fig:facet-plot}
\end{figure}

\subsection{Physical Interpretation of Parameter Independence}
\label{sec:param-indep-interp}

The parameter independence of the $\alpha=0$ boundary originates from
three structural facts:
\begin{enumerate}
  \item \textbf{Weyl tensor as a ``shape'' quantity.}
    $C^2$ depends only on the conformal structure $(r,\varepsilon)$ and
    is independent of the torsion amplitudes and the NY coupling
    $(V,\eta,\theta_{\NY})$.

  \item \textbf{Exact $\alpha$-linearity.}
    $\Veff = \VEC - \alpha\,C^2\cdot\Vol$ is strictly linear in
    $\alpha$; the critical value of $\alpha$ is determined solely by
    the bounded/unbounded character of $\VEC$ and $C^2\cdot\Vol$.

  \item \textbf{Unboundedness of $C^2\cdot\Vol$.}
    The divergence $C^2\cdot\Vol\sim 1/r$ as $r\to 0$ is a geometric
    property independent of the paper~I parameters.
\end{enumerate}
Together, these ensure that the stability boundary $\alpha=0$ cannot be
shifted by tuning the paper~I parameters.
