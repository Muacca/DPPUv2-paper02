%==============================================================================
% Section 1: Introduction
%==============================================================================
\section{Introduction}
\label{sec:introduction}

\subsection{Background: Results of Paper~I and Two Open Questions}
\label{sec:intro-background}

In paper~I~\cite{paper1}, we analysed the Einstein--Cartan (EC) gravity
supplemented with the Nieh--Yan (NY) term~\cite{NiehYan1982} in a Euclidean-signature
minisuperspace framework and established a systematic phase classification of
the effective potential $\Veff(r)$. In particular, by comparing three spatial
topologies---$\Sthree$ (SU(2)), $\Tthree$ (flat), and $\Nilthree$
(Heisenberg group)---within a unified reduction procedure, we showed that the
$\Sthree\times\Sone$ configuration forms the energetically most favourable
stable vacuum over a wide region of parameter space: the
\emph{topology-selection principle}.

Two natural questions regarding the physical robustness of these results were
left open:

\begin{enumerate}
  \item \textbf{Topological stability.}
    Could self-dual instanton solutions exist on $\Sthree\times\Sone$,
    mediating vacuum decay via quantum tunnelling~\cite{Chandia1997}?

  \item \textbf{Dynamical stability.}
    Is the stability of the isotropic vacuum preserved under
    higher-curvature corrections to general relativity, notably the
    Weyl-squared term $\alpha\,C^2$?
\end{enumerate}

\subsection{Approach of the Present Work}
\label{sec:intro-approach}

In this paper we address both questions through independent analyses and
provide a comprehensive demonstration of the robustness of the
$\Sthree\times\Sone$ isotropic vacuum.

\paragraph{Topological stability.}
We show that the Pontryagin density
$P = \innerprod{R}{*R}$ vanishes identically under the
$M^3\times\Sone$ minisuperspace ansatz with EC connection. This
algebraic identity---reflecting what we call \emph{chiral equilibrium}
between self-dual and anti-self-dual curvature components---implies that
self-dual instanton solutions are forbidden within this framework.

\paragraph{Dynamical stability.}
We extend the EC+NY Lagrangian by a Weyl-squared term $\alpha\,C^2$ and
introduce a volume-preserving squashed ansatz with anisotropy
parameter~$\varepsilon$, constructing the two-variable effective potential
$\Veff(r,\varepsilon;\alpha)$. The stability structure is classified by the
sign of~$\alpha$:

\begin{itemize}
  \item \textbf{$\alpha \le 0$}: The conformal flatness $C^2=0$ of isotropic
    $\Sthree$ shields the vacuum from the Weyl term. For $\alpha<0$ the Weyl
    penalty on $\varepsilon\neq 0$ directions further stabilises isotropy.

  \item \textbf{$\alpha > 0$}: The Weyl term asymptotically dominates
    $\VEC$ as $r\to 0$, rendering the effective potential unbounded below
    ($\Veff\to -\infty$). This is the minisuperspace manifestation of the
    Ostrogradsky ghost instability of Weyl gravity.
\end{itemize}

\subsection{Main Results: Four Theoretical Pillars}
\label{sec:intro-results}

The principal results of this paper are summarised by the following
four statements:

\begin{enumerate}
  \item \textbf{Proposition~1 (Chiral equilibrium, $P=0$).}
    Under the $M^3\times\Sone$ minisuperspace ansatz with EC connection,
    the Pontryagin density satisfies
    $P = \innerprod{R}{*R} = 0$
    identically. Self-dual instanton decay paths are closed within this
    ansatz.

  \item \textbf{Theorem~1 (Weyl stability of the isotropic vacuum).}
    For $\alpha\le 0$, the isotropic vacuum of $\Sthree\times\Sone$ is
    shielded from the Weyl term by conformal flatness, and the global
    minimum of $\VEC$ is analytically protected.

  \item \textbf{Theorem~2 (Unbounded instability for $\alpha>0$).}
    For $\alpha>0$, $\Veff$ is unbounded below
    ($\inf\Veff = -\infty$); for $\alpha\le 0$ and bounded $\VEC$, $\Veff$
    is bounded below. Hence $\alpha=0$ is the sharp stability boundary.

  \item \textbf{Theorem~3 (Parameter independence of the stability boundary).}
    The stability boundary $\alpha=0$ is independent of the paper~I
    parameters $(V,\eta,\theta_{\NY})$, owing to the geometric decoupling
    of~$C^2$ from torsion.
\end{enumerate}

\subsection{Logical Flow}
\label{sec:intro-flow}

\begin{figure}[ht]
\centering
\includegraphics[width=0.85\textwidth]{figures/fig01_argument_flow.png}
\caption{Logical flow of the paper. Starting from the EC+NY isotropic
  $\Sthree\times\Sone$ vacuum (paper~I), the topological threat
  (Question~1) is addressed by Proposition~1 (\S\ref{sec:chiral}), while
  the dynamical threat (Question~2) is resolved by Theorems~1--3
  (\S\ref{sec:stability}--\S\ref{sec:universality}). The topology
  comparison (\S\ref{sec:topology}) confirms preservation of the
  $\Sthree$ dominance.}
\label{fig:argument-flow}
\end{figure}

\subsection{Organisation of the Paper}
\label{sec:intro-outline}

The remainder of this paper is organised as follows.

\begin{itemize}
  \item \textbf{\S\ref{sec:framework} (Framework)}: EC+NY+Weyl Lagrangian
    and the squashed ansatz.
  \item \textbf{\S\ref{sec:chiral} (Chiral Equilibrium)}: Analytic proof of
    $P=0$ (Proposition~1).
  \item \textbf{\S\ref{sec:weyl} (Weyl Extension)}: Closed-form Weyl scalar
    $C^2(r,\varepsilon)$ and effective potential structure.
  \item \textbf{\S\ref{sec:stability} (Stability, $\alpha\le 0$)}: Proof of
    Theorem~1.
  \item \textbf{\S\ref{sec:instability} (Instability, $\alpha>0$)}: Proof of
    Theorem~2.
  \item \textbf{\S\ref{sec:universality} (Universality)}: Proof of
    Theorem~3.
  \item \textbf{\S\ref{sec:topology} (Topology Comparison)}: Robustness of
    the topology-selection principle under Weyl extension.
  \item \textbf{\S\ref{sec:discussion} (Discussion)}: Repulsive core
    ($\alpha<0$), Lorentzian extension, limitations.
  \item \textbf{\S\ref{sec:conclusion} (Conclusion)}: Summary.
\end{itemize}

\noindent
The DPPUv2 computation engine v4 specification is given in
Appendix~\ref{app:engine}, numerical verification details in
Appendix~\ref{app:numerical}, and symbolic computation details in
Appendix~\ref{app:symbolic}.
