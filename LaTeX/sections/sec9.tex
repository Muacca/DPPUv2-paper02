%==============================================================================
% Section 9: Discussion
%==============================================================================
\section{Discussion}
\label{sec:discussion}

\subsection{Robustness of Paper~I Results}
\label{sec:robustness}

This paper has examined two independent threats to the
$\Sthree\times\Sone$ isotropic vacuum discovered in paper~I and
demonstrated its robustness against both.

\paragraph{Topological threat (self-dual instantons).}
Proposition~\ref{prop:chiral} establishes that the Pontryagin density
$P=\innerprod{R}{*R}=0$ vanishes identically under the
$M^3\times\Sone$ minisuperspace ansatz with EC connection. This
precludes self-dual instanton solutions within this framework and
eliminates vacuum decay via instanton-mediated quantum tunnelling.

\paragraph{Dynamical threat (higher-curvature corrections).}
Theorems~\ref{thm:stability}--\ref{thm:universality} show that the
$\alpha\le 0$ Weyl extension leaves the paper~I phase diagram
unaffected, under the stated assumptions. The conformal flatness
$C^2=0$ of $\Sthree\times\Sone$ causes the Weyl contribution to
vanish at the isotropic vacuum, leaving $V_{\min}$, $r^*$, and
$\varepsilon^*=0$ invariant throughout $\alpha\le 0$.

These two protection mechanisms are mutually independent, resting on
distinct mathematical structures (topological orthogonality and
conformal flatness). Their combination provides strong evidence for the
physical robustness of the $\Sthree\times\Sone$ isotropic vacuum.

\subsection{Repulsive Core (Regularisation) from
  \texorpdfstring{$\alpha<0$}{α<0}}
\label{sec:repulsive-core}

Beyond stabilising isotropy (\S\ref{sec:weyl-penalty}), the $\alpha<0$
Weyl term generates a second important physical effect: a
\emph{repulsive core} at $r\to 0$.

For $\alpha<0$ and $\varepsilon\neq 0$:
\begin{equation}
  \Veff(r,\varepsilon;\alpha)
  = \VEC(r,\varepsilon)
  + |\alpha|\cdot
    \frac{2048\pi^2 L\,\varepsilon^2(\varepsilon{+}2)^2}
         {3\,r\,(1{+}\varepsilon)^{16/3}}\,.
\end{equation}
As $r\to 0$, the Weyl penalty $|\alpha|\,C^2\cdot\Vol\sim |\alpha|/r
\to +\infty$, forming an \emph{infinite barrier} against
volume collapse in anisotropic configurations.

This barrier carries the following physical implications:
\begin{enumerate}
  \item \textbf{Singularity avoidance}: it prevents collapse to $r=0$,
    potentially regularising the initial singularity of the early
    universe.

  \item \textbf{Physical motivation for $\alpha\le 0$}: $\alpha>0$
    triggers ghost instability (Theorem~\ref{thm:instability}), while
    $\alpha<0$ provides singularity regularisation---offering a
    physical rationale for Nature to select $\alpha\le 0$.

  \item \textbf{Natural extension of EC+NY theory}: adding a Weyl term
    with $\alpha\le 0$ may be viewed as a ``natural'' UV improvement
    of the theory.
\end{enumerate}

However, at the isotropic point ($\varepsilon=0$), $C^2=0$ and the
repulsive core is absent. Avoiding an isotropic singularity would
require a different mechanism (e.g.\ quantum effects, non-perturbative
corrections).

\subsection{Sign Constraint on \texorpdfstring{$\alpha$}{α}}
\label{sec:alpha-constraint}

Combining Theorems~\ref{thm:instability} and~\ref{thm:universality},
the consistency of the EC+NY+Weyl theory requires $\alpha\le 0$:

\begin{table}[ht]
\centering
\caption{Physical interpretation of the sign of $\alpha$.}
\label{tab:alpha-sign-physics}
\begin{tabularx}{\textwidth}{@{}lXX@{}}
\toprule
Sign of $\alpha$ & Potential structure & Physical interpretation \\
\midrule
$\alpha>0$ & Unbounded below (ghost) & Unstable---Weyl ghost \\
$\alpha=0$ & Paper~I recovered & Standard EC+NY theory \\
$\alpha<0$ & Isotropic vacuum stabilised + repulsive core
  & Conformal-curvature penalty + regularisation \\
\bottomrule
\end{tabularx}
\end{table}

This constraint is the minisuperspace counterpart of Ostrogradsky's
theorem and is consistent with known ghost-avoidance conditions in
higher-curvature gravity.

\subsection{Outlook for Future Extensions}
\label{sec:outlook}

An important corollary of Theorem~\ref{thm:stability} is that under
the constraint $\alpha\le 0$, the Weyl term vanishes at the isotropic
vacuum, and the effective potential reduces to $\VEC$. This motivates
the following strategy for future extensions involving shear degrees
of freedom:

\begin{enumerate}
  \item \textbf{Paper~I ansatz as baseline}: for $\alpha\le 0$ the
    isotropic vacuum is unaffected, so the paper~I starting point
    $(r,\varepsilon=0)$ remains a valid foundation.

  \item \textbf{Size--shape decoupling}: the coupling between $r$
    (size) and $\varepsilon$ (shape) occurs only through the Weyl term;
    at $\alpha=0$ they decouple, enabling independent introduction of
    shear degrees of freedom.

  \item \textbf{Weyl penalty on shear}: for general deformations beyond
    squashing (shear modes), $C^2$ will generically be non-zero, and
    the $\alpha<0$ Weyl penalty is expected to stabilise the vacuum
    against shear perturbations as well.
\end{enumerate}

\subsection{Limitations}
\label{sec:limitations}

\subsubsection{Minisuperspace approximation}
\label{sec:limit-minisuperspace}

All results are based on the $M^3\times\Sone$ product structure and
spatial homogeneity of the minisuperspace reduction. The following
effects are not captured:
\begin{itemize}
  \item Inhomogeneous modes (gravitational waves, density perturbations),
  \item Local defects (cosmic strings, domain walls),
  \item Degrees of freedom beyond the minisuperspace variables
    $(r,\varepsilon)$.
\end{itemize}
Analysis of Weyl stability in the full field theory, starting from the
minisuperspace results, is a natural next step.

\subsubsection{Lorentzian-signature extension}
\label{sec:limit-lorentz}

The present computation uses Euclidean signature $(+,+,+,+)$. Wick
rotation to Lorentzian signature $(-,+,+,+)$ and interpretation in
real-time cosmology is an important open problem. In particular:
\begin{itemize}
  \item The eigenspace structure of the Hodge dual changes, so the
    proof of $P=0$ (\S\ref{sec:chiral}) does not directly carry over.
  \item The stability structure of the Weyl term may also be
    signature-dependent.
\end{itemize}

\subsubsection{Relation to the full field theory}
\label{sec:limit-full}

The extent to which minisuperspace results generalise to the full
theory is a non-trivial question:
\begin{itemize}
  \item $P=0$ is a consequence of the $M^3\times\Sone$ product
    structure and the minisuperspace ansatz; it does not hold for
    generic four-dimensional manifolds.
  \item Weyl stability relies on the conformal flatness of
    isotropic~$\Sthree$; more general backgrounds require additional
    analysis.
\end{itemize}

\subsubsection{Rigorous proof of global minimality of
  \texorpdfstring{$\VEC$}{V\_EC}}
\label{sec:limit-global}

This paper combines an analytic proof of local stability (Theorem~1)
with extensive numerical exploration to provide evidence that the
$\Sthree\times\Sone$ isotropic vacuum is the \emph{de facto} global
minimum. A complete analytic proof of global minimality of $\VEC$
remains open due to the highly non-linear dependence of the effective
potential on $(r,\varepsilon)$.

\subsubsection{Vacuum structure and the ghost problem: geometric
  observations}
\label{sec:limit-ghost}

The stability established in \S\ref{sec:stability} for the homogeneous
vacuum does not automatically resolve the ghost problem in the full
theory including inhomogeneous fluctuations. Nevertheless, the present
model simultaneously possesses three geometric structures---conformal
flatness of the isotropic vacuum, chiral torsion background, and
positive background curvature---each of which may have non-trivial
implications for ghost avoidance.

Whether these structures provide an effective ghost-avoidance mechanism
requires second-order perturbation theory around the isotropic
$\Sthree$ vacuum in the $\alpha\le 0$ EC+NY+Weyl theory, which lies
beyond the scope of the present minisuperspace analysis.
