%==============================================================================
% Structural Robustness of Isotropic S³ Vacua in Einstein–Cartan
% Minisuperspace via Chiral Equilibrium and Weyl Stability
%==============================================================================
\documentclass[12pt,a4paper]{article}

%------------------------------------------------------------------------------
% Packages
%------------------------------------------------------------------------------

% Fonts / encoding (helps text extraction)
\usepackage[T1]{fontenc}
\usepackage{lmodern}
\usepackage{textcomp}
\usepackage[final]{microtype}

% Math
\usepackage{amsmath,amssymb,amsfonts,amsthm}
\usepackage{mathtools}
\usepackage{bm}

% Graphics and figures
\usepackage{graphicx}
\usepackage{float}
\usepackage{subcaption}

% Tables
\usepackage{booktabs}
\usepackage{tabularx}
\usepackage{array}
\usepackage{multirow}

% TikZ for diagrams
\usepackage{tikz}
\usetikzlibrary{shapes.geometric, arrows.meta, positioning, calc, fit, backgrounds}

% Layout and formatting
\usepackage[margin=2.5cm]{geometry}
\usepackage{setspace}
\onehalfspacing

% Misc (xcolor should come before hyperref)
\usepackage{xcolor}
\usepackage{enumitem}

% References and links
\definecolor{linkblue}{RGB}{0,0,180}
\usepackage[colorlinks=true,linkcolor=linkblue,citecolor=linkblue,urlcolor=linkblue]{hyperref}
\usepackage{cleveref}

% Code listings (for appendices)
\usepackage{listings}
\lstset{
  basicstyle=\ttfamily\small,
  breaklines=true,
  frame=single,
  language=Python
}

% section break
\usepackage{titlesec}
\newcommand{\sectionbreak}{\clearpage}

%------------------------------------------------------------------------------
% Theorem environments
%------------------------------------------------------------------------------
\newtheorem{theorem}{Theorem}
\newtheorem{proposition}{Proposition}
\newtheorem{lemma}{Lemma}
\newtheorem{corollary}{Corollary}
\theoremstyle{definition}
\newtheorem{definition}{Definition}
\theoremstyle{remark}
\newtheorem{remark}{Remark}

%------------------------------------------------------------------------------
% Custom commands
%------------------------------------------------------------------------------
% Math shortcuts
\newcommand{\dd}{\mathrm{d}}
\newcommand{\Veff}{V_{\mathrm{eff}}}
\newcommand{\VEC}{V_{\mathrm{EC}}}
\newcommand{\VWeyl}{V_{\mathrm{Weyl}}}
\newcommand{\Ric}{R}
\newcommand{\LC}{\mathrm{LC}}
\newcommand{\EC}{\mathrm{EC}}
\newcommand{\NY}{\mathrm{NY}}
\newcommand{\TT}{\mathrm{TT}}
\newcommand{\REE}{\mathrm{REE}}
\newcommand{\FULL}{\mathrm{FULL}}
\newcommand{\Vol}{\mathrm{Vol}}

% Topology notation
\newcommand{\Sthree}{S^3}
\newcommand{\Tthree}{T^3}
\newcommand{\Nilthree}{\mathrm{Nil}^3}
\newcommand{\Sone}{S^1}

% Differential forms
\newcommand{\wedgep}{\wedge}

% Inner product notation
\newcommand{\innerprod}[2]{\langle #1,\, #2 \rangle}

%------------------------------------------------------------------------------
% Title and authors
%------------------------------------------------------------------------------
\title{Structural Robustness of Isotropic $\Sthree$ Vacua\\
in Einstein--Cartan Minisuperspace\\
via Chiral Equilibrium and Weyl Stability}

\author{Muacca}

\date{}

%==============================================================================
\begin{document}
%==============================================================================

\maketitle
\begin{center}
Version: v1\\
First release: 2026-02-28\\
\end{center}
%------------------------------------------------------------------------------
% Abstract
%------------------------------------------------------------------------------
\begin{abstract}
We examine the robustness of the isotropic $\Sthree\times\Sone$ vacuum
established in our preceding work (paper~I) within the Einstein--Cartan +
Nieh--Yan (EC+NY) Euclidean minisuperspace framework. Two independent
extensions are analysed: a topological one (self-dual instanton decay
paths) and a dynamical one (higher-curvature Weyl-squared corrections).

First, we prove that the Pontryagin density $P=\innerprod{R}{*R}$
vanishes identically under the $M^3\times\Sone$ minisuperspace ansatz
with EC connection---a property we term \emph{chiral equilibrium}---thereby
closing the self-dual instanton decay channel within this framework
(Proposition~1).

Second, we introduce a Weyl-squared term $\alpha\,C^2$ and a
volume-preserving squashed ansatz parameterised by an anisotropy
variable~$\varepsilon$. For $\alpha\le 0$ the conformal flatness
$C^2=0$ of isotropic $\Sthree\times\Sone$ renders the Weyl term inert,
analytically protecting the global minimum of paper~I (Theorem~1). For
$\alpha>0$ the effective potential becomes unbounded below
(Theorem~2). The stability boundary $\alpha=0$ is sharp and independent
of the paper~I matter parameters $(V,\eta,\theta_{\NY})$, a consequence
of the geometric decoupling of the Weyl scalar from torsion
(Theorem~3).

A systematic comparison of three topologies ($\Sthree$, $\Tthree$,
$\Nilthree$) confirms that the topology-selection principle---the
energetic dominance of $\Sthree$---is preserved under the Weyl
extension.
\end{abstract}

%------------------------------------------------------------------------------
% Main text
%------------------------------------------------------------------------------
%==============================================================================
% Section 1: Introduction
%==============================================================================
\section{Introduction}
\label{sec:introduction}

\subsection{Background: Results of Paper~I and Two Open Questions}
\label{sec:intro-background}

In paper~I~\cite{paper1}, we analysed the Einstein--Cartan (EC) gravity
supplemented with the Nieh--Yan (NY) term~\cite{NiehYan1982} in a Euclidean-signature
minisuperspace framework and established a systematic phase classification of
the effective potential $\Veff(r)$. In particular, by comparing three spatial
topologies---$\Sthree$ (SU(2)), $\Tthree$ (flat), and $\Nilthree$
(Heisenberg group)---within a unified reduction procedure, we showed that the
$\Sthree\times\Sone$ configuration forms the energetically most favourable
stable vacuum over a wide region of parameter space: the
\emph{topology-selection principle}.

Two natural questions regarding the physical robustness of these results were
left open:

\begin{enumerate}
  \item \textbf{Topological stability.}
    Could self-dual instanton solutions exist on $\Sthree\times\Sone$,
    mediating vacuum decay via quantum tunnelling~\cite{Chandia1997}?

  \item \textbf{Dynamical stability.}
    Is the stability of the isotropic vacuum preserved under
    higher-curvature corrections to general relativity, notably the
    Weyl-squared term $\alpha\,C^2$?
\end{enumerate}

\subsection{Approach of the Present Work}
\label{sec:intro-approach}

In this paper we address both questions through independent analyses and
provide a comprehensive demonstration of the robustness of the
$\Sthree\times\Sone$ isotropic vacuum.

\paragraph{Topological stability.}
We show that the Pontryagin density
$P = \innerprod{R}{*R}$ vanishes identically under the
$M^3\times\Sone$ minisuperspace ansatz with EC connection. This
algebraic identity---reflecting what we call \emph{chiral equilibrium}
between self-dual and anti-self-dual curvature components---implies that
self-dual instanton solutions are forbidden within this framework.

\paragraph{Dynamical stability.}
We extend the EC+NY Lagrangian by a Weyl-squared term $\alpha\,C^2$ and
introduce a volume-preserving squashed ansatz with anisotropy
parameter~$\varepsilon$, constructing the two-variable effective potential
$\Veff(r,\varepsilon;\alpha)$. The stability structure is classified by the
sign of~$\alpha$:

\begin{itemize}
  \item \textbf{$\alpha \le 0$}: The conformal flatness $C^2=0$ of isotropic
    $\Sthree$ shields the vacuum from the Weyl term. For $\alpha<0$ the Weyl
    penalty on $\varepsilon\neq 0$ directions further stabilises isotropy.

  \item \textbf{$\alpha > 0$}: The Weyl term asymptotically dominates
    $\VEC$ as $r\to 0$, rendering the effective potential unbounded below
    ($\Veff\to -\infty$). This is the minisuperspace manifestation of the
    Ostrogradsky ghost instability of Weyl gravity.
\end{itemize}

\subsection{Main Results: Four Theoretical Pillars}
\label{sec:intro-results}

The principal results of this paper are summarised by the following
four statements:

\begin{enumerate}
  \item \textbf{Proposition~1 (Chiral equilibrium, $P=0$).}
    Under the $M^3\times\Sone$ minisuperspace ansatz with EC connection,
    the Pontryagin density satisfies
    $P = \innerprod{R}{*R} = 0$
    identically. Self-dual instanton decay paths are closed within this
    ansatz.

  \item \textbf{Theorem~1 (Weyl stability of the isotropic vacuum).}
    For $\alpha\le 0$, the isotropic vacuum of $\Sthree\times\Sone$ is
    shielded from the Weyl term by conformal flatness, and the global
    minimum of $\VEC$ is analytically protected.

  \item \textbf{Theorem~2 (Unbounded instability for $\alpha>0$).}
    For $\alpha>0$, $\Veff$ is unbounded below
    ($\inf\Veff = -\infty$); for $\alpha\le 0$ and bounded $\VEC$, $\Veff$
    is bounded below. Hence $\alpha=0$ is the sharp stability boundary.

  \item \textbf{Theorem~3 (Parameter independence of the stability boundary).}
    The stability boundary $\alpha=0$ is independent of the paper~I
    parameters $(V,\eta,\theta_{\NY})$, owing to the geometric decoupling
    of~$C^2$ from torsion.
\end{enumerate}

\subsection{Logical Flow}
\label{sec:intro-flow}

\begin{figure}[ht]
\centering
\includegraphics[width=0.85\textwidth]{figures/fig01_argument_flow.png}
\caption{Logical flow of the paper. Starting from the EC+NY isotropic
  $\Sthree\times\Sone$ vacuum (paper~I), the topological threat
  (Question~1) is addressed by Proposition~1 (\S\ref{sec:chiral}), while
  the dynamical threat (Question~2) is resolved by Theorems~1--3
  (\S\ref{sec:stability}--\S\ref{sec:universality}). The topology
  comparison (\S\ref{sec:topology}) confirms preservation of the
  $\Sthree$ dominance.}
\label{fig:argument-flow}
\end{figure}

\subsection{Organisation of the Paper}
\label{sec:intro-outline}

The remainder of this paper is organised as follows.

\begin{itemize}
  \item \textbf{\S\ref{sec:framework} (Framework)}: EC+NY+Weyl Lagrangian
    and the squashed ansatz.
  \item \textbf{\S\ref{sec:chiral} (Chiral Equilibrium)}: Analytic proof of
    $P=0$ (Proposition~1).
  \item \textbf{\S\ref{sec:weyl} (Weyl Extension)}: Closed-form Weyl scalar
    $C^2(r,\varepsilon)$ and effective potential structure.
  \item \textbf{\S\ref{sec:stability} (Stability, $\alpha\le 0$)}: Proof of
    Theorem~1.
  \item \textbf{\S\ref{sec:instability} (Instability, $\alpha>0$)}: Proof of
    Theorem~2.
  \item \textbf{\S\ref{sec:universality} (Universality)}: Proof of
    Theorem~3.
  \item \textbf{\S\ref{sec:topology} (Topology Comparison)}: Robustness of
    the topology-selection principle under Weyl extension.
  \item \textbf{\S\ref{sec:discussion} (Discussion)}: Repulsive core
    ($\alpha<0$), Lorentzian extension, limitations.
  \item \textbf{\S\ref{sec:conclusion} (Conclusion)}: Summary.
\end{itemize}

\noindent
The DPPUv2 computation engine v4 specification is given in
Appendix~\ref{app:engine}, numerical verification details in
Appendix~\ref{app:numerical}, and symbolic computation details in
Appendix~\ref{app:symbolic}.
   % Introduction
%==============================================================================
% Section 2: Framework
%==============================================================================
\section{Framework}
\label{sec:framework}

This section formulates the extended Lagrangian obtained by adding a
Weyl-squared term to the EC+NY theory of paper~I, and introduces the
volume-preserving squashed ansatz.

\subsection{EC+NY+Weyl Lagrangian}
\label{sec:lagrangian}

The extended Lagrangian studied in this paper is
\begin{equation}
  \mathcal{L}
  = \frac{R_{\EC}}{2\kappa^2}
  + \theta_{\NY}\,N
  + \alpha\,C^2\,,
  \label{eq:lagrangian}
\end{equation}
where each term is defined as follows:

\begin{itemize}
  \item $R_{\EC}/(2\kappa^2)$: the Ricci scalar computed from the
    Einstein--Cartan connection (including torsion), with gravitational
    coupling constant~$\kappa$.

  \item $\theta_{\NY}\,N$: the Nieh--Yan term, where
    $N = \dd(e^a\wedge T_a)$ is the Nieh--Yan density (a 4-form)
    and $\theta_{\NY}$ its coupling constant, following the notation
    of paper~I~\cite{paper1}.

  \item $\alpha\,C^2$: the Weyl-squared term, where
    $C^2 = C_{abcd}\,C^{abcd}$ is the Kretschner-type Weyl scalar
    computed from the Levi-Civita connection, and $\alpha$ is a
    dimensionless coupling constant.
\end{itemize}

The Weyl tensor is defined with respect to the Levi-Civita connection.
This choice avoids torsion contamination that would generically break
conformal invariance if the EC connection were used instead.
The quantities $R_{\EC}$ and $N$ are computed from the EC connection.

\paragraph{Sign conventions.}
We adopt the same conventions as paper~I:
\begin{itemize}
  \item Frame metric:
    $\eta_{ab}=\mathrm{diag}(+1,+1,+1,+1)$ (Euclidean signature).
  \item Riemann tensor:
    $R^{a}{}_{bcd} = \partial_c\Gamma^{a}{}_{bd}
    - \partial_d\Gamma^{a}{}_{bc}
    + \Gamma^{a}{}_{ec}\Gamma^{e}{}_{bd}
    - \Gamma^{a}{}_{ed}\Gamma^{e}{}_{bc}$.
  \item Contortion:
    $K_{abc} = \tfrac{1}{2}(T_{abc}+T_{bca}-T_{cab})$
    (Hehl et al.~\cite{Hehl1976}).
  \item Levi-Civita symbol: $\varepsilon_{0123}=+1$.
\end{itemize}

\subsection{Review of the $M^3\times\Sone$ Minisuperspace Ansatz}
\label{sec:ansatz-review}

Following paper~I, we decompose the four-dimensional Euclidean manifold as
$\mathcal{M}_4 = \mathcal{M}_3\times\Sone$, where $\mathcal{M}_3$ is a
compact quotient of a three-dimensional Lie group admitting a left-invariant
coframe $\{\sigma^i\}$ ($i=0,1,2$), and $\Sone$ has circumference~$L$.

In paper~I the isotropic coframe $e^a = r\,\sigma^a$ ($a=0,1,2$),
$e^3 = L\,\dd\tau$ was employed, with the scale variable~$r$ serving as
the principal argument of the effective potential $\Veff(r)$.

\subsection{Squashed Ansatz}
\label{sec:squashed}

To diagnose the effect of the Weyl term, a deformation parameter away
from isotropy is required. We introduce an axisymmetric,
volume-preserving squashing on~$\Sthree$:
\begin{equation}
  e^0 = r\,(1{+}\varepsilon)^{1/3}\,\sigma^0,\quad
  e^1 = r\,(1{+}\varepsilon)^{1/3}\,\sigma^1,\quad
  e^2 = r\,(1{+}\varepsilon)^{-2/3}\,\sigma^2,\quad
  e^3 = L\,\dd\tau\,,
  \label{eq:squashed-coframe}
\end{equation}
where
\begin{itemize}
  \item $r>0$ is the scale variable (identical to paper~I),
  \item $\varepsilon$ is the anisotropy parameter; $\varepsilon=0$ is the
    isotropic point and $\varepsilon>-1$ is required for physical
    significance:
    \begin{itemize}
      \item $\varepsilon>0$: oblate deformation ($e^0,e^1$ directions
        expand, $e^2$ contracts),
      \item $\varepsilon<0$: prolate deformation ($e^2$ expands, $e^0,e^1$
        contract),
      \item $\varepsilon=-1$: singular point (complete collapse of one
        direction).
    \end{itemize}
\end{itemize}

\paragraph{Volume preservation.}
The squashing factors satisfy
$(1{+}\varepsilon)^{1/3}\times(1{+}\varepsilon)^{1/3}
\times(1{+}\varepsilon)^{-2/3}=1$,
so that the four-volume is $\varepsilon$-independent:
\begin{equation}
  \Vol(\mathcal{M}_4) = 2\pi^2 L\,r^3
  \qquad (\text{independent of }\varepsilon).
  \label{eq:volume}
\end{equation}
This ensures that $\varepsilon$ modifies the \emph{shape} without
changing the \emph{size}~($r$).

\paragraph{Squashed structure constants.}
The squashing modifies the $\Sthree$ structure constants to
\begin{equation}
  C^i{}_{jk}(\varepsilon)
  = \frac{4}{r}\,\varepsilon_{ijk}\times f_i(\varepsilon),
  \label{eq:squashed-structure}
\end{equation}
with
\begin{equation}
  f_0(\varepsilon) = f_1(\varepsilon) = (1{+}\varepsilon)^{2/3},
  \qquad
  f_2(\varepsilon) = (1{+}\varepsilon)^{-4/3}.
  \label{eq:squash-factors}
\end{equation}
At the isotropic point $\varepsilon=0$, $f_0=f_1=f_2=1$ and one
recovers the paper~I structure constants.

\subsection{Notation Summary}
\label{sec:notation}

Table~\ref{tab:notation} collects the principal symbols used throughout
the paper.

\begin{table}[ht]
\centering
\caption{Principal symbols and their numerical scan ranges.}
\label{tab:notation}
\begin{tabular}{@{}lll@{}}
\toprule
Symbol & Meaning & Scan range \\
\midrule
$r$ & Scale variable & $[0.01,\,10]$ \\
$\varepsilon$ & Anisotropy parameter & $[-0.95,\,5.0]$ \\
$\alpha$ & Weyl coupling constant & $[-1,\,1]$ \\
$V$ & Vector-torsion amplitude & Fixed ($V=4$) \\
$\eta$ & Axial-torsion amplitude & Fixed ($\eta=-2$, varied in \S\ref{sec:universality}) \\
$\theta_{\NY}$ & Nieh--Yan coupling & Fixed ($\theta_{\NY}=1$) \\
$\kappa$ & Gravitational coupling & Fixed ($\kappa=1$) \\
$L$ & $\Sone$ circumference & Fixed ($L=1$) \\
\bottomrule
\end{tabular}
\end{table}

The paper~I parameters $(V,\eta,\theta_{\NY})$ are scanned in
\S\ref{sec:universality}; \S\ref{sec:weyl}--\ref{sec:instability} use
the reference values listed above.

\subsection{Separation of the Effective Potential}
\label{sec:separation}

Under the squashed ansatz, the effective potential separates
\emph{exactly linearly} in~$\alpha$:
\begin{equation}
  \Veff(r,\varepsilon;\alpha)
  = \VEC(r,\varepsilon)
  - \alpha\,C^2(r,\varepsilon)\cdot\Vol(r)\,.
  \label{eq:Veff-separation}
\end{equation}

Note the sign: since the Euclidean effective potential is defined as
$\Veff = -\mathcal{L}\times\Vol$ (see Appendix~\ref{app:engine},
\S\ref{sec:app-flow}), the $+\alpha\,C^2$ term in the
Lagrangian~\eqref{eq:lagrangian} enters as $-\alpha\,C^2\cdot\Vol$ in
the effective potential.

The two constituents are:
\begin{itemize}
  \item $\VEC(r,\varepsilon)$: the EC+NY effective potential, independent
    of~$\alpha$.
    \begin{equation}
      \VEC = -\biggl(\frac{R_{\EC}}{2\kappa^2}
      + \theta_{\NY}\,N\biggr)\times\Vol\,.
    \end{equation}

  \item $C^2(r,\varepsilon)\cdot\Vol(r)$: a purely geometric quantity,
    independent of the paper~I parameters $(V,\eta,\theta_{\NY})$.
\end{itemize}

This separation is the foundation of our analysis:
\begin{enumerate}
  \item $\VEC$ inherits the full content of paper~I.
  \item $C^2\cdot\Vol$ is a geometric invariant decoupled from the
    torsion parameters.
  \item The stability classification reduces to the sign of~$\alpha$.
\end{enumerate}

The computational engine (DPPUv2 Engine~v4)---including the derivation of
the Levi-Civita connection, construction of the EC connection, and
evaluation of the Weyl tensor---is documented in
Appendix~\ref{app:engine}.
   % Framework
%==============================================================================
% Section 3: Chiral Equilibrium
%==============================================================================
\section{Chiral Equilibrium: \texorpdfstring{$P=0$}{P=0}}
\label{sec:chiral}

In this section we prove that the Pontryagin density
$P=\innerprod{R}{*R}$ vanishes identically under the
$M^3\times\Sone$ minisuperspace ansatz with EC connection. This
eliminates the topological threat of vacuum decay through self-dual
instanton tunnelling within the present framework.

\subsection{Pontryagin Density and Self-Duality}
\label{sec:pontryagin}

On a four-dimensional Euclidean manifold, the Pontryagin density is
defined as the inner product of the curvature 2-form $R^{ab}$ with its
Hodge dual~$*R^{ab}$:
\begin{equation}
  P = \innerprod{R}{*R}
  = \tfrac{1}{4}\,\varepsilon^{abcd}\,R_{abef}\,R_{cd}{}^{ef}\,.
  \label{eq:pontryagin}
\end{equation}
$P$ is the density of the Pontryagin class and is related to
self-duality as follows:
\begin{itemize}
  \item $R=*R$ (self-dual) $\;\Leftrightarrow\;$ $P = E > 0$,
  \item $R=-*R$ (anti-self-dual) $\;\Leftrightarrow\;$ $P = -E < 0$,
\end{itemize}
where $E = \innerprod{R}{R} = R_{abcd}\,R^{abcd}$ is the scalar
related to the Euler (Gauss--Bonnet) density.

When (anti-)self-dual instanton solutions exist, they furnish
finite-action extrema that can mediate vacuum decay via quantum
tunnelling. Since $P\neq 0$ is a necessary condition for self-duality,
the identity $P=0$ excludes such solutions.

\subsection{Proposition~1 (Chiral Equilibrium)}
\label{sec:prop1}

\begin{proposition}[Chiral equilibrium]
\label{prop:chiral}
Under the $M^3\times\Sone$ minisuperspace ansatz with Einstein--Cartan
connection, the Pontryagin density vanishes identically:
\begin{equation}
  P = \innerprod{R}{*R} = 0\,.
  \label{eq:P-zero}
\end{equation}
This holds as an algebraic identity for:
\begin{itemize}
  \item all choices of $M^3$ ($\Sthree$, $\Tthree$, $\Nilthree$),
  \item all torsion modes (MX, AX, VT),
  \item all Nieh--Yan variants (FULL, TT, REE),
  \item all parameter values $(r,L,\eta,V,\kappa,\theta_{\NY})$.
\end{itemize}
\end{proposition}

\subsection{Proof}
\label{sec:proof-prop1}

\paragraph{Step~1: Orthogonal decomposition of 2-forms.}
Let $(x^0,x^1,x^2,\tau)$ denote coordinates on
$M^4 = M^3\times\Sone$. The space of 2-forms decomposes into two
mutually orthogonal blocks:
\begin{equation}
  \Lambda^2(M^4)
  = \underbrace{\Lambda^2(M^3)}_{\text{Spatial (S)}}
  \;\oplus\;
  \underbrace{\Lambda^1(M^3)\wedge\dd\tau}_{\text{Mixed (M)}}\,.
  \label{eq:2form-decomp}
\end{equation}
Explicitly, the spatial block~(S) consists of 2-forms with index pairs
$\{(01),(02),(12)\}$, and the mixed block~(M) of
$\{(03),(13),(23)\}$.

\paragraph{Step~2: The Hodge dual exchanges blocks.}
The four-dimensional Hodge dual (with $\varepsilon_{0123}=+1$) acts as
\begin{equation}
\begin{aligned}
  *(01) &= +(23), & *(03) &= +(12), \\
  *(02) &= -(13), & *(13) &= -(02), \\
  *(12) &= +(03), & *(23) &= +(01).
\end{aligned}
\label{eq:hodge-exchange}
\end{equation}
That is, the Hodge dual maps $\Lambda^2(M^3)$ to
$\Lambda^1(M^3)\wedge\dd\tau$ and vice versa.

\paragraph{Step~3: Curvature lies in the spatial block.}
Under the minisuperspace ansatz, the $\Sone$ direction~$\tau$ is a
Killing direction and the connection is spatially homogeneous. As a
consequence, the curvature 2-form $R^{ab}{}_{cd}$ has non-vanishing
lower-index pairs $(c,d)$ only in the spatial block:
$(c,d)\in\{(01),(02),(12)\}$. Components with $(c,d)$ in the mixed
block vanish identically.

This property has been verified by symbolic computation for all three
topologies (see \S\ref{sec:symbolic-verification} and
Appendix~\ref{app:symbolic}).

\paragraph{Step~4: Orthogonality implies $P=0$.}
Since $R$ has lower indices in the spatial block only, $*R$ has lower
indices in the mixed block only (Step~2). The two blocks are orthogonal,
hence
\begin{equation}
  \innerprod{R}{*R} = 0\,.
  \qquad\qed
\end{equation}

\subsection{One-Line Summary}
\label{sec:one-line}

The proof may be condensed to a single chain of implications:
\begin{equation}
  R \in \Lambda^2(M^3)
  \;\Rightarrow\;
  *R \in \Lambda^1(M^3)\wedge\dd\tau
  \;\Rightarrow\;
  \innerprod{R}{*R} = 0\,.
\end{equation}

\subsection{Symbolic Verification}
\label{sec:symbolic-verification}

Algebraic verification using symbolic computation software confirms
$P=0$ for all three topologies (full details in
Appendix~\ref{app:symbolic}):

\begin{table}[ht]
\centering
\caption{Symbolic verification of $P=0$ across topologies.}
\label{tab:P-zero}
\begin{tabular}{@{}lcccc@{}}
\toprule
Topology & Spatial components & Mixed components
  & $P=\innerprod{R}{*R}$ & Status \\
\midrule
$\Sthree\times\Sone$ & 6 & 0 & $\mathbf{0}$ & $\checkmark$ \\
$\Tthree\times\Sone$  & 6 & 0 & $\mathbf{0}$ & $\checkmark$ \\
$\Nilthree\times\Sone$ & 6 & 0 & $\mathbf{0}$ & $\checkmark$ \\
\bottomrule
\end{tabular}
\end{table}

Representative non-vanishing curvature components for
$\Sthree\times\Sone$:
\begin{equation}
  R^{01}{}_{01}
  = \frac{-V^2 r^2/9 - \eta^2 - 8\eta - 12}{r^2}\,,
  \qquad
  R^{03}{}_{12}
  = \frac{-2V(\eta+4)}{3r}\,.
  \label{eq:curvature-examples}
\end{equation}
All non-vanishing lower-index pairs $(c,d)$ belong to the spatial block
$\{01,02,12\}$, with no mixed-block components.

\subsection{Physical Interpretation}
\label{sec:chiral-interpretation}

\subsubsection{Chiral equilibrium}

The identity $P=0$ means that the self-dual and anti-self-dual
components of the curvature are balanced in norm. Decomposing
\begin{equation}
  R = R^+ + R^-\,,\qquad
  R^{\pm} = \tfrac{1}{2}(R \pm *R)\,,
\end{equation}
one finds
$P = \innerprod{R}{*R} = \|R^+\|^2 - \|R^-\|^2$, so that $P=0$
implies
\begin{equation}
  \|R^+\| = \|R^-\|\,.
  \label{eq:chiral-balance}
\end{equation}
Both chiralities are present in precisely equal measure; we call this
\emph{chiral equilibrium}.

\subsubsection{Structural prohibition of self-dual instantons}

Self-duality $R=*R$ requires $R^-=0$, i.e.\ $\|R^-\|=0$. By chiral
equilibrium, $\|R^-\|=0$ forces $\|R^+\|=0$ simultaneously, hence
$R=0$. Therefore, \emph{as long as the curvature is non-vanishing},
neither self-dual nor anti-self-dual solutions can exist on
$M^3\times\Sone$ within the minisuperspace ansatz.

The $\Sthree\times\Sone$ isotropic vacuum is thus protected from
instanton-mediated tunnelling decay. This protection is a
\emph{geometric consequence} of the product structure
$M^3\times\Sone$ and the minisuperspace ansatz, independent of
specific parameter values or torsion configurations.

\subsection{Conditions for Breakdown of \texorpdfstring{$P=0$}{P=0}}
\label{sec:P-breakdown}

Achieving $P\neq 0$ and thereby admitting self-dual solutions requires
extending beyond the present geometric setting:

\begin{table}[ht]
\centering
\caption{Extensions that can break $P=0$.}
\label{tab:P-breakdown}
\begin{tabularx}{\textwidth}{@{}llX@{}}
\toprule
Direction of extension & Modification & Why $P=0$ breaks \\
\midrule
Inhomogeneous torsion & Position-dependent $T^a(x)$
  & Breaks homogeneity; mixed-block curvature appears \\
Non-product manifold & Beyond $M^3\times\Sone$
  & Product structure underlies the block restriction \\
Lorentzian signature & $(-,+,+,+)$
  & Eigenspace structure of Hodge $*$ changes \\
Non-compact $M^3$ & Inhomogeneous geometry
  & Curvature structure may differ \\
\bottomrule
\end{tabularx}
\end{table}

These extensions are discussed as future directions in
\S\ref{sec:discussion}.
   % Chiral Equilibrium
%==============================================================================
% Section 4: Weyl Extension
%==============================================================================
\section{Weyl Extension: Squashed Ansatz}
\label{sec:weyl}

In this section we derive the closed-form expression for the Weyl scalar
$C^2(r,\varepsilon)$ under the squashed ansatz of \S\ref{sec:squashed}
and elucidate the structure of the effective potential.

\subsection{Computation of the Weyl Tensor}
\label{sec:weyl-computation}

The Weyl tensor is computed from the Levi-Civita connection via the
following standard procedure:

\begin{enumerate}
  \item \textbf{Levi-Civita connection} in the orthonormal frame
    (generalised Koszul formula):
    \begin{equation}
      \Gamma^{a}{}_{bc}
      = \tfrac{1}{2}\bigl(C^{a}{}_{bc} + C^{c}{}_{ba}
      - C^{b}{}_{ac}\bigr)\,.
    \end{equation}

  \item \textbf{Riemann tensor} in the frame basis:
    \begin{equation}
      R^{a}{}_{bcd}
      = \Gamma^{a}{}_{ec}\,\Gamma^{e}{}_{bd}
      - \Gamma^{a}{}_{ed}\,\Gamma^{e}{}_{bc}
      + \Gamma^{a}{}_{be}\,C^{e}{}_{cd}\,.
    \end{equation}

  \item \textbf{Ricci tensor and scalar}:
    $R_{bd} = R^a{}_{bad}$, $\;R = R^a{}_a$.

  \item \textbf{Weyl tensor} (4-dimensional definition):
    \begin{equation}
      C_{abcd} = R_{abcd}
      - \tfrac{1}{2}\bigl(g_{ac}R_{bd} - g_{ad}R_{bc}
      - g_{bc}R_{ad} + g_{bd}R_{ac}\bigr)
      + \tfrac{R}{6}\bigl(g_{ac}g_{bd} - g_{ad}g_{bc}\bigr)\,.
      \label{eq:weyl-def}
    \end{equation}

  \item \textbf{Weyl scalar}:
    $C^2 = C_{abcd}\,C^{abcd}$ (with $g^{ab}=\delta^{ab}$ in the
    orthonormal frame).
\end{enumerate}

\subsection{Lemma~1: Closed-Form Weyl Scalar}
\label{sec:lemma1}

\begin{lemma}
\label{lem:weyl-scalar}
The four-dimensional Weyl scalar of the squashed $\Sthree\times\Sone$
metric, computed from the Levi-Civita connection, is
\begin{equation}
  C^2(r,\varepsilon)
  = \frac{1024\,\varepsilon^2(\varepsilon{+}2)^2}
         {3\,r^4\,(1{+}\varepsilon)^{16/3}}\,.
  \label{eq:C2}
\end{equation}
\end{lemma}

\begin{proof}
By symbolic algebraic computation following the procedure of
\S\ref{sec:weyl-computation}, with the squashed structure
constants~\eqref{eq:squashed-structure} as input. All components of the
Weyl tensor are computed, and the sum of squares is simplified and
factored to yield~\eqref{eq:C2}. Full details are given in
Appendix~\ref{app:symbolic}.
\end{proof}

\subsection{Key Properties of Lemma~1}
\label{sec:C2-properties}

\paragraph{(i) Vanishing at the isotropic point: $C^2(r,0)=0$.}
The factor $\varepsilon^2$ in the numerator ensures $C^2=0$ at
$\varepsilon=0$. Physically, the round $\Sthree$ is a space of
constant curvature and hence conformally flat; the product metric
$\Sthree\times\Sone$ is therefore conformally flat in four dimensions,
with identically vanishing Weyl tensor.

\paragraph{(ii) Non-negativity: $C^2(r,\varepsilon)\ge 0$
  $(\varepsilon>-1)$.}
As a sum of squares in the orthonormal frame, $C^2\ge 0$ by
definition. The only zero (for $\varepsilon>-1$) is $\varepsilon=0$;
the other root $\varepsilon=-2$ lies beyond the singularity at
$\varepsilon=-1$.

\paragraph{(iii) Quadratic behaviour at the isotropic point.}
\begin{equation}
  \frac{\partial C^2}{\partial\varepsilon}\bigg|_{\varepsilon=0}=0,
  \qquad
  \frac{\partial^2 C^2}{\partial\varepsilon^2}\bigg|_{\varepsilon=0}
  = \frac{8192}{3r^4} > 0\,.
\end{equation}
Hence $C^2$ has a quadratic minimum at the isotropic point:
\begin{equation}
  C^2(r,\varepsilon)
  \approx \frac{8192}{3r^4}\,\varepsilon^2
  \qquad (|\varepsilon|\ll 1)\,.
\end{equation}

\paragraph{(iv) Divergence as $r\to 0$.}
For fixed $\varepsilon\neq 0$, $C^2\sim 1/r^4$. The product with the
volume~\eqref{eq:volume} gives
\begin{equation}
  C^2\cdot\Vol
  = \frac{2048\pi^2 L\,\varepsilon^2(\varepsilon{+}2)^2}
         {3\,r\,(1{+}\varepsilon)^{16/3}}
  \sim \frac{1}{r}
  \qquad (r\to 0)\,.
  \label{eq:C2Vol}
\end{equation}
This unboundedness is the key to the instability proof in
\S\ref{sec:instability}.

\subsection{Structure of Non-Vanishing Weyl Components}
\label{sec:weyl-components}

For general $\varepsilon$, there are 24 non-vanishing Weyl-tensor
components, all sharing the common factor
\begin{equation}
  C_{abcd} \propto
  \frac{\varepsilon(\varepsilon{+}2)}{r^2\,(1{+}\varepsilon)^{8/3}}\,.
\end{equation}
Representative independent components:

\begin{table}[ht]
\centering
\caption{Representative non-vanishing Weyl-tensor components.}
\label{tab:weyl-components}
\begin{tabular}{@{}ll@{}}
\toprule
Component & Value \\
\midrule
$C_{0101}$
  & $+\dfrac{16\,\varepsilon(\varepsilon{+}2)}
            {3r^2(1{+}\varepsilon)^{8/3}}$ \\[8pt]
$C_{0202}$, $C_{0303}$
  & $-\dfrac{8\,\varepsilon(\varepsilon{+}2)}
            {3r^2(1{+}\varepsilon)^{8/3}}$ \\[8pt]
$C_{2323}$
  & $+\dfrac{16\,\varepsilon(\varepsilon{+}2)}
            {3r^2(1{+}\varepsilon)^{8/3}}$ \\[8pt]
$C_{1212}$, $C_{1313}$
  & $-\dfrac{8\,\varepsilon(\varepsilon{+}2)}
            {3r^2(1{+}\varepsilon)^{8/3}}$ \\
\bottomrule
\end{tabular}
\end{table}

All components contain $\varepsilon(\varepsilon{+}2)$ as a factor and
vanish at $\varepsilon=0$. A complete listing is provided in
Appendix~\ref{app:symbolic}.

\subsection{Effective Potential with Explicit Weyl Contribution}
\label{sec:Veff-explicit}

Substituting the result of Lemma~1 into the
separation~\eqref{eq:Veff-separation} yields
\begin{equation}
  \Veff(r,\varepsilon;\alpha)
  = \VEC(r,\varepsilon)
  - \alpha\cdot
    \frac{2048\pi^2 L\,\varepsilon^2(\varepsilon{+}2)^2}
         {3\,r\,(1{+}\varepsilon)^{16/3}}\,.
  \label{eq:Veff-explicit}
\end{equation}

The effect of the Weyl term depends on the sign of~$\alpha$:

\begin{table}[ht]
\centering
\caption{Effect of the Weyl term $-\alpha\,C^2\cdot\Vol$ on
  $\varepsilon\neq 0$ configurations.}
\label{tab:alpha-sign}
\begin{tabular}{@{}lll@{}}
\toprule
Sign of $\alpha$ & Weyl contribution $-\alpha\,C^2\cdot\Vol$
  & Effect on $\varepsilon\neq 0$ \\
\midrule
$\alpha=0$ & $0$ & None (paper~I recovered) \\
$\alpha<0$ & $+|\alpha|\,C^2\cdot\Vol > 0$ & Positive penalty (stabilises isotropy) \\
$\alpha>0$ & $-\alpha\,C^2\cdot\Vol < 0$ & Negative reward (promotes anisotropy) \\
\bottomrule
\end{tabular}
\end{table}

\subsection{Effective Potential Contour Maps}
\label{sec:contour}

\begin{figure}[ht]
\centering
\includegraphics[width=\textwidth]{figures/fig02_landscape.png}
\caption{Contour maps of $\Veff$ in the $(r,\varepsilon)$ plane for
  $\Sthree$ at $\alpha=-10$, $0$, and $+10$. For $\alpha\le 0$ the
  isotropic vacuum ($\varepsilon=0$) forms a stable valley, while for
  $\alpha>0$ the valley is destroyed and the potential landscape
  collapses. The analytic foundations of this qualitative change are
  established in \S\ref{sec:stability} and \S\ref{sec:instability}.}
\label{fig:landscape}
\end{figure}
   % Weyl Extension
%==============================================================================
% Section 5: Stability (alpha <= 0)
%==============================================================================
\section{Stability of the Isotropic Vacuum (\texorpdfstring{$\alpha\le 0$}{α≤0})}
\label{sec:stability}

We now prove that for $\alpha\le 0$ the isotropic vacuum of
$\Sthree\times\Sone$ is analytically protected from the Weyl extension.

\subsection{Theorem~1 (Weyl Stability of the Isotropic Vacuum)}
\label{sec:thm1}

\begin{theorem}[Weyl stability]
\label{thm:stability}
Under the EC+NY+Weyl Lagrangian~\eqref{eq:lagrangian} on
$\Sthree\times\Sone$, the isotropic vacuum ($\varepsilon=0$) satisfies:

\begin{enumerate}[label=(\alph*)]
  \item The Weyl scalar vanishes identically at the isotropic point:
    \begin{equation}
      C^2(r,\varepsilon{=}0) = 0
      \qquad (\forall\,r>0)\,.
    \end{equation}

  \item For $\alpha\le 0$, the Weyl term reinforces (or leaves
    unaffected) the isotropic minimum of the effective potential:
    \begin{equation}
      \Veff(r,\varepsilon;\alpha)
      = \VEC(r,\varepsilon)
      + |\alpha|\,C^2(r,\varepsilon)\cdot\Vol(r)\,,
    \end{equation}
    where $|\alpha|\,C^2\cdot\Vol\ge 0$, with equality if and only if
    $\varepsilon=0$.

  \item As a corollary, if $\VEC$ attains a global minimum at
    $(r_0,0)$, then $\Veff$ also attains its global minimum at
    $(r_0,0)$ with the same minimum value $\VEC(r_0,0)$, for all
    $\alpha\le 0$.
\end{enumerate}
\end{theorem}

\subsection{Proof}
\label{sec:proof-thm1}

\paragraph{Part (a).}
Setting $\varepsilon=0$ in Lemma~\ref{lem:weyl-scalar}, the factor
$\varepsilon^2$ in the numerator of~\eqref{eq:C2} gives
$C^2(r,0)=0$. \qed

\paragraph{Part (b).}
For $\alpha\le 0$, write $-\alpha=|\alpha|\ge 0$. Then
\begin{equation}
  \Veff(r,\varepsilon;\alpha)
  = \VEC(r,\varepsilon) + |\alpha|\,C^2(r,\varepsilon)\cdot\Vol(r)\,.
\end{equation}
By Lemma~\ref{lem:weyl-scalar}~(ii), $C^2\ge 0$. Since
$\Vol(r)=2\pi^2 Lr^3>0$, it follows that
$|\alpha|\,C^2\cdot\Vol\ge 0$, with equality if and only if $C^2=0$,
i.e.\ $\varepsilon=0$. Hence:
\begin{itemize}
  \item $\Veff(r,\varepsilon;\alpha)\ge \VEC(r,\varepsilon)$
    for all $r,\varepsilon$;
  \item $\Veff(r,0;\alpha) = \VEC(r,0)$ for all $r$.
\end{itemize}
The Weyl term adds a positive penalty only in the $\varepsilon\neq 0$
directions and leaves the isotropic slice unchanged. \qed

\paragraph{Part (c).}
Suppose $\VEC$ has a global minimum at $(r_0,0)$:
\begin{equation}
  \VEC(r_0,0) \le \VEC(r,\varepsilon)
  \qquad \forall\,r,\varepsilon\,.
\end{equation}
For any $(r,\varepsilon)$:
\begin{equation}
  \Veff(r,\varepsilon;\alpha)
  = \VEC(r,\varepsilon) + |\alpha|\,C^2\cdot\Vol
  \ge \VEC(r,\varepsilon)
  \ge \VEC(r_0,0)\,.
\end{equation}
On the other hand:
\begin{equation}
  \Veff(r_0,0;\alpha)
  = \VEC(r_0,0) + |\alpha|\cdot 0\cdot\Vol
  = \VEC(r_0,0)\,.
\end{equation}
Hence $(r_0,0)$ is a global minimum of $\Veff$ with value
$\VEC(r_0,0)$.

Moreover, if the minimum of $\VEC$ is strict (attained only at
$(r_0,0)$), the Weyl penalty $|\alpha|\,C^2\cdot\Vol>0$ for
$\varepsilon\neq 0$ makes the minimum of $\Veff$ \emph{even more}
strictly isolated. \qed

\subsection{Assumptions of Theorem~1}
\label{sec:thm1-assumptions}

The conclusion of part~(c) relies on the premise that $\VEC$ has a
global minimum at $\varepsilon=0$. This premise is:
\begin{itemize}
  \item \textbf{Numerically verified}: the Stage~2D optimisation with
    reference parameters $(V{=}4,\,\eta{=}{-}2,\,\theta_{\NY}{=}1,\,
    \kappa{=}1,\,L{=}1)$ yields a minimum at $(r_0,\varepsilon) =
    (2.000,\,0)$ with $V_{\min}=-421.103$.
  \item \textbf{Partially supported analytically}: the positive
    definiteness of the $\varepsilon$-direction Hessian of $\VEC$ at
    the isotropic point provides a local-minimum proof. A complete
    analytic proof of global minimality remains an open problem.
\end{itemize}

\subsection{Numerical Verification}
\label{sec:thm1-numerics}

The predictions of Theorem~1 are compared with Stage~2D numerical
results:

\begin{table}[ht]
\centering
\caption{Theorem~1 predictions versus numerical results.}
\label{tab:thm1-verification}
\begin{tabularx}{\textwidth}{@{}lXl@{}}
\toprule
Prediction (from Theorem~1) & Numerical result & Agreement \\
\midrule
$V_{\min}$ constant for $\alpha\le 0$
  & $-421.103$ at all 201 points ($\alpha\in[-1,1]$), 11-digit match
  & $\checkmark$ \\
$r^*$ independent of $\alpha$
  & $r^*=2.000\pm 10^{-7}$ for all $\alpha\le 0$
  & $\checkmark$ \\
$\varepsilon^*$ independent of $\alpha$
  & $\varepsilon^*\approx 0$ ($10^{-8}$ precision) for all $\alpha\le 0$
  & $\checkmark$ \\
\bottomrule
\end{tabularx}
\end{table}

The 11-digit agreement provides strong numerical evidence for the
correctness of the theorem.

\subsection{Bifurcation Diagrams}
\label{sec:bifurcation}

\begin{figure}[ht]
\centering
\includegraphics[width=0.8\textwidth]{figures/fig03_bifurcation.png}
\caption{Bifurcation diagram $\varepsilon^*(\alpha)$ for $\Sthree$.
  For $\alpha\le 0$ the optimal anisotropy remains at
  $\varepsilon^*=0$; for $\alpha>0$ a bifurcation to
  $\varepsilon^*\neq 0$ occurs.}
\label{fig:bifurcation}
\end{figure}

\begin{figure}[ht]
\centering
\includegraphics[width=0.8\textwidth]{figures/fig04_size_stability.png}
\caption{Optimal scale $r^*(\alpha)$ for $\Sthree$. For $\alpha\le 0$,
  $r^*$ remains constant at $r^*=2.000$; for $\alpha>0$, $r^*$
  collapses to the search boundary.}
\label{fig:size-stability}
\end{figure}

\subsection{Physical Interpretation: Weyl Penalty Stabilises Isotropy}
\label{sec:weyl-penalty}

For $\alpha<0$, the Weyl term $|\alpha|\,C^2\cdot\Vol$ acts as a
\emph{penalty} on anisotropy ($\varepsilon\neq 0$). Near the isotropic
point, the Taylor expansion of \S\ref{sec:C2-properties}~(iii) gives
\begin{equation}
  \Delta\Veff
  \approx \frac{16384\pi^2\,|\alpha|\,L}{3r}\,\varepsilon^2\,.
\end{equation}
At $r=r^*=2$, $L=1$:
\begin{equation}
  \Delta\Veff \approx 26947\,|\alpha|\,\varepsilon^2\,.
\end{equation}
Even for $|\alpha|=1$ and a small anisotropy $\varepsilon=0.01$,
the penalty is $\Delta V\approx 2.7$. Larger $|\alpha|$ deepens the
isotropic valley and more strongly suppresses anisotropic fluctuations.

This stabilisation mechanism provides a clear physical picture: the
$\alpha<0$ Weyl term penalises conformal curvature ($C^2$), thereby
dynamically protecting spatial isotropy.
   % Stability (alpha <= 0)
%==============================================================================
% Section 6: Instability (alpha > 0)
%==============================================================================
\section{Instability for \texorpdfstring{$\alpha>0$}{α>0}}
\label{sec:instability}

We now prove that the effective potential is unbounded below for
$\alpha>0$ and establish that $\alpha=0$ is the sharp stability
boundary.

\subsection{Theorem~2 (Unbounded Instability)}
\label{sec:thm2}

\begin{theorem}[Unbounded instability]
\label{thm:instability}
For the EC+NY+Weyl effective potential on $\Sthree\times\Sone$:

\begin{enumerate}[label=(\alph*)]
  \item If $\alpha>0$, then $\Veff$ is unbounded below:
    \begin{equation}
      \inf_{r>0,\;\varepsilon>-1}\Veff(r,\varepsilon;\alpha) = -\infty\,.
    \end{equation}

  \item If $\alpha\le 0$ and $\VEC$ is bounded below, then $\Veff$ is
    also bounded below:
    \begin{equation}
      \inf_{r>0,\;\varepsilon>-1}\Veff(r,\varepsilon;\alpha) > -\infty\,.
    \end{equation}
\end{enumerate}

Hence $\alpha=0$ is the \emph{sharp boundary} between stability and
instability.
\end{theorem}

\subsection{Proof}
\label{sec:proof-thm2}

\paragraph{Part (a): Unboundedness for $\alpha>0$.}
Let $\alpha>0$ and fix any $\varepsilon_0\neq 0$ (e.g.\
$\varepsilon_0=-1/2$). The effective potential reads
\begin{equation}
  \Veff(r,\varepsilon_0;\alpha)
  = \VEC(r,\varepsilon_0)
  - \alpha\cdot
    \frac{2048\pi^2 L\,\varepsilon_0^2(\varepsilon_0{+}2)^2}
         {3\,r\,(1{+}\varepsilon_0)^{16/3}}\,.
\end{equation}
The asymptotic behaviour as $r\to 0^+$:
\begin{itemize}
  \item EC part: $\VEC(r,\varepsilon_0)\sim r \to 0$.
  \item Weyl part:
    $-\alpha\times\text{const}/r \to -\infty$.
\end{itemize}
Therefore
\begin{equation}
  \lim_{r\to 0^+}\Veff(r,\varepsilon_0;\alpha) = -\infty\,.
  \qquad\qed
\end{equation}

\paragraph{Part (b): Boundedness for $\alpha\le 0$.}
For $\alpha\le 0$:
\begin{equation}
  \Veff(r,\varepsilon;\alpha)
  = \VEC(r,\varepsilon) + |\alpha|\,C^2\cdot\Vol
  \ge \VEC(r,\varepsilon)\,.
\end{equation}
If $\VEC$ is bounded below ($\inf\VEC>-\infty$), then
\begin{equation}
  \inf\Veff \ge \inf\VEC > -\infty\,.
  \qquad\qed
\end{equation}

\subsection{Analytic Confirmation of the \texorpdfstring{$\VEC$}{V\_EC}
  Asymptotics}
\label{sec:VEC-asymptotics}

The key asymptotic estimate ``$\VEC\sim r$ as $r\to 0$'' used in part~(a)
is confirmed from the exact symbolic expression at the isotropic point
(paper~I, \S3.1.3):
\begin{equation}
  \VEC(r,0)
  = \frac{2\pi^2 Lr}{3\kappa^2}
    \bigl(V^2 r^2 + 6V\eta\kappa^2 r\,\theta_{\NY}
    + 9\eta^2 - 36\bigr)\,.
  \label{eq:VEC-exact}
\end{equation}
The individual $r$-scalings are:

\begin{table}[ht]
\centering
\caption{$r$-scaling of each term in $\VEC(r,0)$.}
\label{tab:r-scaling}
\begin{tabular}{@{}llc@{}}
\toprule
Term & Origin & Power of $r$ \\
\midrule
$V^2 r^2$ & Vector-torsion self-interaction & $r^3$ \\
$6V\eta\kappa^2 r\,\theta_{\NY}$ & Nieh--Yan cross term & $r^2$ \\
$9\eta^2-36$ & Curvature + axial torsion & $r^1$ \\
\bottomrule
\end{tabular}
\end{table}

The lowest power is $r^1$, so $\VEC\to 0$ as $r\to 0$ (bounded below).
The Weyl contribution scales as
\begin{equation}
  \VWeyl = -\alpha\,C^2\cdot\Vol \sim -\frac{\alpha}{r}
  \qquad (r\to 0)\,,
\end{equation}
giving a dominance ratio
\begin{equation}
  \frac{|\VWeyl|}{|\VEC|}
  \sim \frac{1/r}{r} = \frac{1}{r^2} \to \infty
  \qquad (r\to 0)\,.
\end{equation}
The Weyl term asymptotically dominates the EC part, and the
two-order gap in $r$-scaling is not reversed within the present
parameterisation.

\subsection{Analytic Linear Coefficient and Numerical Agreement}
\label{sec:linear-coeff}

The numerically observed linear behaviour $V_{\min}(\alpha)\approx
-K\times\alpha$ for $\alpha>0$ has its coefficient~$K$ predicted
analytically. Numerical optimisation converges to the search boundary
$(r_{\min},\varepsilon_{\min})=(0.01,-0.95)$, at which
\begin{equation}
  K(0.01,-0.95)
  = C^2(0.01,-0.95)\times\Vol(0.01)
  = 5.823\times 10^{12}\,.
\end{equation}

\begin{table}[ht]
\centering
\caption{Analytic prediction versus numerical result for the linear
  coefficient~$K$.}
\label{tab:linear-coeff}
\begin{tabular}{@{}lll@{}}
\toprule
Quantity & Analytic prediction & Numerical result \\
\midrule
$K$ & $5.823\times 10^{12}$ & $5.82\times 10^{12}$ (0.05\% accuracy) \\
\bottomrule
\end{tabular}
\end{table}

\textbf{Important caveat.} This linear coefficient $K$ is not a
physical quantity; it depends on the search boundary and is an
artefact of the finite scan range. The true physical conclusion is
$\inf\Veff=-\infty$ for $\alpha>0$; the finite $V_{\min}$ is merely
the minimum within the explored region.

\subsection{Direction of Instability}
\label{sec:instability-direction}

For $\alpha>0$, the instability drives the system simultaneously in two
directions:

\begin{table}[ht]
\centering
\caption{Instability directions for $\alpha>0$.}
\label{tab:instability-dir}
\begin{tabularx}{\textwidth}{@{}llXX@{}}
\toprule
Direction & Scaling & Geometric meaning & Physical meaning \\
\midrule
$r\to 0$
  & $C^2\cdot\Vol\sim 1/r$
  & Volume collapse of $\Sthree$
  & Contraction / singularity \\
$\varepsilon\to -1$
  & $C^2\cdot\Vol\sim 1/\delta^{16/3}$
  & Extreme uniaxial 
  & Complete breakdown \\
  \quad & \quad & \quad deformation & \quad of 3d isotropy \\[4pt]
\bottomrule
\end{tabularx}
\end{table}

\noindent
Here $\delta=1+\varepsilon\to 0^+$. Both directions proceed
simultaneously, and no lower bound on $\Veff$ exists.

\subsection{Why \texorpdfstring{$\alpha=0$}{α=0} Is the Sharp Boundary}
\label{sec:sharp-boundary}

The mathematical reason that $\alpha=0$ is \emph{exactly} the boundary
is the \emph{unboundedness} of $C^2\cdot\Vol$.

If $C^2\cdot\Vol$ were bounded ($\sup C^2\cdot\Vol = M<\infty$), there
would be room for stability up to some critical $\alpha_c>0$
(satisfying $\alpha_c M\lesssim|\inf\VEC|$). However, since
$C^2\cdot\Vol\sim 1/r$ diverges as $r\to 0$, any $\alpha>0$---no
matter how small---is sufficient to make $\Veff\to -\infty$. The
transition therefore occurs precisely and sharply at $\alpha=0$.

\subsection{Relation to Ghost Instability of Weyl Gravity}
\label{sec:ghost}

The $\alpha>0$ instability is the minisuperspace manifestation of the
well-known ghost instability of conformal (Weyl) gravity. The generic
Weyl-gravity action
\begin{equation}
  S_{\text{Weyl}}
  = \int\dd^4 x\,\sqrt{g}\;\alpha\,C_{\mu\nu\rho\sigma}\,
    C^{\mu\nu\rho\sigma}
\end{equation}
yields fourth-order equations of motion, which by Ostrogradsky's
theorem~\cite{Woodard2015} generically lead to an energy unbounded
below (ghost instability).

In our minisuperspace calculation, this instability manifests concretely
as collapse towards $r\to 0$ and maximal anisotropy $\varepsilon\to -1$.
The condition $\alpha\le 0$ for stability is the minisuperspace
counterpart of the ghost-avoidance condition in Weyl gravity.

\subsection{Three-Topology Comparison:
  \texorpdfstring{$V_{\min}(\alpha)$}{Vmin(α)}}
\label{sec:Vmin-comparison}

\begin{figure}[ht]
\centering
\includegraphics[width=0.85\textwidth]{figures/fig05_topology_comparison.png}
\caption{$V_{\min}(\alpha)$ for the three topologies $\Sthree$, $\Tthree$,
  and $\Nilthree$. The stability transition at $\alpha=0$ is most
  dramatic for $\Sthree$, while $\Tthree$ provides a null test.
  The topology comparison is discussed in detail in
  \S\ref{sec:topology}.}
\label{fig:topology-comparison}
\end{figure}
   % Instability (alpha > 0)
%==============================================================================
% Section 7: Universality across paper I Parameters
%==============================================================================
\section{Universality across Paper~I Parameters}
\label{sec:universality}

We show that the stability boundary $\alpha=0$ is independent of the
paper~I parameters $(V,\eta,\theta_{\NY})$ and verify this numerically.

\subsection{Theorem~3 (Parameter Independence of the Stability
  Boundary)}
\label{sec:thm3}

\begin{theorem}[Parameter independence]
\label{thm:universality}
For any paper~I parameters $(V,\eta,\theta_{\NY})$ for which $\VEC$ is
bounded below, $\alpha=0$ is the stability boundary:
\begin{itemize}
  \item $\alpha>0$: $\inf\Veff = -\infty$ (unbounded below);
  \item $\alpha\le 0$: $\inf\Veff > -\infty$ (bounded below).
\end{itemize}
This boundary does not depend on the values of
$(V,\eta,\theta_{\NY})$.
\end{theorem}

\subsection{Proof: Geometric Decoupling}
\label{sec:proof-thm3}

Theorem~3 is a structural corollary of the proof of Theorem~2. The
argument rests on two key properties:

\paragraph{(i) Geometric decoupling.}
The Weyl scalar $C^2(r,\varepsilon)$ is computed entirely from the
Levi-Civita connection and is a purely geometric quantity independent
of the paper~I parameters $(V,\eta,\theta_{\NY})$:
\begin{equation}
  C^2(r,\varepsilon)
  = \frac{1024\,\varepsilon^2(\varepsilon{+}2)^2}
         {3\,r^4\,(1{+}\varepsilon)^{16/3}}\,.
\end{equation}
No torsion amplitude $\eta$, $V$, no Nieh--Yan coupling
$\theta_{\NY}$, and no gravitational coupling $\kappa$ appears. $C^2$
is a function of $(r,\varepsilon)$ alone.

\paragraph{(ii) Asymptotic dominance.}
The $r\to 0$ scaling comparison:
\begin{itemize}
  \item $\VEC(r,\varepsilon)\sim r$ ($r\to 0$, $\varepsilon$ fixed):
    the coefficient depends on paper~I parameters, but the leading
    power is uniformly $r^1$.
  \item $\VWeyl = -\alpha\,C^2\cdot\Vol \sim -\alpha/r$
    ($r\to 0$, $\varepsilon\neq 0$ fixed): independent of paper~I
    parameters.
\end{itemize}
The dominance ratio $|\VWeyl|/|\VEC|\sim 1/r^2\to\infty$ diverges
regardless of the values of $(V,\eta,\theta_{\NY})$.

\paragraph{Completion of the proof.}
The proof of Theorem~\ref{thm:instability}(a) does not depend on the
specific form of $\VEC$ (i.e.\ on the paper~I parameters). The
divergence $C^2\cdot\Vol\sim 1/r$ is geometrically guaranteed, so for
any $\alpha>0$, $\Veff\to -\infty$. Part~(b) requires only that
$\VEC$ be bounded below, without reference to its specific lower
bound.

Therefore, under the condition that $\VEC$ admits a stable vacuum
(Type~I or Type~II in the paper~I classification), the stability
boundary $\alpha=0$ holds irrespective of the paper~I parameters.
\qed

\subsection{Numerical Verification}
\label{sec:thm3-numerics}

The parameter independence is verified at four representative points
spanning the paper~I parameter space. In all cases $V=4$,
$\theta_{\NY}=1$, $\kappa=1$, $L=1$ are fixed, and $\eta$ is varied
to traverse different paper~I phases.

\begin{table}[ht]
\centering
\caption{Numerical verification of Theorem~3 across paper~I parameter
  space.}
\label{tab:thm3-verification}
\begin{tabular}{@{}lllrrl@{}}
\toprule
Parameter set & $\eta$ & Paper~I class & $V_{\min}$ ($\alpha\le 0$) & $r^*$
  & $\alpha=0$ boundary \\
\midrule
Type~I centre  & $-3.0$ & Type~I          & $-583$  & $2.65$ & Sharp \\
I/II boundary  & $-2.0$ & I/II boundary   & $-421$  & $2.00$ & Sharp \\
Type~II centre & $\phantom{-}0.0$ & Type~II & $-137$  & $0.87$ & Sharp \\
II/III boundary & $+2.0$ & II/III boundary & $+0.03$ & $0.01$ & Sharp \\
\bottomrule
\end{tabular}
\end{table}

\subsubsection{Type~I ($\eta=-3.0$)}
A deep stable vacuum at $V_{\min}\approx -583$, $r^*\approx 2.65$.
Immediate destabilisation upon $\alpha>0$. Even the most stable vacuum
in the paper~I landscape is susceptible to the Weyl instability for
$\alpha>0$.

\subsubsection{I/II boundary ($\eta=-2.0$)}
The reference parameter set, consistent with the detailed analysis of
\S\ref{sec:stability}--\ref{sec:instability}.

\subsubsection{Type~II ($\eta=0.0$)}
A clear stable vacuum at $r^*\approx 0.866$,
$V_{\min}\approx -137$, with a sharp stability boundary at
$\alpha=0$. Note that $\eta=0$ corresponds to zero axial torsion;
stability is maintained by vector torsion~$V$ and the Nieh--Yan
coupling~$\theta_{\NY}$ alone.

\subsubsection{II/III boundary ($\eta=2.0$)}
$V_{\min}\approx +0.03$ (positive), with $r^*$ at the search boundary.
This lies near the paper~I Type~III transition, where the premise
``$\VEC$ is bounded below with a stable vacuum'' barely holds.
Nevertheless, the sharp transition at $\alpha=0$ is observed.

\subsection{Visual Evidence of Parameter Independence}
\label{sec:facet-plot}

\begin{figure}[ht]
\centering
\includegraphics[width=\textwidth]{figures/fig06_gamma2_facet.png}
\caption{$V_{\min}(\alpha)$ for four paper~I parameter sets spanning
  the full range from Type~I centre to II/III boundary. In all cases
  $\alpha=0$ acts as the bifurcation point, providing direct visual
  evidence for Theorem~3.}
\label{fig:facet-plot}
\end{figure}

\subsection{Physical Interpretation of Parameter Independence}
\label{sec:param-indep-interp}

The parameter independence of the $\alpha=0$ boundary originates from
three structural facts:
\begin{enumerate}
  \item \textbf{Weyl tensor as a ``shape'' quantity.}
    $C^2$ depends only on the conformal structure $(r,\varepsilon)$ and
    is independent of the torsion amplitudes and the NY coupling
    $(V,\eta,\theta_{\NY})$.

  \item \textbf{Exact $\alpha$-linearity.}
    $\Veff = \VEC - \alpha\,C^2\cdot\Vol$ is strictly linear in
    $\alpha$; the critical value of $\alpha$ is determined solely by
    the bounded/unbounded character of $\VEC$ and $C^2\cdot\Vol$.

  \item \textbf{Unboundedness of $C^2\cdot\Vol$.}
    The divergence $C^2\cdot\Vol\sim 1/r$ as $r\to 0$ is a geometric
    property independent of the paper~I parameters.
\end{enumerate}
Together, these ensure that the stability boundary $\alpha=0$ cannot be
shifted by tuning the paper~I parameters.
   % Universality
%==============================================================================
% Section 8: Topology Comparison under Weyl Extension
%==============================================================================
\section{Topology Comparison under Weyl Extension}
\label{sec:topology}

We now verify that the topology-selection principle of paper~I---the
energetic dominance of $\Sthree$---is preserved under the Weyl
extension. The three topologies $\Sthree$, $\Tthree$, and $\Nilthree$
are compared systematically.

\subsection{\texorpdfstring{$\alpha\le 0$}{α≤0}: Preservation of
  \texorpdfstring{$\Sthree$}{S³} Dominance}
\label{sec:topo-neg-alpha}

For $\alpha\le 0$, the lowest-energy vacua of the three topologies are:

\begin{table}[ht]
\centering
\caption{Topology comparison for $\alpha\le 0$ (reference parameters).}
\label{tab:topo-neg}
\begin{tabularx}{\textwidth}{@{}lrrlX@{}}
\toprule
Topology & $V_{\min}$ & $r^*$ & $\varepsilon^*$ & Status \\
\midrule
$\Sthree$   & $-421$   & $2.000$ & $0$
  & Deep stable vacuum (lowest energy) \\
$\Tthree$   & $\approx 0$ ($5.5\times 10^{-12}$) & $1.500$ & $-0.642$
  & Effectively flat \\
$\Nilthree$ & $\approx 5$ (flat-limit asymptote) & $1.500$
  & $\to\infty$ (flat limit) & Energetically unfavourable \\
\bottomrule
\end{tabularx}
\end{table}

$\Sthree$ achieves $V_{\min}=-421$, significantly the lowest energy,
confirming that the topology-selection principle is preserved under the
Weyl extension.

By Theorem~\ref{thm:stability}, $V_{\min}$ of $\Sthree$ is constant
(independent of~$\alpha$) for all $\alpha\le 0$. $\Tthree$ also has
$V_{\min}\approx 0$ independent of $\alpha$ (confirmed by the null
test below). $\Nilthree$ asymptotes to $V_{\min}\approx 5$ with weak
$\alpha$-dependence (see Appendix~\ref{app:nil3}).

\subsection{\texorpdfstring{$\alpha>0$}{α>0}: Loss of the Dominance
  Concept}
\label{sec:topo-pos-alpha}

For $\alpha>0$, the situation changes qualitatively:

\begin{table}[ht]
\centering
\caption{Topology comparison for $\alpha>0$.}
\label{tab:topo-pos}
\begin{tabularx}{\textwidth}{@{}llX@{}}
\toprule
Topology & $V_{\min}$ ($\alpha>0$) & Behaviour \\
\midrule
$\Sthree$ & $\to -\infty$ (linear divergence) & Unstable ($r\to 0$,
  $\varepsilon\to -1$) \\
$\Tthree$ & $\approx 0$ ($\alpha$-independent) & Stable but
  $V_{\min}\approx 0$ ($C^2=0$ $\Rightarrow$ Weyl term vanishes) \\
$\Nilthree$ & $\to -\infty$ (slower divergence than $\Sthree$)
  & Unstable \\
\bottomrule
\end{tabularx}
\end{table}

Both $\Sthree$ and $\Nilthree$ become unstable ($\Veff\to -\infty$).
$\Tthree$ remains stable but with $V_{\min}\approx 0$, offering no
deep stable vacuum. For $\alpha>0$ the concept of ``topology
selection'' itself becomes vacuous: no topology provides a physically
viable stable vacuum.

\subsection{\texorpdfstring{$\Tthree$}{T³} Null Test}
\label{sec:null-test}

$\Tthree\times\Sone$ provides a null test for all values of $\alpha$.
Since $\Tthree$ is flat, $C^2=0$ identically, and the Weyl term
$\alpha\,C^2$ vanishes; consequently, the effective potential should
exhibit no $\alpha$-dependence whatsoever.

\paragraph{Numerical verification.}
At all 201 points ($\alpha\in[-1,1]$, step~0.01):

\begin{table}[ht]
\centering
\caption{$\Tthree$ null-test results.}
\label{tab:null-test}
\begin{tabular}{@{}llc@{}}
\toprule
Quantity & Value & $\alpha$-dependence \\
\midrule
$r^*$ & $1.500$ & None (identical at all points) \\
$\varepsilon^*$ & $-0.642$ & None (identical at all points) \\
$V_{\min}$ & $5.54\times 10^{-12}$ & None (machine precision) \\
\bottomrule
\end{tabular}
\end{table}

The maximum deviation
$\max|\Veff(\alpha)-\Veff(0)| = 0.00$ (machine precision) confirms the
null test. This serves a dual purpose:
\begin{enumerate}
  \item \textbf{Engine validation}: the $\alpha$ implementation is
    correct---$C^2=0$ topologies produce exactly zero Weyl contribution.
  \item \textbf{Theoretical consistency}: the flatness of $\Tthree$ and
    the conformal invariance of the Weyl term are correctly coupled.
\end{enumerate}

\begin{figure}[ht]
\centering
\includegraphics[width=0.8\textwidth]{figures/fig07_t3_null_test.png}
\caption{$\Tthree$ null-test visualisation, confirming
  $\alpha$-independence of the $\Tthree$ effective potential at machine
  precision.}
\label{fig:null-test}
\end{figure}

\paragraph{On the non-zero $\varepsilon^*$ of $\Tthree$.}
The optimal anisotropy $\varepsilon^*\approx -0.642$ is non-isotropic,
but the potential landscape is extremely flat
($V_{\min}\approx 0$), so this ``minimum'' carries limited physical
significance. The near-zero energy cost for deformation reflects
the structural flexibility of~$\Tthree$.

\subsection{\texorpdfstring{$\Nilthree$}{Nil³} Behaviour: Asymptotic
  Approach to the Flat Limit}
\label{sec:nil3-behaviour}

$\Nilthree$ exhibits qualitatively distinct behaviour from the other
two topologies. An extended search over $\varepsilon\in[-0.95,5.0]$
(see Appendix~\ref{app:nil3}) reveals:
\begin{itemize}
  \item \textbf{No global minimum at finite $\varepsilon^*$}:
    For each fixed $\varepsilon$, $\Veff(r)$ possesses a local minimum
    in $r$ ($\Tthree$-like profile), but the minimised value
    $V_{\min}(\varepsilon)\equiv\min_r\Veff(r,\varepsilon)$ decreases
    monotonically in $\varepsilon$ with no interior minimum at finite
    $\varepsilon$; it asymptotes to the flat limit ($\Tthree$-like
    behaviour) as $\varepsilon\to\infty$.

  \item \textbf{Asymptotic $V_{\min}\approx 5 > 0$}: weakly
    $\alpha$-dependent, and energetically unfavourable compared to
    $\Sthree$ ($V_{\min}=-421$).

  \item \textbf{Physical interpretation}: the $\Nilthree$ structure
    constants contain the squashing factor $(1{+}\varepsilon)^{-4/3}$,
    which vanishes as $\varepsilon\to\infty$. The effective potential
    minimises the curvature cost ($C^2>0$) arising from the Heisenberg
    group structure by driving the structure constants to zero.
\end{itemize}

This result is consistent with the central theme of this paper:
conformally flat configurations ($C^2=0$) are energetically favoured.
$\Sthree$ achieves $C^2=0$ exactly at the finite deformation
$\varepsilon=0$, while $\Nilthree$ can only approach $C^2\to 0$
asymptotically in the $\varepsilon\to\infty$ limit---a fundamental
geometric asymmetry.

The universality of the $\alpha=0$ stability boundary for $\Nilthree$
is confirmed by high-resolution scans (Appendix~\ref{app:nil3-alpha}).

\subsection{Summary of the Topology Comparison}
\label{sec:topo-summary}

\begin{table}[ht]
\centering
\caption{Three-topology comparison summary (representative values at
  $\alpha\le 0$).}
\label{tab:topo-summary}
\begin{tabularx}{\textwidth}{@{}lllX@{}}
\toprule
Quantity & $\Sthree$ & $\Tthree$ & $\Nilthree$ \\
\midrule
$V_{\min}$ & $-421$ & $\approx 0$ & $\approx 5$ (flat-limit asymptote) \\
$r^*$ & $2.000$ & $1.500$ & $1.500$ \\
$\varepsilon^*$ & $0$ & $-0.642$ & $\to\infty$ (flat limit) \\
$C^2(\varepsilon^*)$ & $0$ & $0$ & $\to 0$ (asymptotically) \\
$\alpha$-dependence & None (Thm~1) & None ($C^2=0$) & Weak \\
Paper~I dominance & \textbf{Lowest energy} & Neutral & Unfavourable \\
Weyl-extended dominance & \textbf{Preserved} & Unchanged
  & No global minimum at finite $\varepsilon^*$ (asymptotes to flat limit) \\
\bottomrule
\end{tabularx}
\end{table}

The topology-selection principle of paper~I is preserved under the Weyl
extension for $\alpha\le 0$. $\Sthree$ achieves $C^2=0$ exactly at
$\varepsilon^*=0$ and forms a stable isotropic vacuum, whereas
$\Nilthree$ has no global minimum at finite $\varepsilon^*$ and can
only approach the flat limit asymptotically. $\Tthree$ is unaffected by the Weyl term ($C^2=0$
identically) but has $V_{\min}\approx 0$.

The mathematical root $\varepsilon=-2$ of $C^2=0$ lies beyond the
physical singularity at $\varepsilon=-1$ and is therefore excluded from
discussion as physically inaccessible. Note that $\Tthree$ has
$C^2=0$ identically everywhere, and $\Nilthree$ approaches $C^2\to 0$
only in the asymptotic limit $\varepsilon\to\infty$; hence the existence
of a finite root of $C^2=0$ in the region $\varepsilon<-1$ is a
peculiarity of the parametric representation of~$\Sthree$.
   % Topology Comparison
%==============================================================================
% Section 9: Discussion
%==============================================================================
\section{Discussion}
\label{sec:discussion}

\subsection{Robustness of Paper~I Results}
\label{sec:robustness}

This paper has examined two independent threats to the
$\Sthree\times\Sone$ isotropic vacuum discovered in paper~I and
demonstrated its robustness against both.

\paragraph{Topological threat (self-dual instantons).}
Proposition~\ref{prop:chiral} establishes that the Pontryagin density
$P=\innerprod{R}{*R}=0$ vanishes identically under the
$M^3\times\Sone$ minisuperspace ansatz with EC connection. This
precludes self-dual instanton solutions within this framework and
eliminates vacuum decay via instanton-mediated quantum tunnelling.

\paragraph{Dynamical threat (higher-curvature corrections).}
Theorems~\ref{thm:stability}--\ref{thm:universality} show that the
$\alpha\le 0$ Weyl extension leaves the paper~I phase diagram
unaffected, under the stated assumptions. The conformal flatness
$C^2=0$ of $\Sthree\times\Sone$ causes the Weyl contribution to
vanish at the isotropic vacuum, leaving $V_{\min}$, $r^*$, and
$\varepsilon^*=0$ invariant throughout $\alpha\le 0$.

These two protection mechanisms are mutually independent, resting on
distinct mathematical structures (topological orthogonality and
conformal flatness). Their combination provides strong evidence for the
physical robustness of the $\Sthree\times\Sone$ isotropic vacuum.

\subsection{Repulsive Core (Regularisation) from
  \texorpdfstring{$\alpha<0$}{α<0}}
\label{sec:repulsive-core}

Beyond stabilising isotropy (\S\ref{sec:weyl-penalty}), the $\alpha<0$
Weyl term generates a second important physical effect: a
\emph{repulsive core} at $r\to 0$.

For $\alpha<0$ and $\varepsilon\neq 0$:
\begin{equation}
  \Veff(r,\varepsilon;\alpha)
  = \VEC(r,\varepsilon)
  + |\alpha|\cdot
    \frac{2048\pi^2 L\,\varepsilon^2(\varepsilon{+}2)^2}
         {3\,r\,(1{+}\varepsilon)^{16/3}}\,.
\end{equation}
As $r\to 0$, the Weyl penalty $|\alpha|\,C^2\cdot\Vol\sim |\alpha|/r
\to +\infty$, forming an \emph{infinite barrier} against
volume collapse in anisotropic configurations.

This barrier carries the following physical implications:
\begin{enumerate}
  \item \textbf{Singularity avoidance}: it prevents collapse to $r=0$,
    potentially regularising the initial singularity of the early
    universe.

  \item \textbf{Physical motivation for $\alpha\le 0$}: $\alpha>0$
    triggers ghost instability (Theorem~\ref{thm:instability}), while
    $\alpha<0$ provides singularity regularisation---offering a
    physical rationale for Nature to select $\alpha\le 0$.

  \item \textbf{Natural extension of EC+NY theory}: adding a Weyl term
    with $\alpha\le 0$ may be viewed as a ``natural'' UV improvement
    of the theory.
\end{enumerate}

However, at the isotropic point ($\varepsilon=0$), $C^2=0$ and the
repulsive core is absent. Avoiding an isotropic singularity would
require a different mechanism (e.g.\ quantum effects, non-perturbative
corrections).

\subsection{Sign Constraint on \texorpdfstring{$\alpha$}{α}}
\label{sec:alpha-constraint}

Combining Theorems~\ref{thm:instability} and~\ref{thm:universality},
the consistency of the EC+NY+Weyl theory requires $\alpha\le 0$:

\begin{table}[ht]
\centering
\caption{Physical interpretation of the sign of $\alpha$.}
\label{tab:alpha-sign-physics}
\begin{tabularx}{\textwidth}{@{}lXX@{}}
\toprule
Sign of $\alpha$ & Potential structure & Physical interpretation \\
\midrule
$\alpha>0$ & Unbounded below (ghost) & Unstable---Weyl ghost \\
$\alpha=0$ & Paper~I recovered & Standard EC+NY theory \\
$\alpha<0$ & Isotropic vacuum stabilised + repulsive core
  & Conformal-curvature penalty + regularisation \\
\bottomrule
\end{tabularx}
\end{table}

This constraint is the minisuperspace counterpart of Ostrogradsky's
theorem and is consistent with known ghost-avoidance conditions in
higher-curvature gravity.

\subsection{Outlook for Future Extensions}
\label{sec:outlook}

An important corollary of Theorem~\ref{thm:stability} is that under
the constraint $\alpha\le 0$, the Weyl term vanishes at the isotropic
vacuum, and the effective potential reduces to $\VEC$. This motivates
the following strategy for future extensions involving shear degrees
of freedom:

\begin{enumerate}
  \item \textbf{Paper~I ansatz as baseline}: for $\alpha\le 0$ the
    isotropic vacuum is unaffected, so the paper~I starting point
    $(r,\varepsilon=0)$ remains a valid foundation.

  \item \textbf{Size--shape decoupling}: the coupling between $r$
    (size) and $\varepsilon$ (shape) occurs only through the Weyl term;
    at $\alpha=0$ they decouple, enabling independent introduction of
    shear degrees of freedom.

  \item \textbf{Weyl penalty on shear}: for general deformations beyond
    squashing (shear modes), $C^2$ will generically be non-zero, and
    the $\alpha<0$ Weyl penalty is expected to stabilise the vacuum
    against shear perturbations as well.
\end{enumerate}

\subsection{Limitations}
\label{sec:limitations}

\subsubsection{Minisuperspace approximation}
\label{sec:limit-minisuperspace}

All results are based on the $M^3\times\Sone$ product structure and
spatial homogeneity of the minisuperspace reduction. The following
effects are not captured:
\begin{itemize}
  \item Inhomogeneous modes (gravitational waves, density perturbations),
  \item Local defects (cosmic strings, domain walls),
  \item Degrees of freedom beyond the minisuperspace variables
    $(r,\varepsilon)$.
\end{itemize}
Analysis of Weyl stability in the full field theory, starting from the
minisuperspace results, is a natural next step.

\subsubsection{Lorentzian-signature extension}
\label{sec:limit-lorentz}

The present computation uses Euclidean signature $(+,+,+,+)$. Wick
rotation to Lorentzian signature $(-,+,+,+)$ and interpretation in
real-time cosmology is an important open problem. In particular:
\begin{itemize}
  \item The eigenspace structure of the Hodge dual changes, so the
    proof of $P=0$ (\S\ref{sec:chiral}) does not directly carry over.
  \item The stability structure of the Weyl term may also be
    signature-dependent.
\end{itemize}

\subsubsection{Relation to the full field theory}
\label{sec:limit-full}

The extent to which minisuperspace results generalise to the full
theory is a non-trivial question:
\begin{itemize}
  \item $P=0$ is a consequence of the $M^3\times\Sone$ product
    structure and the minisuperspace ansatz; it does not hold for
    generic four-dimensional manifolds.
  \item Weyl stability relies on the conformal flatness of
    isotropic~$\Sthree$; more general backgrounds require additional
    analysis.
\end{itemize}

\subsubsection{Rigorous proof of global minimality of
  \texorpdfstring{$\VEC$}{V\_EC}}
\label{sec:limit-global}

This paper combines an analytic proof of local stability (Theorem~1)
with extensive numerical exploration to provide evidence that the
$\Sthree\times\Sone$ isotropic vacuum is the \emph{de facto} global
minimum. A complete analytic proof of global minimality of $\VEC$
remains open due to the highly non-linear dependence of the effective
potential on $(r,\varepsilon)$.

\subsubsection{Vacuum structure and the ghost problem: geometric
  observations}
\label{sec:limit-ghost}

The stability established in \S\ref{sec:stability} for the homogeneous
vacuum does not automatically resolve the ghost problem in the full
theory including inhomogeneous fluctuations. Nevertheless, the present
model simultaneously possesses three geometric structures---conformal
flatness of the isotropic vacuum, chiral torsion background, and
positive background curvature---each of which may have non-trivial
implications for ghost avoidance.

Whether these structures provide an effective ghost-avoidance mechanism
requires second-order perturbation theory around the isotropic
$\Sthree$ vacuum in the $\alpha\le 0$ EC+NY+Weyl theory, which lies
beyond the scope of the present minisuperspace analysis.
   % Discussion
%==============================================================================
% Section 10: Conclusion
%==============================================================================
\section{Conclusion}
\label{sec:conclusion}

We have subjected the $\Sthree\times\Sone$ isotropic vacuum of the
Einstein--Cartan + Nieh--Yan minisuperspace (paper~I) to two
independent extensions---topological and dynamical---and demonstrated
its structural robustness both analytically and numerically.

\subsection{Summary of Main Results}

The principal results are encapsulated in four statements:

\paragraph{(1) Proposition~1 (Chiral equilibrium, $P=0$).}
Under the $M^3\times\Sone$ minisuperspace ansatz with EC connection,
the Pontryagin density $P=\innerprod{R}{*R}=0$ vanishes identically.
This follows from the orthogonal decomposition
$\Lambda^2(M^4)=\Lambda^2(M^3)\oplus\Lambda^1(M^3)\wedge\dd\tau$ and
the block-exchange property of the Hodge dual: when the curvature lies
entirely in the spatial block, its dual lies in the mixed block, and
orthogonality gives $P=0$.

This identity means that self-dual instantons are forbidden within this
framework, eliminating the threat of instanton-mediated vacuum decay.

\paragraph{(2) Theorem~1 (Weyl stability of the isotropic vacuum).}
For $\alpha\le 0$, the conformal flatness $C^2=0$ of isotropic
$\Sthree\times\Sone$ shields the vacuum from the Weyl term, and the
global minimum of paper~I is analytically protected. Numerically, at
all 201 scan points ($\alpha\in[-1,1]$) the minimum value
$V_{\min}=-421.103$ (11-digit precision), $r^*=2.000$, and
$\varepsilon^*=0$ are maintained independently of~$\alpha$.

\paragraph{(3) Theorem~2 (Unbounded instability for $\alpha>0$).}
For $\alpha>0$, the asymptotic dominance of the Weyl term
($\VWeyl\sim -\alpha/r$ versus $\VEC\sim r$ as $r\to 0$) renders the
effective potential unbounded below ($\inf\Veff=-\infty$). The boundary
$\alpha=0$ is sharp, owing to the unboundedness of
$C^2\cdot\Vol\sim 1/r$.

\paragraph{(4) Theorem~3 (Parameter independence of the stability
  boundary).}
The stability boundary $\alpha=0$ is independent of the paper~I
parameters $(V,\eta,\theta_{\NY})$, a consequence of the geometric
decoupling of $C^2$ from torsion. This is confirmed analytically and
verified numerically at four representative points spanning Type~I
centre to II/III boundary.

\subsection{Sign Constraint on the Weyl Coupling Constant}

Theorems~2 and~3 jointly imply the sign constraint $\alpha\le 0$ for
the EC+NY+Weyl theory. $\alpha>0$ triggers ghost instability (the
minisuperspace manifestation of Ostrogradsky's theorem), while
$\alpha<0$ provides isotropy stabilisation and a repulsive core
(regularisation) at $r\to 0$. This constraint is independent of
paper~I parameter tuning.

\subsection{Preservation of the Topology-Selection Principle}

A systematic comparison of three topologies ($\Sthree$, $\Tthree$,
$\Nilthree$) confirms that the topology-selection principle of
paper~I---$\Sthree$ forming the lowest-energy vacuum---is preserved for
$\alpha\le 0$. $\Sthree$ attains $C^2=0$ exactly at $\varepsilon^*=0$,
forming a stable isotropic vacuum. $\Nilthree$ lacks a stable
anisotropic vacuum and asymptotes to a flat limit
($\varepsilon\to\infty$) with $V_{\min}\approx 5>0$, which is
energetically unfavourable relative to $\Sthree$.
  % Conclusion

%------------------------------------------------------------------------------
% Acknowledgments
%------------------------------------------------------------------------------
\section*{Acknowledgments}

The author acknowledges the use of the following AI tools during manuscript
preparation: Claude Opus 4.6 (Anthropic; accessed 2026-02),
Gemini 3.1 Pro (Google; accessed 2026-02),
Grok 4.2 Beta (xAI; accessed 2026-02), and
GPT-5.2 Thinking via ChatGPT (OpenAI; accessed 2025-12).
These tools were used for language editing (including Japanese--English
translation), outlining and improving exposition, drafting and reviewing
auxiliary code (scripts and pseudocode) that was subsequently tested and
validated by the author, and proposing candidate consistency checks and
alternative derivations to be independently verified by the author. They were
additionally used for literature discovery support (keyword generation and
preliminary summaries of candidate papers); all references and factual claims
were verified by the author using primary sources.

The AI tools did not determine the scientific claims of this work and were not
used to generate or modify research data or evidentiary figures. The author
takes full responsibility for the content and for any remaining errors.

This work was developed within the informal collaborative project
\textbf{DPPU} (\textbf{D}onut-like topology, \textbf{P}lanck-scale
compactness, \textbf{P}recession dynamics, \textbf{U}niverse).

%------------------------------------------------------------------------------
% Appendices
%------------------------------------------------------------------------------
\appendix
%==============================================================================
% Appendix A: DPPUv2 Engine v4 Specification
%==============================================================================
\section{DPPUv2 Engine v4 Specification}
\label{app:engine}

This appendix documents the specification of the DPPUv2 computation
engine~v4 and its reliability verification.

\subsection{Engine Overview}
\label{sec:app-overview}

The DPPUv2 Engine~v4 is a symbolic--numerical computation engine that
performs minisuperspace reduction of the EC+NY+Weyl theory. Building on
the EC+NY implementation of paper~I, the following extensions have been
made for the present work:

\begin{enumerate}
  \item \textbf{Squashed ansatz}: introduction of the anisotropy
    parameter~$\varepsilon$ and implementation of the two-variable
    $(r,\varepsilon)$ effective-potential computation.

  \item \textbf{Weyl tensor from Levi-Civita connection}: symbolic
    computation of all Weyl-tensor components and the scalar~$C^2$
    from the Levi-Civita connection.

  \item \textbf{Modular architecture with automated consistency
    checks}: separation into Levi-Civita, EC-connection, Weyl-tensor,
    and effective-potential layers, with independent verification at
    each level (\S\ref{sec:app-checks}).
\end{enumerate}

\subsection{Computation Flow}
\label{sec:app-flow}

\paragraph{Theory construction (symbolic computation).}

\begin{enumerate}
  \item \textbf{Geometric setup}: topology-dependent structure constants
    $C^i{}_{jk}$.
  \item \textbf{Levi-Civita connection}: Koszul formula
    $\Gamma^a{}_{bc}=\tfrac{1}{2}(C^a{}_{bc}+C^c{}_{ba}-C^b{}_{ac})$.
  \item \textbf{Contortion}: from the torsion ansatz,
    $K_{abc}=\tfrac{1}{2}(T_{abc}+T_{bca}-T_{cab})$.
  \item \textbf{EC connection}:
    $\Gamma^a_{\EC,bc}=\Gamma^a_{\LC,bc}+K^a{}_{bc}$.
  \item \textbf{Scalar quantities}: $R_{\EC}$, $T_{abc}\,T^{abc}$,
    $N_{\TT}$, $N_{\REE}$, $N_{\FULL}$.
  \item \textbf{Weyl tensor}: from the Levi-Civita connection,
    $R^a{}_{bcd}\to R_{bd}\to R\to C_{abcd}\to C^2$.
  \item \textbf{Effective potential}: assembly via
    $\Veff = -\mathcal{L}\times\Vol$.
\end{enumerate}

\paragraph{Numerical exploration.}

\begin{enumerate}
  \item \textbf{Parameter scan}: $\alpha\in[-1,1]$ (201 points),
    $(r,\varepsilon)$ 2D grid.
  \item \textbf{Optimisation}: \texttt{scipy.optimize.brute} (Ns\,=\,20)
    grid search + multi-start L-BFGS-B.
  \item \textbf{Classification}: stability determination based on
    optimisation results (convergence flag, boundary-attachment
    detection).
\end{enumerate}

\subsection{Automated Consistency Checks}
\label{sec:app-checks}

The engine automatically performs the following checks at each
computation step:

\paragraph{Level~1: Metric compatibility.}
\begin{equation}
  \nabla_c\,g_{ab} = 0\,.
\end{equation}
Verification that the EC connection is compatible with the frame metric
$\eta_{ab}=\delta_{ab}$.

\paragraph{Level~2: Riemann-tensor symmetries.}
\begin{equation}
  R_{abcd} = -R_{abdc} = -R_{bacd} = R_{cdab}\,.
\end{equation}
All components are checked against the three symmetry conditions.

\paragraph{Level~3: Comparison with known analytic values.}
\begin{itemize}
  \item $\Sthree$: $R_{\LC}(r,0)=24/r^2$.
  \item $\Tthree$: $R_{\LC}=0$, $C^2=0$.
  \item $\Sthree$: $C^2(r,0)=0$ (conformal flatness at the isotropic
    point).
\end{itemize}

\subsection{\texorpdfstring{$\Tthree$}{T³} Null Test: Engine Sanity
  Check}
\label{sec:app-null}

Since $\Tthree$ has $C^2=0$, the effective potential should exhibit no
$\alpha$-dependence. Numerical confirmation at all 201 points
($\alpha\in[-1,1]$) with machine-precision agreement
(\S\ref{sec:null-test}) validates:
\begin{enumerate}
  \item Correct implementation of the $\alpha\,C^2$ term.
  \item Zero residual Weyl contribution for $C^2=0$ topologies.
  \item Maintenance of machine-level numerical precision.
\end{enumerate}

%------------------------------------------------------------------------------
\subsection{License}
\label{app:license}

The code and data are released under the MIT License:

\begin{quote}
\small
MIT License

Copyright (c) 2026 Muacca

Permission is hereby granted, free of charge, to any person obtaining a copy
of this software and associated documentation files (the ``Software''), to deal
in the Software without restriction, including without limitation the rights
to use, copy, modify, merge, publish, distribute, sublicense, and/or sell
copies of the Software, and to permit persons to whom the Software is
furnished to do so, subject to the following conditions:

The above copyright notice and this permission notice shall be included in all
copies or substantial portions of the Software.

THE SOFTWARE IS PROVIDED ``AS IS'', WITHOUT WARRANTY OF ANY KIND, EXPRESS OR
IMPLIED, INCLUDING BUT NOT LIMITED TO THE WARRANTIES OF MERCHANTABILITY,
FITNESS FOR A PARTICULAR PURPOSE AND NONINFRINGEMENT. IN NO EVENT SHALL THE
AUTHORS OR COPYRIGHT HOLDERS BE LIABLE FOR ANY CLAIM, DAMAGES OR OTHER
LIABILITY, WHETHER IN AN ACTION OF CONTRACT, TORT OR OTHERWISE, ARISING FROM,
OUT OF OR IN CONNECTION WITH THE SOFTWARE OR THE USE OR OTHER DEALINGS IN THE
SOFTWARE.
\end{quote}

%------------------------------------------------------------------------------
\subsection{Citation}
\label{app:citation}

If you use this code or data in your research, please cite:

\begin{quote}
\small
Muacca, ``Structural Robustness of Isotropic $\Sthree$ Vacua\\
in Einstein--Cartan Minisuperspace\\
via Chiral Equilibrium and Weyl Stability,'' 
\end{quote}

The Zenodo DOI for the code repository is:
\begin{center}
\texttt{10.5281/zenodo.18815498}
\end{center}

%------------------------------------------------------------------------------
\subsection{Contact}
\label{app:contact}

For questions, bug reports, or collaboration inquiries:

\begin{itemize}
  \item Email: \texttt{muacca@dmwp.jp}
  \item GitHub Issues: \texttt{https://github.com/Muacca/DPPUv2-paper02/issues}
\end{itemize}
  % DPPUv2 Engine v4 Specification
%==============================================================================
% Appendix B: Numerical Verification Log & Nil³ Analysis
%==============================================================================
\section{Numerical Verification Log and \texorpdfstring{$\Nilthree$}{Nil³}
  Analysis}
\label{app:numerical}

This appendix provides the search parameters, numerical verification
details, and additional analysis of the $\Nilthree$ flat-limit
asymptotics and the universality of the $\alpha=0$ stability boundary.

\subsection{Search Parameter Summary}
\label{sec:app-search}

\subsubsection{Search ranges}

\begin{table}[ht]
\centering
\caption{Search parameter ranges.}
\label{tab:search-ranges}
\begin{tabular}{@{}lll@{}}
\toprule
Parameter & Range & Resolution \\
\midrule
$r$ & $[0.01,\,10]$ & brute: 30 points; L-BFGS-B: continuous \\
$\varepsilon$ & $[-0.95,\,5.0]$ & brute: 30 points; L-BFGS-B: continuous \\
$\alpha$ & $[-1,\,1]$ & 201 points (step 0.01) \\
\bottomrule
\end{tabular}
\end{table}

The upper bound $\varepsilon=5$ was chosen because: (i)~$\varepsilon^*$
remains at the boundary with $V_{\min}>0$ even at $\varepsilon=5$; and
(ii)~for larger~$\varepsilon$, the squashing factor
$(1{+}\varepsilon)^{-2/3}\ll 1$ challenges the interpretation of the
minisuperspace ansatz.

\subsubsection{Optimisation method}

\begin{enumerate}
  \item \textbf{Coarse search}: \texttt{scipy.optimize.brute} (Ns\,=\,30)
    over the 2D $(r,\varepsilon)$ grid.
  \item \textbf{Refinement}: multi-start L-BFGS-B from the top $n=8$
    grid candidates.
  \item \textbf{Convergence}: determined by the L-BFGS-B convergence
    flag; boundary attachment is reported as
    \texttt{converged\,=\,False}.
\end{enumerate}

\subsubsection{Paper~I reference parameters}

\begin{table}[ht]
\centering
\caption{Paper~I reference parameters.}
\label{tab:ref-params}
\begin{tabular}{@{}lr@{}}
\toprule
Parameter & Value \\
\midrule
$V$ & 4.0 \\
$\eta$ & $-2.0$, $-3.0$, $0.0$, $2.0$ \\
$\theta_{\NY}$ & 1.0 \\
$\kappa$ & 1.0 \\
$L$ & 1.0 \\
\bottomrule
\end{tabular}
\end{table}

\subsection{Output File Summary}
\label{sec:app-files}

Filenames follow the convention \texttt{\{script\}\_\{YYYYMMDD\_HHMMSS\}.csv};
\texttt{*} below denotes the timestamp.
All scripts reside under \texttt{scripts/}.

\begin{table}[ht]
\centering
\caption{Principal output files.}
\label{tab:output-files}
\begin{tabularx}{\textwidth}{@{}ll@{}}
\toprule
File pattern & Content \\
/ Script & / Section \\
\midrule
\texttt{potential\_landscape\_mapping\_*.csv}
  & $\Sthree$ $\Veff$ on $(r,\varepsilon)$ grid \\
\texttt{/ potential\_landscape\_mapping.py}
  & / \S\ref{sec:contour} (Fig.~\ref{fig:landscape}) \\
\quad\ \\[4pt]

\texttt{critical\_analysis\_*.csv}
  & $\Sthree$ $\alpha$-scan ($r^*,\varepsilon^*,V_{\min}$) \\
\texttt{/ critical\_analysis.py}
  & / \S\ref{sec:bifurcation} (Figs.~\ref{fig:bifurcation}, \ref{fig:size-stability}) \\
\quad\texttt{{-}{-}topology S3} \\[4pt]

\texttt{critical\_analysis\_*.csv}
  & $\Sthree$ fine scan ($\alpha\in[-0.1,0.1]$) \\
\texttt{/ critical\_analysis.py}
  & / \S\ref{sec:bifurcation} (inset) \\
\quad\texttt{{-}{-}topology S3 {-}{-}fine} \\[4pt]

\texttt{critical\_analysis\_*.csv} 
  & $\Tthree$ $\alpha$-scan \\
\texttt{/ critical\_analysis.py}
  & / \S\ref{sec:null-test} (Fig.~\ref{fig:null-test}) \\
\quad\texttt{{-}{-}topology T3} \\[4pt]

\texttt{topology\_comparison\_*.csv}
  & 3-topology $V_{\min}(\alpha)$ comparison \\
\texttt{/ topology\_comparison.py}
  & / \S\ref{sec:Vmin-comparison} (Fig.~\ref{fig:topology-comparison}) \\
\quad \\[4pt]

\texttt{run\_alpha\_boundary\_scan\_*.csv}
  & Parameter-dependence scans (4 sets) \\
\texttt{/ run\_alpha\_boundary\_scan.py}
  & / \S\ref{sec:facet-plot} (Fig.~\ref{fig:facet-plot}) \\
\quad ($\times4$ param.\ sets) \\
\bottomrule
\end{tabularx}
\end{table}

\subsection{Numerical Reliability}
\label{sec:app-reliability}

\subsubsection{Grid-resolution verification}

\begin{table}[ht]
\centering
\caption{Comparison of grid and optimisation results.}
\label{tab:grid-vs-opt}
\begin{tabular}{@{}llll@{}}
\toprule
Quantity & Grid minimum & Optimisation & Source of discrepancy \\
\midrule
$V_{\min}$ ($\Sthree$, $\alpha=0$) & $-415.5$ & $-421.103$
  & Grid resolution \\
$r^*$ & $\approx 2.13$ & $2.000$ & Grid resolution \\
\bottomrule
\end{tabular}
\end{table}

The coarse grid provides an overview of the potential landscape, while
L-BFGS-B identifies the precise minimum. The discrepancy is consistent
with grid-resolution limitations.

\subsubsection{Convergence flags for $\alpha>0$}

In the $\alpha>0$ region, \texttt{converged\,=\,False} is consistently
reported, with the optimum converging to the search boundary
$(r^*,\varepsilon^*)=(0.01,-0.95)$. This is \emph{not} an optimisation
failure but the correct report that the potential is unbounded below
within the search range. The true minimum lies along
$(r,\varepsilon)\to(0,-1)$, inaccessible due to search bounds.

\subsection{\texorpdfstring{$\Nilthree$}{Nil³} Behaviour: Flat-Limit
  Asymptotics}
\label{app:nil3}

\subsubsection{Numerical search and results}

An extended search over $\varepsilon\in[-0.95,5.0]$ (grid Ns\,=\,30,
multi-start $n=8$) yields:

\begin{table}[ht]
\centering
\caption{$\Nilthree$ search results ($\varepsilon_{\max}=5.0$).}
\label{tab:nil3}
\begin{tabular}{@{}ll@{}}
\toprule
Quantity & Result \\
\midrule
$r^*$ & $\approx 1.500$ \\
$\varepsilon^*$ & $5.0$ (upper-bound attachment) \\
$V_{\min}$ ($\alpha=-1$) & $5.0$ \\
$V_{\min}$ ($\alpha=-0.01$) & $4.9$ \\
\bottomrule
\end{tabular}
\end{table}

The attachment of $\varepsilon^*$ to the upper search bound indicates
that $\Veff$ is monotonically decreasing in $\varepsilon$ with no
interior minimum at finite~$\varepsilon$. The asymptotic value
$V_{\min}\approx 4.9$ is weakly $\alpha$-dependent.

\subsubsection{Physical interpretation: flat-limit asymptotics}

The $\Nilthree$ structure constants contain the factor
$(1{+}\varepsilon)^{-4/3}$:
\begin{equation}
  C^2{}_{01} = -\frac{(1{+}\varepsilon)^{-4/3}}{r}\,,\qquad
  C^2{}_{10} = +\frac{(1{+}\varepsilon)^{-4/3}}{r}\,.
\end{equation}
As $\varepsilon\to\infty$, $(1{+}\varepsilon)^{-4/3}\to 0$ and the
structure constants vanish, so $\Nilthree$ approaches the flat limit
($\Tthree$-like behaviour). At $\varepsilon=5$,
$(1{+}5)^{-4/3}\approx 0.11$, already suppressing the structure
constants to below 11\%.

Physically, the $\Nilthree$ effective potential minimises the curvature
cost ($C^2>0$) from the Heisenberg group structure by driving the
structure constants to zero ($\varepsilon\to\infty$): $\Nilthree$
``collapses'' towards a flat configuration.

\subsubsection{Impact on conclusions}

The $\Nilthree$ flat-limit asymptotics do not affect the main
conclusions:
\begin{enumerate}
  \item $\Nilthree$ has $V_{\min}\approx 5>0$, energetically
    unfavourable relative to $\Sthree$ ($V_{\min}=-421$). The ordering
    $V_{\min}(\Sthree)<0<V_{\min}(\Tthree)\approx 0
    <V_{\min}(\Nilthree)$ is preserved.
  \item The $\Sthree$ results ($\varepsilon^*=0$) are analytically
    established (Theorem~\ref{thm:stability}) and do not depend on the
    search range.
  \item The flat-limit approach is consistent with the paper's central
    theme: conformally flat configurations ($C^2=0$) are energetically
    favoured.
\end{enumerate}

\subsection{\texorpdfstring{$\alpha=0$}{α=0} Stability Boundary for
  \texorpdfstring{$\Nilthree$}{Nil³}: Additional Confirmation}
\label{app:nil3-alpha}

A high-resolution scan at $\alpha\in[-0.05,0.05]$ (step~0.001) with
$\varepsilon\in[-0.95,5.0]$ confirms:

\begin{table}[ht]
\centering
\caption{$\Nilthree$ stability-boundary confirmation.}
\label{tab:nil3-boundary}
\begin{tabular}{@{}ll@{}}
\toprule
Quantity & Value \\
\midrule
Transition location & $\alpha=0.000\to +0.001$ \\
Status at $\alpha=0$ & \texttt{converged\,=\,True}
  ($V_{\min}\approx 4.9$) \\
\bottomrule
\end{tabular}
\end{table}

The stability transition occurs at $\alpha=0\to +0.001$, with
$\alpha=0$ itself classified as stable. This provides additional
numerical evidence that the $\alpha=0$ stability boundary holds
universally across all three topologies ($\Sthree$, $\Tthree$,
$\Nilthree$).
  % Numerical Verification Log & Nil³ Analysis
%==============================================================================
% Appendix C: Symbolic Computation Details
%==============================================================================
\section{Symbolic Computation Details}
\label{app:symbolic}

This appendix documents the SymPy-based derivation of the Weyl scalar
$C^2(r,\varepsilon)$ and the isotropic effective potential
$\VEC(r,0)$.

\subsection{Derivation of \texorpdfstring{$C^2(r,\varepsilon)$}{C²(r,ε)}}
\label{sec:app-C2}

\subsubsection{Input: squashed structure constants}

The squashed $\Sthree$ structure constants (\S\ref{sec:squashed}):
\begin{equation}
  C^i{}_{jk}(\varepsilon)
  = \frac{4}{r}\,\varepsilon_{ijk}\times f_i(\varepsilon)\,,
\end{equation}
with
\begin{equation}
  f_0 = f_1 = (1{+}\varepsilon)^{1/3}\times(1{+}\varepsilon)^{1/3}
            = (1{+}\varepsilon)^{2/3}\,,\qquad
  f_2 = (1{+}\varepsilon)^{-2/3}\times(1{+}\varepsilon)^{-2/3}
      = (1{+}\varepsilon)^{-4/3}\,.
\end{equation}
Here each factor $f_i$ originates from the squashing of the
orthonormal coframe $e^a = r\,\lambda_a(\varepsilon)\,\sigma^a$ (no
sum), where $\lambda_0=\lambda_1=(1{+}\varepsilon)^{1/3}$ and
$\lambda_2=(1{+}\varepsilon)^{-2/3}$. The product form of $f_i$
reflects the two coframe factors that appear in the left-hand sides of
the Cartan structure equations for the corresponding index.

\subsubsection{Step~1: Levi-Civita connection}

Koszul formula:
\begin{equation}
  \Gamma^a{}_{bc}
  = \tfrac{1}{2}\bigl(C^a{}_{bc}+C^c{}_{ba}-C^b{}_{ac}\bigr)\,.
\end{equation}
Non-vanishing components are enumerated symbolically. At $\varepsilon=0$
these reduce to the paper~I results (Appendix~A.2.1 of~\cite{paper1}).

\subsubsection{Step~2: Riemann tensor}

Frame-basis curvature tensor:
\begin{equation}
  R^a{}_{bcd}
  = \Gamma^a{}_{ec}\,\Gamma^e{}_{bd}
  - \Gamma^a{}_{ed}\,\Gamma^e{}_{bc}
  + \Gamma^a{}_{be}\,C^e{}_{cd}\,.
\end{equation}
The third term is specific to frame bases (analogous to the
$\partial\Gamma$ term in coordinate bases). All $4^4=256$ components
are computed and the antisymmetries $R_{abcd}=-R_{abdc}$,
$R_{abcd}=-R_{bacd}$ are verified automatically.

\subsubsection{Step~3: Ricci tensor and scalar}

\begin{equation}
  R_{bd} = R^a{}_{bad}\,,\qquad R = R^a{}_a = R_{bb}\,.
\end{equation}
The Levi-Civita Ricci scalar:
\begin{equation}
  R_{\LC}(r,\varepsilon)
  = \frac{8\bigl(4(1{+}\varepsilon)^2-1\bigr)}
         {r^2\,(1{+}\varepsilon)^{8/3}}\,.
\end{equation}
At the isotropic point: $R_{\LC}(r,0)=24/r^2$, matching the standard
value for~$\Sthree$.

\subsubsection{Step~4: Weyl tensor}

The four-dimensional Weyl tensor:
\begin{equation}
  C_{abcd} = R_{abcd}
  - \tfrac{1}{2}\bigl(g_{ac}R_{bd}-g_{ad}R_{bc}
  -g_{bc}R_{ad}+g_{bd}R_{ac}\bigr)
  + \tfrac{R}{6}\bigl(g_{ac}g_{bd}-g_{ad}g_{bc}\bigr)\,.
\end{equation}
Computed in the orthonormal frame ($g_{ab}=\delta_{ab}$). All
$\binom{4}{2}^2=36$ independent components are evaluated symbolically.

\subsubsection{Step~5: Weyl scalar}

\begin{equation}
  C^2 = C_{abcd}\,C^{abcd}
  = \sum_{a,b,c,d}C_{abcd}^2\,.
\end{equation}
After applying SymPy's \texttt{simplify} and \texttt{factor}:
\begin{equation}
  \boxed{C^2(r,\varepsilon)
  = \frac{1024\,\varepsilon^2(\varepsilon{+}2)^2}
         {3\,r^4\,(1{+}\varepsilon)^{16/3}}\,.}
\end{equation}

\subsection{Complete List of Non-Vanishing Weyl Components}
\label{sec:app-weyl-list}

All 24 non-vanishing components share the common factor
$\varepsilon(\varepsilon{+}2)/[r^2(1{+}\varepsilon)^{8/3}]$.

\paragraph{Diagonal type ($C_{abab}$).}

\begin{table}[ht]
\centering
\caption{Independent diagonal Weyl-tensor components.}
\label{tab:weyl-diag}
\begin{tabular}{@{}ll@{}}
\toprule
Component & Value \\
\midrule
$C_{0101}$
  & $+\dfrac{16\,\varepsilon(\varepsilon{+}2)}
            {3r^2(1{+}\varepsilon)^{8/3}}$ \\[8pt]
$C_{0202}$
  & $-\dfrac{8\,\varepsilon(\varepsilon{+}2)}
            {3r^2(1{+}\varepsilon)^{8/3}}$ \\[8pt]
$C_{0303}$
  & $-\dfrac{8\,\varepsilon(\varepsilon{+}2)}
            {3r^2(1{+}\varepsilon)^{8/3}}$ \\[8pt]
$C_{1212}$
  & $-\dfrac{8\,\varepsilon(\varepsilon{+}2)}
            {3r^2(1{+}\varepsilon)^{8/3}}$ \\[8pt]
$C_{1313}$
  & $-\dfrac{8\,\varepsilon(\varepsilon{+}2)}
            {3r^2(1{+}\varepsilon)^{8/3}}$ \\[8pt]
$C_{2323}$
  & $+\dfrac{16\,\varepsilon(\varepsilon{+}2)}
            {3r^2(1{+}\varepsilon)^{8/3}}$ \\
\bottomrule
\end{tabular}
\end{table}

\paragraph{Cross-type components.}
The Weyl-tensor symmetries $C_{abcd}=C_{cdab}$,
$C_{abcd}=-C_{abdc}=-C_{bacd}$ generate the remaining 18 non-vanishing
components from the 6 independent diagonal components listed above.

\paragraph{Tracelessness verification.}
\begin{equation}
  C^a{}_{bac} = 0\qquad (\forall\,b,c)\,,
\end{equation}
confirmed by SymPy for all index combinations.

\subsection{Complete Symbolic Expression for
  \texorpdfstring{$\VEC(r,0)$}{V\_EC(r,0)}}
\label{sec:app-VEC}

The EC engine (\texttt{S3S1Engine}, MX mode, FULL variant) yields the
isotropic effective potential:
\begin{equation}
  \VEC(r,0)
  = \frac{2\pi^2 Lr}{3\kappa^2}
    \bigl(V^2 r^2
    + 6V\eta\kappa^2 r\,\theta_{\NY}
    + 9\eta^2 - 36\bigr)\,.
  \label{eq:VEC-symbolic}
\end{equation}
Substituting the reference parameters
$(V{=}4,\,\eta{=}{-}2,\,\theta_{\NY}{=}1,\,\kappa{=}1,\,L{=}1)$:
\begin{equation}
  \VEC(r,0) = \frac{32\pi^2}{3}\,r^2(r-3)\,.
\end{equation}
This cubic (in~$r$) has a minimum at:
\begin{equation}
  \frac{\dd\VEC}{\dd r}
  = \frac{32\pi^2}{3}\cdot 3r(r-2) = 0
  \quad\Rightarrow\quad r^*=2\,,
\end{equation}
with value
\begin{equation}
  \VEC(2,0) = \frac{32\pi^2}{3}\cdot 4\cdot(2-3)
  = -\frac{128\pi^2}{3} \approx -421.1\,.
\end{equation}

\subsection{Closed-Form Expression for
  \texorpdfstring{$C^2\cdot\Vol$}{C²·Vol}}
\label{sec:app-C2Vol}

The product of the Weyl scalar and the four-volume:
\begin{equation}
  C^2(r,\varepsilon)\cdot\Vol(r)
  = \frac{1024\,\varepsilon^2(\varepsilon{+}2)^2}
         {3\,r^4\,(1{+}\varepsilon)^{16/3}}
    \times 2\pi^2 Lr^3
  = \frac{2048\pi^2 L\,\varepsilon^2(\varepsilon{+}2)^2}
         {3\,r\,(1{+}\varepsilon)^{16/3}}\,.
\end{equation}
Asymptotic scalings:
\begin{itemize}
  \item $r\to 0$: $C^2\cdot\Vol\sim 1/r$ (diverges).
  \item $\varepsilon\to -1$ ($\delta=1{+}\varepsilon\to 0^+$):
    $C^2\cdot\Vol\sim 1/\delta^{16/3}$ (diverges).
\end{itemize}
This two-directional unboundedness is the mathematical basis of
Theorem~\ref{thm:instability}(a).

\subsection{Verification Script}
\label{sec:app-script}

All computations above are reproduced by the script
\texttt{scripts/proofs/analytical\_proof.py}, which automatically executes:
\begin{enumerate}
  \item Setup of squashed structure constants.
  \item Levi-Civita connection computation.
  \item Riemann tensor computation with antisymmetry verification.
  \item Ricci tensor and scalar computation.
  \item Weyl tensor computation with tracelessness verification.
  \item $C^2$ simplification and factoring.
  \item Confirmation of $C^2=0$ at $\varepsilon=0$.
  \item First and second $\varepsilon$-derivatives at the isotropic
    point.
  \item $R_{\LC}$ computation and comparison with known values.
\end{enumerate}
  % Symbolic Computation Details
%==============================================================================
% Appendix D: Phase Atlas
%==============================================================================
\section{Phase Atlas: Effective Potential and Phase Diagram
  in the \texorpdfstring{$(\alpha_W,\varepsilon)$}{(αW,ε)} Parameter Space}
\label{app:phase-atlas}

This appendix systematically illustrates how the effective potential
$\Veff(r)$ and the $\Sthree\times\Sone$ phase diagram vary across five
representative cases in the $(\alpha_W,\varepsilon)$ parameter space
(Figs.~\ref{fig:potential-a0e0}--\ref{fig:potential-a+e-}).
These figures provide the visual support for Theorems~1, 2, and~3 in
the main text.

Each figure shows: (left panel) the phase diagram in the $(V,\eta)$
plane at fixed $\theta_{\NY}=1.00$, MX/FULL mode; (right panel) the
effective potential $\Veff(r)$ at three representative points.
The representative points are selected at fixed $V=4.0$ with varying
$\eta$, one per phase region ($\circ$: Stable Well; $\square$:
Intermediate; $\triangle$: No Well).

%------------------------------------------------------------------------------
\subsection{Baseline: \texorpdfstring{$\alpha_W=0$, $\varepsilon=0$}{αW=0, ε=0}}
\label{sec:appD-baseline}

\begin{figure}[ht]
  \centering
  \includegraphics[width=\textwidth]{figures/fig08_potential_a0e0}
  \caption{Effective potential and phase diagram for
    $\alpha_W=0$, $\varepsilon=0$ (isotropic $\Sthree$, Weyl term
    inactive). This reference case reproduces the three-phase structure
    of paper~I~\cite{paper1} and confirms the backward compatibility of
    engine~v4.}
  \label{fig:potential-a0e0}
\end{figure}

The case $\alpha_W=0$ (Weyl term inactive) and $\varepsilon=0$
(isotropic $\Sthree$) reproduces the three-phase structure of paper~I
exactly.  It serves as the reference case for verifying the backward
compatibility of engine~v4.

\paragraph{Phase diagram.}
\begin{itemize}
  \item \textbf{Type~I (Stable Well, green):} $\eta\lesssim -2.5$.
    The effective potential has a globally stable minimum.
  \item \textbf{Type~II (Rolling, yellow):} $-2.5\lesssim\eta\lesssim 2.3$.
    No barrier near $r\to 0$; the potential ``rolls'' without a true well.
  \item \textbf{Type~III (No Well, grey):} $\eta\gtrsim 2.3$.
    No well structure in the effective potential.
\end{itemize}

\paragraph{Effective potential.}
\begin{itemize}
  \item \textbf{Point~1} ($\eta=-4$, Type~I): well-type; deep global
    minimum near $r\approx 3$, high barrier as $r\to 0$, divergence
    as $r\to\infty$.  Typical isotropic $\Sthree$ vacuum profile.
  \item \textbf{Point~2} ($\eta=0$, Type~II): local minimum survives
    near $r\approx 1$, but the $r\to 0$ barrier disappears.
  \item \textbf{Point~3} ($\eta=3$, Type~III): monotonically
    increasing; no well formed at any~$r$.
\end{itemize}

%------------------------------------------------------------------------------
\subsection{Weyl Stable Region: \texorpdfstring{$\alpha_W<0$}{αW<0}}
\label{sec:appD-weyl-stable}

For $\alpha_W<0$, Theorem~1 guarantees that the Weyl term acts as a
penalty in the anisotropy direction, protecting the isotropic vacuum
$(\varepsilon=0)$.  In this regime, Types~I and~II are further
subdivided into \textbf{Type~I-W} (Double Well) and \textbf{Type~II-W}
(Single Well), which are specific to $\alpha_W<0$.

The two cases below show how the sign of~$\varepsilon$ shifts the
phase boundary.

%- - - - - - - - - - - - - - - - - - - - - - - - - - - - - - - - - - - - - -
\subsubsection{\texorpdfstring{$\alpha_W=-0.100$, $\varepsilon=+0.150$}{αW=-0.100, ε=+0.150}}
\label{sec:appD-a-e+}

\begin{figure}[ht]
  \centering
  \includegraphics[width=\textwidth]{figures/fig09_potential_a-e+}
  \caption{Effective potential and phase diagram for
    $\alpha_W<0$, $\varepsilon>0$.  The combination produces the widest
    Type~I-W (Double Well) region and strongly supports the stable
    vacuum.}
  \label{fig:potential-a-e+}
\end{figure}

\paragraph{Phase diagram.}
\begin{itemize}
  \item \textbf{Type~I-W (Double Well, blue):} wide stable region at
    $\eta\lesssim -2.5$.  The effective potential has a local maximum
    near the repulsive core and a deep global minimum further out.
  \item \textbf{Type~II-W (Single Well, orange):} upper region
    ($\eta\gtrsim -2.5$); single-well potential profile.
\end{itemize}

\paragraph{Effective potential.}
\begin{itemize}
  \item \textbf{Point~1} ($\eta=-4$, Type~I-W): double-well type;
    local maximum near $r\to 0$ (potential barrier), local minimum
    between the barrier and the repulsive core, and a deep global
    minimum near $r\approx 3$.  The Weyl correction acts as a barrier
    against collapse toward $r=0$.
  \item \textbf{Point~2} ($\eta=0$, Type~II-W): single-well type;
    the $r\to 0$ barrier disappears, leaving one clear minimum near
    $r\approx 1$.
  \item \textbf{Point~3} ($\eta=3$, Type~II-W): also single-well;
    formerly Type~III at $\varepsilon=0$, now acquiring a minimum at
    the repulsive-core edge.
\end{itemize}

The combination $\alpha_W<0$ and $\varepsilon>0$ produces the widest
Type~I-W region and most strongly supports the stable vacuum.

%- - - - - - - - - - - - - - - - - - - - - - - - - - - - - - - - - - - - - -
\subsubsection{\texorpdfstring{$\alpha_W=-0.100$, $\varepsilon=-0.150$}{αW=-0.100, ε=-0.150}}
\label{sec:appD-a-e-}

\begin{figure}[ht]
  \centering
  \includegraphics[width=\textwidth]{figures/fig10_potential_a-e-}
  \caption{Effective potential and phase diagram for
    $\alpha_W<0$, $\varepsilon<0$.  Compared with Fig.~\ref{fig:potential-a-e+},
    the Type~I-W region contracts and the phase boundary shifts to
    smaller~$\eta$, reflecting the asymmetric $\varepsilon$-dependence of
    the Weyl scalar.}
  \label{fig:potential-a-e-}
\end{figure}

\paragraph{Phase diagram.}
The same two-phase structure (Type~I-W blue, Type~II-W orange) as
Fig.~\ref{fig:potential-a-e+}, but the Type~I-W (double-well) region
contracts and the phase boundary shifts toward smaller~$\eta$ compared
with the $\varepsilon>0$ case.

\paragraph{Effective potential.}
All three representative points fall in Type~II-W (single-well).
The potential barrier seen at Point~1 in Fig.~\ref{fig:potential-a-e+}
disappears, and the potential descends smoothly to a single minimum.

The asymmetric reduction of the Type~I-W region when $\varepsilon$
changes sign corresponds to the fact that the Weyl scalar
$C^2\propto\varepsilon^2(\varepsilon{+}2)^2/r^4$ is not symmetric
under $\varepsilon\leftrightarrow-\varepsilon$.

%------------------------------------------------------------------------------
\subsection{Weyl Unstable Region: \texorpdfstring{$\alpha_W>0$}{αW>0}}
\label{sec:appD-weyl-unstable}

For $\alpha_W>0$, Theorem~2 guarantees
$\Veff(r,\varepsilon;\alpha_W)\to-\infty$ as $r\to 0^+$, so no global
stable minimum exists.  We define the phase exhibiting this behaviour
as \textbf{Type~G} (Ghost / Metastable).  Type~G may possess a
metastable local minimum at finite~$r$, but the true global minimum is
absent.

%- - - - - - - - - - - - - - - - - - - - - - - - - - - - - - - - - - - - - -
\subsubsection{\texorpdfstring{$\alpha_W=+0.100$, $\varepsilon=+0.150$}{αW=+0.100, ε=+0.150}}
\label{sec:appD-a+e+}

\begin{figure}[ht]
  \centering
  \includegraphics[width=\textwidth]{figures/fig11_potential_a+e+}
  \caption{Effective potential and phase diagram for
    $\alpha_W>0$, $\varepsilon>0$.  All stable-well phases
    (Types~I, II, I-W, II-W) are absent; $\alpha_W>0$ fundamentally
    destroys the stable vacuum.}
  \label{fig:potential-a+e+}
\end{figure}

\paragraph{Phase diagram.}
\begin{itemize}
  \item \textbf{Type~G (Metastable, red):} occupies the region
    $\eta\lesssim 2$.  No globally stable minimum exists.
  \item \textbf{Type~III-G (No Well, grey):} upper region
    ($\eta\gtrsim 2$).
\end{itemize}
Stable-well phases (Types~I, II, I-W, II-W) are entirely absent,
confirming visually that $\alpha_W>0$ fundamentally destroys the stable
vacuum.

\paragraph{Effective potential.}
\begin{itemize}
  \item \textbf{Point~1} ($\eta=-4$, Type~G): $\Veff\to-\infty$ as
    $r\to 0^+$.  A remnant of a metastable local minimum is visible at
    finite~$r$, but the true global minimum descends toward $r=0$
    without bound.  This is a direct numerical confirmation of
    Theorem~2.
  \item \textbf{Point~2} ($\eta=0$, Type~G): also $\Veff\to-\infty$
    as $r\to 0^+$.
  \item \textbf{Point~3} ($\eta=3$, Type~III-G): monotonically
    increasing in~$r$, but with $\Veff\to-\infty$ as $r\to 0^+$.
\end{itemize}

%- - - - - - - - - - - - - - - - - - - - - - - - - - - - - - - - - - - - - -
\subsubsection{\texorpdfstring{$\alpha_W=+0.100$, $\varepsilon=-0.150$}{αW=+0.100, ε=-0.150}}
\label{sec:appD-a+e-}

\begin{figure}[ht]
  \centering
  \includegraphics[width=\textwidth]{figures/fig12_potential_a+e-}
  \caption{Effective potential and phase diagram for
    $\alpha_W>0$, $\varepsilon<0$.  The two-phase structure
    (Type~G + Type~III-G) is essentially identical to
    Fig.~\ref{fig:potential-a+e+}; the sign of~$\varepsilon$ does not
    alter the destabilisation caused by $\alpha_W>0$.}
  \label{fig:potential-a+e-}
\end{figure}

\paragraph{Phase diagram.}
Essentially the same two-phase structure as
Fig.~\ref{fig:potential-a+e+} (Type~G + Type~III-G).  Although the
phase boundary shifts slightly relative to Fig.~\ref{fig:potential-a+e+},
the dominance of Type~G is maintained regardless of the sign
of~$\varepsilon$.

\paragraph{Effective potential.}
\begin{itemize}
  \item \textbf{Point~1} ($\eta=-4$, Type~G): $\Veff\to-\infty$ as
    $r\to 0^+$; same collapse behaviour as Fig.~\ref{fig:potential-a+e+}.
  \item \textbf{Point~2} ($\eta=0$, Type~III-G): classified as Type~G
    in Fig.~\ref{fig:potential-a+e+}, but transitions to Type~III-G
    under $\varepsilon<0$.
  \item \textbf{Point~3} ($\eta=3$, Type~III-G): monotonically
    increasing.
\end{itemize}

The uniform destabilisation by $\alpha_W>0$ regardless of the sign
of~$\varepsilon$ is consistent with Theorem~3 (universality of the
$\alpha=0$ stability boundary).
  % Phase Atlas

%------------------------------------------------------------------------------
% References
%------------------------------------------------------------------------------
\bibliographystyle{unsrt}

\begin{thebibliography}{99}

\bibitem{paper1}
Muacca,
``Topology-Dependent Phase Classification of Effective Potentials
in Einstein--Cartan + Nieh--Yan Minisuperspace,''
Zenodo.\ \texttt{10.5281/zenodo.18213677} (2026).

\bibitem{Hehl1976}
F.~W.~Hehl, P.~von der Heyde, G.~D.~Kerlick, and J.~M.~Nester,
``General relativity with spin and torsion: Foundations and prospects,''
Rev.\ Mod.\ Phys.\ \textbf{48}, 393 (1976).

\bibitem{Chandia1997}
O.~Chandia and J.~Zanelli,
``Topological invariants, instantons, and the chiral anomaly on spaces
with torsion,''
Phys.\ Rev.\ D \textbf{55}, 7580 (1997).

\bibitem{NiehYan1982}
H.~T.~Nieh and M.~L.~Yan,
``An identity in Riemann--Cartan geometry,''
J.\ Math.\ Phys.\ \textbf{23}, 373 (1982).

\bibitem{Woodard2015}
R.~P.~Woodard,
``Ostrogradsky's theorem on Hamiltonian instability,''
Scholarpedia \textbf{10}(8), 32243 (2015).

\end{thebibliography}

%==============================================================================
\end{document}
%==============================================================================
