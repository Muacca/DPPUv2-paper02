%==============================================================================
% Appendix C: Symbolic Computation Details
%==============================================================================
\section{Symbolic Computation Details}
\label{app:symbolic}

This appendix documents the SymPy-based derivation of the Weyl scalar
$C^2(r,\varepsilon)$ and the isotropic effective potential
$\VEC(r,0)$.

\subsection{Derivation of \texorpdfstring{$C^2(r,\varepsilon)$}{C²(r,ε)}}
\label{sec:app-C2}

\subsubsection{Input: squashed structure constants}

The squashed $\Sthree$ structure constants (\S\ref{sec:squashed}):
\begin{equation}
  C^i{}_{jk}(\varepsilon)
  = \frac{4}{r}\,\varepsilon_{ijk}\times f_i(\varepsilon)\,,
\end{equation}
with
\begin{equation}
  f_0 = f_1 = (1{+}\varepsilon)^{1/3}\times(1{+}\varepsilon)^{1/3}
            = (1{+}\varepsilon)^{2/3}\,,\qquad
  f_2 = (1{+}\varepsilon)^{-2/3}\times(1{+}\varepsilon)^{-2/3}
      = (1{+}\varepsilon)^{-4/3}\,.
\end{equation}
Here each factor $f_i$ originates from the squashing of the
orthonormal coframe $e^a = r\,\lambda_a(\varepsilon)\,\sigma^a$ (no
sum), where $\lambda_0=\lambda_1=(1{+}\varepsilon)^{1/3}$ and
$\lambda_2=(1{+}\varepsilon)^{-2/3}$. The product form of $f_i$
reflects the two coframe factors that appear in the left-hand sides of
the Cartan structure equations for the corresponding index.

\subsubsection{Step~1: Levi-Civita connection}

Koszul formula:
\begin{equation}
  \Gamma^a{}_{bc}
  = \tfrac{1}{2}\bigl(C^a{}_{bc}+C^c{}_{ba}-C^b{}_{ac}\bigr)\,.
\end{equation}
Non-vanishing components are enumerated symbolically. At $\varepsilon=0$
these reduce to the paper~I results (Appendix~A.2.1 of~\cite{paper1}).

\subsubsection{Step~2: Riemann tensor}

Frame-basis curvature tensor:
\begin{equation}
  R^a{}_{bcd}
  = \Gamma^a{}_{ec}\,\Gamma^e{}_{bd}
  - \Gamma^a{}_{ed}\,\Gamma^e{}_{bc}
  + \Gamma^a{}_{be}\,C^e{}_{cd}\,.
\end{equation}
The third term is specific to frame bases (analogous to the
$\partial\Gamma$ term in coordinate bases). All $4^4=256$ components
are computed and the antisymmetries $R_{abcd}=-R_{abdc}$,
$R_{abcd}=-R_{bacd}$ are verified automatically.

\subsubsection{Step~3: Ricci tensor and scalar}

\begin{equation}
  R_{bd} = R^a{}_{bad}\,,\qquad R = R^a{}_a = R_{bb}\,.
\end{equation}
The Levi-Civita Ricci scalar:
\begin{equation}
  R_{\LC}(r,\varepsilon)
  = \frac{8\bigl(4(1{+}\varepsilon)^2-1\bigr)}
         {r^2\,(1{+}\varepsilon)^{8/3}}\,.
\end{equation}
At the isotropic point: $R_{\LC}(r,0)=24/r^2$, matching the standard
value for~$\Sthree$.

\subsubsection{Step~4: Weyl tensor}

The four-dimensional Weyl tensor:
\begin{equation}
  C_{abcd} = R_{abcd}
  - \tfrac{1}{2}\bigl(g_{ac}R_{bd}-g_{ad}R_{bc}
  -g_{bc}R_{ad}+g_{bd}R_{ac}\bigr)
  + \tfrac{R}{6}\bigl(g_{ac}g_{bd}-g_{ad}g_{bc}\bigr)\,.
\end{equation}
Computed in the orthonormal frame ($g_{ab}=\delta_{ab}$). All
$\binom{4}{2}^2=36$ independent components are evaluated symbolically.

\subsubsection{Step~5: Weyl scalar}

\begin{equation}
  C^2 = C_{abcd}\,C^{abcd}
  = \sum_{a,b,c,d}C_{abcd}^2\,.
\end{equation}
After applying SymPy's \texttt{simplify} and \texttt{factor}:
\begin{equation}
  \boxed{C^2(r,\varepsilon)
  = \frac{1024\,\varepsilon^2(\varepsilon{+}2)^2}
         {3\,r^4\,(1{+}\varepsilon)^{16/3}}\,.}
\end{equation}

\subsection{Complete List of Non-Vanishing Weyl Components}
\label{sec:app-weyl-list}

All 24 non-vanishing components share the common factor
$\varepsilon(\varepsilon{+}2)/[r^2(1{+}\varepsilon)^{8/3}]$.

\paragraph{Diagonal type ($C_{abab}$).}

\begin{table}[ht]
\centering
\caption{Independent diagonal Weyl-tensor components.}
\label{tab:weyl-diag}
\begin{tabular}{@{}ll@{}}
\toprule
Component & Value \\
\midrule
$C_{0101}$
  & $+\dfrac{16\,\varepsilon(\varepsilon{+}2)}
            {3r^2(1{+}\varepsilon)^{8/3}}$ \\[8pt]
$C_{0202}$
  & $-\dfrac{8\,\varepsilon(\varepsilon{+}2)}
            {3r^2(1{+}\varepsilon)^{8/3}}$ \\[8pt]
$C_{0303}$
  & $-\dfrac{8\,\varepsilon(\varepsilon{+}2)}
            {3r^2(1{+}\varepsilon)^{8/3}}$ \\[8pt]
$C_{1212}$
  & $-\dfrac{8\,\varepsilon(\varepsilon{+}2)}
            {3r^2(1{+}\varepsilon)^{8/3}}$ \\[8pt]
$C_{1313}$
  & $-\dfrac{8\,\varepsilon(\varepsilon{+}2)}
            {3r^2(1{+}\varepsilon)^{8/3}}$ \\[8pt]
$C_{2323}$
  & $+\dfrac{16\,\varepsilon(\varepsilon{+}2)}
            {3r^2(1{+}\varepsilon)^{8/3}}$ \\
\bottomrule
\end{tabular}
\end{table}

\paragraph{Cross-type components.}
The Weyl-tensor symmetries $C_{abcd}=C_{cdab}$,
$C_{abcd}=-C_{abdc}=-C_{bacd}$ generate the remaining 18 non-vanishing
components from the 6 independent diagonal components listed above.

\paragraph{Tracelessness verification.}
\begin{equation}
  C^a{}_{bac} = 0\qquad (\forall\,b,c)\,,
\end{equation}
confirmed by SymPy for all index combinations.

\subsection{Complete Symbolic Expression for
  \texorpdfstring{$\VEC(r,0)$}{V\_EC(r,0)}}
\label{sec:app-VEC}

The EC engine (\texttt{S3S1Engine}, MX mode, FULL variant) yields the
isotropic effective potential:
\begin{equation}
  \VEC(r,0)
  = \frac{2\pi^2 Lr}{3\kappa^2}
    \bigl(V^2 r^2
    + 6V\eta\kappa^2 r\,\theta_{\NY}
    + 9\eta^2 - 36\bigr)\,.
  \label{eq:VEC-symbolic}
\end{equation}
Substituting the reference parameters
$(V{=}4,\,\eta{=}{-}2,\,\theta_{\NY}{=}1,\,\kappa{=}1,\,L{=}1)$:
\begin{equation}
  \VEC(r,0) = \frac{32\pi^2}{3}\,r^2(r-3)\,.
\end{equation}
This cubic (in~$r$) has a minimum at:
\begin{equation}
  \frac{\dd\VEC}{\dd r}
  = \frac{32\pi^2}{3}\cdot 3r(r-2) = 0
  \quad\Rightarrow\quad r^*=2\,,
\end{equation}
with value
\begin{equation}
  \VEC(2,0) = \frac{32\pi^2}{3}\cdot 4\cdot(2-3)
  = -\frac{128\pi^2}{3} \approx -421.1\,.
\end{equation}

\subsection{Closed-Form Expression for
  \texorpdfstring{$C^2\cdot\Vol$}{C²·Vol}}
\label{sec:app-C2Vol}

The product of the Weyl scalar and the four-volume:
\begin{equation}
  C^2(r,\varepsilon)\cdot\Vol(r)
  = \frac{1024\,\varepsilon^2(\varepsilon{+}2)^2}
         {3\,r^4\,(1{+}\varepsilon)^{16/3}}
    \times 2\pi^2 Lr^3
  = \frac{2048\pi^2 L\,\varepsilon^2(\varepsilon{+}2)^2}
         {3\,r\,(1{+}\varepsilon)^{16/3}}\,.
\end{equation}
Asymptotic scalings:
\begin{itemize}
  \item $r\to 0$: $C^2\cdot\Vol\sim 1/r$ (diverges).
  \item $\varepsilon\to -1$ ($\delta=1{+}\varepsilon\to 0^+$):
    $C^2\cdot\Vol\sim 1/\delta^{16/3}$ (diverges).
\end{itemize}
This two-directional unboundedness is the mathematical basis of
Theorem~\ref{thm:instability}(a).

\subsection{Verification Script}
\label{sec:app-script}

All computations above are reproduced by the script
\texttt{scripts/proofs/analytical\_proof.py}, which automatically executes:
\begin{enumerate}
  \item Setup of squashed structure constants.
  \item Levi-Civita connection computation.
  \item Riemann tensor computation with antisymmetry verification.
  \item Ricci tensor and scalar computation.
  \item Weyl tensor computation with tracelessness verification.
  \item $C^2$ simplification and factoring.
  \item Confirmation of $C^2=0$ at $\varepsilon=0$.
  \item First and second $\varepsilon$-derivatives at the isotropic
    point.
  \item $R_{\LC}$ computation and comparison with known values.
\end{enumerate}
