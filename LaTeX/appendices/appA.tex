%==============================================================================
% Appendix A: DPPUv2 Engine v4 Specification
%==============================================================================
\section{DPPUv2 Engine v4 Specification}
\label{app:engine}

This appendix documents the specification of the DPPUv2 computation
engine~v4 and its reliability verification.

\subsection{Engine Overview}
\label{sec:app-overview}

The DPPUv2 Engine~v4 is a symbolic--numerical computation engine that
performs minisuperspace reduction of the EC+NY+Weyl theory. Building on
the EC+NY implementation of paper~I, the following extensions have been
made for the present work:

\begin{enumerate}
  \item \textbf{Squashed ansatz}: introduction of the anisotropy
    parameter~$\varepsilon$ and implementation of the two-variable
    $(r,\varepsilon)$ effective-potential computation.

  \item \textbf{Weyl tensor from Levi-Civita connection}: symbolic
    computation of all Weyl-tensor components and the scalar~$C^2$
    from the Levi-Civita connection.

  \item \textbf{Modular architecture with automated consistency
    checks}: separation into Levi-Civita, EC-connection, Weyl-tensor,
    and effective-potential layers, with independent verification at
    each level (\S\ref{sec:app-checks}).
\end{enumerate}

\subsection{Computation Flow}
\label{sec:app-flow}

\paragraph{Theory construction (symbolic computation).}

\begin{enumerate}
  \item \textbf{Geometric setup}: topology-dependent structure constants
    $C^i{}_{jk}$.
  \item \textbf{Levi-Civita connection}: Koszul formula
    $\Gamma^a{}_{bc}=\tfrac{1}{2}(C^a{}_{bc}+C^c{}_{ba}-C^b{}_{ac})$.
  \item \textbf{Contortion}: from the torsion ansatz,
    $K_{abc}=\tfrac{1}{2}(T_{abc}+T_{bca}-T_{cab})$.
  \item \textbf{EC connection}:
    $\Gamma^a_{\EC,bc}=\Gamma^a_{\LC,bc}+K^a{}_{bc}$.
  \item \textbf{Scalar quantities}: $R_{\EC}$, $T_{abc}\,T^{abc}$,
    $N_{\TT}$, $N_{\REE}$, $N_{\FULL}$.
  \item \textbf{Weyl tensor}: from the Levi-Civita connection,
    $R^a{}_{bcd}\to R_{bd}\to R\to C_{abcd}\to C^2$.
  \item \textbf{Effective potential}: assembly via
    $\Veff = -\mathcal{L}\times\Vol$.
\end{enumerate}

\paragraph{Numerical exploration.}

\begin{enumerate}
  \item \textbf{Parameter scan}: $\alpha\in[-1,1]$ (201 points),
    $(r,\varepsilon)$ 2D grid.
  \item \textbf{Optimisation}: \texttt{scipy.optimize.brute} (Ns\,=\,20)
    grid search + multi-start L-BFGS-B.
  \item \textbf{Classification}: stability determination based on
    optimisation results (convergence flag, boundary-attachment
    detection).
\end{enumerate}

\subsection{Automated Consistency Checks}
\label{sec:app-checks}

The engine automatically performs the following checks at each
computation step:

\paragraph{Level~1: Metric compatibility.}
\begin{equation}
  \nabla_c\,g_{ab} = 0\,.
\end{equation}
Verification that the EC connection is compatible with the frame metric
$\eta_{ab}=\delta_{ab}$.

\paragraph{Level~2: Riemann-tensor symmetries.}
\begin{equation}
  R_{abcd} = -R_{abdc} = -R_{bacd} = R_{cdab}\,.
\end{equation}
All components are checked against the three symmetry conditions.

\paragraph{Level~3: Comparison with known analytic values.}
\begin{itemize}
  \item $\Sthree$: $R_{\LC}(r,0)=24/r^2$.
  \item $\Tthree$: $R_{\LC}=0$, $C^2=0$.
  \item $\Sthree$: $C^2(r,0)=0$ (conformal flatness at the isotropic
    point).
\end{itemize}

\subsection{\texorpdfstring{$\Tthree$}{T³} Null Test: Engine Sanity
  Check}
\label{sec:app-null}

Since $\Tthree$ has $C^2=0$, the effective potential should exhibit no
$\alpha$-dependence. Numerical confirmation at all 201 points
($\alpha\in[-1,1]$) with machine-precision agreement
(\S\ref{sec:null-test}) validates:
\begin{enumerate}
  \item Correct implementation of the $\alpha\,C^2$ term.
  \item Zero residual Weyl contribution for $C^2=0$ topologies.
  \item Maintenance of machine-level numerical precision.
\end{enumerate}

%------------------------------------------------------------------------------
\subsection{License}
\label{app:license}

The code and data are released under the MIT License:

\begin{quote}
\small
MIT License

Copyright (c) 2026 Muacca

Permission is hereby granted, free of charge, to any person obtaining a copy
of this software and associated documentation files (the ``Software''), to deal
in the Software without restriction, including without limitation the rights
to use, copy, modify, merge, publish, distribute, sublicense, and/or sell
copies of the Software, and to permit persons to whom the Software is
furnished to do so, subject to the following conditions:

The above copyright notice and this permission notice shall be included in all
copies or substantial portions of the Software.

THE SOFTWARE IS PROVIDED ``AS IS'', WITHOUT WARRANTY OF ANY KIND, EXPRESS OR
IMPLIED, INCLUDING BUT NOT LIMITED TO THE WARRANTIES OF MERCHANTABILITY,
FITNESS FOR A PARTICULAR PURPOSE AND NONINFRINGEMENT. IN NO EVENT SHALL THE
AUTHORS OR COPYRIGHT HOLDERS BE LIABLE FOR ANY CLAIM, DAMAGES OR OTHER
LIABILITY, WHETHER IN AN ACTION OF CONTRACT, TORT OR OTHERWISE, ARISING FROM,
OUT OF OR IN CONNECTION WITH THE SOFTWARE OR THE USE OR OTHER DEALINGS IN THE
SOFTWARE.
\end{quote}

%------------------------------------------------------------------------------
\subsection{Citation}
\label{app:citation}

If you use this code or data in your research, please cite:

\begin{quote}
\small
Muacca, ``Structural Robustness of Isotropic $\Sthree$ Vacua\\
in Einstein--Cartan Minisuperspace\\
via Chiral Equilibrium and Weyl Stability,'' 
\end{quote}

The Zenodo DOI for the code repository is:
\begin{center}
\texttt{10.5281/zenodo.18815498}
\end{center}

%------------------------------------------------------------------------------
\subsection{Contact}
\label{app:contact}

For questions, bug reports, or collaboration inquiries:

\begin{itemize}
  \item Email: \texttt{muacca@dmwp.jp}
  \item GitHub Issues: \texttt{https://github.com/Muacca/DPPUv2-paper02/issues}
\end{itemize}
