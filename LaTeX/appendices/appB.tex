%==============================================================================
% Appendix B: Numerical Verification Log & Nil³ Analysis
%==============================================================================
\section{Numerical Verification Log and \texorpdfstring{$\Nilthree$}{Nil³}
  Analysis}
\label{app:numerical}

This appendix provides the search parameters, numerical verification
details, and additional analysis of the $\Nilthree$ flat-limit
asymptotics and the universality of the $\alpha=0$ stability boundary.

\subsection{Search Parameter Summary}
\label{sec:app-search}

\subsubsection{Search ranges}

\begin{table}[ht]
\centering
\caption{Search parameter ranges.}
\label{tab:search-ranges}
\begin{tabular}{@{}lll@{}}
\toprule
Parameter & Range & Resolution \\
\midrule
$r$ & $[0.01,\,10]$ & brute: 30 points; L-BFGS-B: continuous \\
$\varepsilon$ & $[-0.95,\,5.0]$ & brute: 30 points; L-BFGS-B: continuous \\
$\alpha$ & $[-1,\,1]$ & 201 points (step 0.01) \\
\bottomrule
\end{tabular}
\end{table}

The upper bound $\varepsilon=5$ was chosen because: (i)~$\varepsilon^*$
remains at the boundary with $V_{\min}>0$ even at $\varepsilon=5$; and
(ii)~for larger~$\varepsilon$, the squashing factor
$(1{+}\varepsilon)^{-2/3}\ll 1$ challenges the interpretation of the
minisuperspace ansatz.

\subsubsection{Optimisation method}

\begin{enumerate}
  \item \textbf{Coarse search}: \texttt{scipy.optimize.brute} (Ns\,=\,30)
    over the 2D $(r,\varepsilon)$ grid.
  \item \textbf{Refinement}: multi-start L-BFGS-B from the top $n=8$
    grid candidates.
  \item \textbf{Convergence}: determined by the L-BFGS-B convergence
    flag; boundary attachment is reported as
    \texttt{converged\,=\,False}.
\end{enumerate}

\subsubsection{Paper~I reference parameters}

\begin{table}[ht]
\centering
\caption{Paper~I reference parameters.}
\label{tab:ref-params}
\begin{tabular}{@{}lr@{}}
\toprule
Parameter & Value \\
\midrule
$V$ & 4.0 \\
$\eta$ & $-2.0$, $-3.0$, $0.0$, $2.0$ \\
$\theta_{\NY}$ & 1.0 \\
$\kappa$ & 1.0 \\
$L$ & 1.0 \\
\bottomrule
\end{tabular}
\end{table}

\subsection{Output File Summary}
\label{sec:app-files}

Filenames follow the convention \texttt{\{script\}\_\{YYYYMMDD\_HHMMSS\}.csv};
\texttt{*} below denotes the timestamp.
All scripts reside under \texttt{scripts/}.

\begin{table}[ht]
\centering
\caption{Principal output files.}
\label{tab:output-files}
\begin{tabularx}{\textwidth}{@{}ll@{}}
\toprule
File pattern & Content \\
/ Script & / Section \\
\midrule
\texttt{potential\_landscape\_mapping\_*.csv}
  & $\Sthree$ $\Veff$ on $(r,\varepsilon)$ grid \\
\texttt{/ potential\_landscape\_mapping.py}
  & / \S\ref{sec:contour} (Fig.~\ref{fig:landscape}) \\
\quad\ \\[4pt]

\texttt{critical\_analysis\_*.csv}
  & $\Sthree$ $\alpha$-scan ($r^*,\varepsilon^*,V_{\min}$) \\
\texttt{/ critical\_analysis.py}
  & / \S\ref{sec:bifurcation} (Figs.~\ref{fig:bifurcation}, \ref{fig:size-stability}) \\
\quad\texttt{{-}{-}topology S3} \\[4pt]

\texttt{critical\_analysis\_*.csv}
  & $\Sthree$ fine scan ($\alpha\in[-0.1,0.1]$) \\
\texttt{/ critical\_analysis.py}
  & / \S\ref{sec:bifurcation} (inset) \\
\quad\texttt{{-}{-}topology S3 {-}{-}fine} \\[4pt]

\texttt{critical\_analysis\_*.csv} 
  & $\Tthree$ $\alpha$-scan \\
\texttt{/ critical\_analysis.py}
  & / \S\ref{sec:null-test} (Fig.~\ref{fig:null-test}) \\
\quad\texttt{{-}{-}topology T3} \\[4pt]

\texttt{topology\_comparison\_*.csv}
  & 3-topology $V_{\min}(\alpha)$ comparison \\
\texttt{/ topology\_comparison.py}
  & / \S\ref{sec:Vmin-comparison} (Fig.~\ref{fig:topology-comparison}) \\
\quad \\[4pt]

\texttt{run\_alpha\_boundary\_scan\_*.csv}
  & Parameter-dependence scans (4 sets) \\
\texttt{/ run\_alpha\_boundary\_scan.py}
  & / \S\ref{sec:facet-plot} (Fig.~\ref{fig:facet-plot}) \\
\quad ($\times4$ param.\ sets) \\
\bottomrule
\end{tabularx}
\end{table}

\subsection{Numerical Reliability}
\label{sec:app-reliability}

\subsubsection{Grid-resolution verification}

\begin{table}[ht]
\centering
\caption{Comparison of grid and optimisation results.}
\label{tab:grid-vs-opt}
\begin{tabular}{@{}llll@{}}
\toprule
Quantity & Grid minimum & Optimisation & Source of discrepancy \\
\midrule
$V_{\min}$ ($\Sthree$, $\alpha=0$) & $-415.5$ & $-421.103$
  & Grid resolution \\
$r^*$ & $\approx 2.13$ & $2.000$ & Grid resolution \\
\bottomrule
\end{tabular}
\end{table}

The coarse grid provides an overview of the potential landscape, while
L-BFGS-B identifies the precise minimum. The discrepancy is consistent
with grid-resolution limitations.

\subsubsection{Convergence flags for $\alpha>0$}

In the $\alpha>0$ region, \texttt{converged\,=\,False} is consistently
reported, with the optimum converging to the search boundary
$(r^*,\varepsilon^*)=(0.01,-0.95)$. This is \emph{not} an optimisation
failure but the correct report that the potential is unbounded below
within the search range. The true minimum lies along
$(r,\varepsilon)\to(0,-1)$, inaccessible due to search bounds.

\subsection{\texorpdfstring{$\Nilthree$}{Nil³} Behaviour: Flat-Limit
  Asymptotics}
\label{app:nil3}

\subsubsection{Numerical search and results}

An extended search over $\varepsilon\in[-0.95,5.0]$ (grid Ns\,=\,30,
multi-start $n=8$) yields:

\begin{table}[ht]
\centering
\caption{$\Nilthree$ search results ($\varepsilon_{\max}=5.0$).}
\label{tab:nil3}
\begin{tabular}{@{}ll@{}}
\toprule
Quantity & Result \\
\midrule
$r^*$ & $\approx 1.500$ \\
$\varepsilon^*$ & $5.0$ (upper-bound attachment) \\
$V_{\min}$ ($\alpha=-1$) & $5.0$ \\
$V_{\min}$ ($\alpha=-0.01$) & $4.9$ \\
\bottomrule
\end{tabular}
\end{table}

The attachment of $\varepsilon^*$ to the upper search bound indicates
that $\Veff$ is monotonically decreasing in $\varepsilon$ with no
interior minimum at finite~$\varepsilon$. The asymptotic value
$V_{\min}\approx 4.9$ is weakly $\alpha$-dependent.

\subsubsection{Physical interpretation: flat-limit asymptotics}

The $\Nilthree$ structure constants contain the factor
$(1{+}\varepsilon)^{-4/3}$:
\begin{equation}
  C^2{}_{01} = -\frac{(1{+}\varepsilon)^{-4/3}}{r}\,,\qquad
  C^2{}_{10} = +\frac{(1{+}\varepsilon)^{-4/3}}{r}\,.
\end{equation}
As $\varepsilon\to\infty$, $(1{+}\varepsilon)^{-4/3}\to 0$ and the
structure constants vanish, so $\Nilthree$ approaches the flat limit
($\Tthree$-like behaviour). At $\varepsilon=5$,
$(1{+}5)^{-4/3}\approx 0.11$, already suppressing the structure
constants to below 11\%.

Physically, the $\Nilthree$ effective potential minimises the curvature
cost ($C^2>0$) from the Heisenberg group structure by driving the
structure constants to zero ($\varepsilon\to\infty$): $\Nilthree$
``collapses'' towards a flat configuration.

\subsubsection{Impact on conclusions}

The $\Nilthree$ flat-limit asymptotics do not affect the main
conclusions:
\begin{enumerate}
  \item $\Nilthree$ has $V_{\min}\approx 5>0$, energetically
    unfavourable relative to $\Sthree$ ($V_{\min}=-421$). The ordering
    $V_{\min}(\Sthree)<0<V_{\min}(\Tthree)\approx 0
    <V_{\min}(\Nilthree)$ is preserved.
  \item The $\Sthree$ results ($\varepsilon^*=0$) are analytically
    established (Theorem~\ref{thm:stability}) and do not depend on the
    search range.
  \item The flat-limit approach is consistent with the paper's central
    theme: conformally flat configurations ($C^2=0$) are energetically
    favoured.
\end{enumerate}

\subsection{\texorpdfstring{$\alpha=0$}{α=0} Stability Boundary for
  \texorpdfstring{$\Nilthree$}{Nil³}: Additional Confirmation}
\label{app:nil3-alpha}

A high-resolution scan at $\alpha\in[-0.05,0.05]$ (step~0.001) with
$\varepsilon\in[-0.95,5.0]$ confirms:

\begin{table}[ht]
\centering
\caption{$\Nilthree$ stability-boundary confirmation.}
\label{tab:nil3-boundary}
\begin{tabular}{@{}ll@{}}
\toprule
Quantity & Value \\
\midrule
Transition location & $\alpha=0.000\to +0.001$ \\
Status at $\alpha=0$ & \texttt{converged\,=\,True}
  ($V_{\min}\approx 4.9$) \\
\bottomrule
\end{tabular}
\end{table}

The stability transition occurs at $\alpha=0\to +0.001$, with
$\alpha=0$ itself classified as stable. This provides additional
numerical evidence that the $\alpha=0$ stability boundary holds
universally across all three topologies ($\Sthree$, $\Tthree$,
$\Nilthree$).
