%==============================================================================
% Appendix D: Phase Atlas
%==============================================================================
\section{Phase Atlas: Effective Potential and Phase Diagram
  in the \texorpdfstring{$(\alpha_W,\varepsilon)$}{(αW,ε)} Parameter Space}
\label{app:phase-atlas}

This appendix systematically illustrates how the effective potential
$\Veff(r)$ and the $\Sthree\times\Sone$ phase diagram vary across five
representative cases in the $(\alpha_W,\varepsilon)$ parameter space
(Figs.~\ref{fig:potential-a0e0}--\ref{fig:potential-a+e-}).
These figures provide the visual support for Theorems~1, 2, and~3 in
the main text.

Each figure shows: (left panel) the phase diagram in the $(V,\eta)$
plane at fixed $\theta_{\NY}=1.00$, MX/FULL mode; (right panel) the
effective potential $\Veff(r)$ at three representative points.
The representative points are selected at fixed $V=4.0$ with varying
$\eta$, one per phase region ($\circ$: Stable Well; $\square$:
Intermediate; $\triangle$: No Well).

%------------------------------------------------------------------------------
\subsection{Baseline: \texorpdfstring{$\alpha_W=0$, $\varepsilon=0$}{αW=0, ε=0}}
\label{sec:appD-baseline}

\begin{figure}[ht]
  \centering
  \includegraphics[width=\textwidth]{figures/fig08_potential_a0e0}
  \caption{Effective potential and phase diagram for
    $\alpha_W=0$, $\varepsilon=0$ (isotropic $\Sthree$, Weyl term
    inactive). This reference case reproduces the three-phase structure
    of paper~I~\cite{paper1} and confirms the backward compatibility of
    engine~v4.}
  \label{fig:potential-a0e0}
\end{figure}

The case $\alpha_W=0$ (Weyl term inactive) and $\varepsilon=0$
(isotropic $\Sthree$) reproduces the three-phase structure of paper~I
exactly.  It serves as the reference case for verifying the backward
compatibility of engine~v4.

\paragraph{Phase diagram.}
\begin{itemize}
  \item \textbf{Type~I (Stable Well, green):} $\eta\lesssim -2.5$.
    The effective potential has a globally stable minimum.
  \item \textbf{Type~II (Rolling, yellow):} $-2.5\lesssim\eta\lesssim 2.3$.
    No barrier near $r\to 0$; the potential ``rolls'' without a true well.
  \item \textbf{Type~III (No Well, grey):} $\eta\gtrsim 2.3$.
    No well structure in the effective potential.
\end{itemize}

\paragraph{Effective potential.}
\begin{itemize}
  \item \textbf{Point~1} ($\eta=-4$, Type~I): well-type; deep global
    minimum near $r\approx 3$, high barrier as $r\to 0$, divergence
    as $r\to\infty$.  Typical isotropic $\Sthree$ vacuum profile.
  \item \textbf{Point~2} ($\eta=0$, Type~II): local minimum survives
    near $r\approx 1$, but the $r\to 0$ barrier disappears.
  \item \textbf{Point~3} ($\eta=3$, Type~III): monotonically
    increasing; no well formed at any~$r$.
\end{itemize}

%------------------------------------------------------------------------------
\subsection{Weyl Stable Region: \texorpdfstring{$\alpha_W<0$}{αW<0}}
\label{sec:appD-weyl-stable}

For $\alpha_W<0$, Theorem~1 guarantees that the Weyl term acts as a
penalty in the anisotropy direction, protecting the isotropic vacuum
$(\varepsilon=0)$.  In this regime, Types~I and~II are further
subdivided into \textbf{Type~I-W} (Double Well) and \textbf{Type~II-W}
(Single Well), which are specific to $\alpha_W<0$.

The two cases below show how the sign of~$\varepsilon$ shifts the
phase boundary.

%- - - - - - - - - - - - - - - - - - - - - - - - - - - - - - - - - - - - - -
\subsubsection{\texorpdfstring{$\alpha_W=-0.100$, $\varepsilon=+0.150$}{αW=-0.100, ε=+0.150}}
\label{sec:appD-a-e+}

\begin{figure}[ht]
  \centering
  \includegraphics[width=\textwidth]{figures/fig09_potential_a-e+}
  \caption{Effective potential and phase diagram for
    $\alpha_W<0$, $\varepsilon>0$.  The combination produces the widest
    Type~I-W (Double Well) region and strongly supports the stable
    vacuum.}
  \label{fig:potential-a-e+}
\end{figure}

\paragraph{Phase diagram.}
\begin{itemize}
  \item \textbf{Type~I-W (Double Well, blue):} wide stable region at
    $\eta\lesssim -2.5$.  The effective potential has a local maximum
    near the repulsive core and a deep global minimum further out.
  \item \textbf{Type~II-W (Single Well, orange):} upper region
    ($\eta\gtrsim -2.5$); single-well potential profile.
\end{itemize}

\paragraph{Effective potential.}
\begin{itemize}
  \item \textbf{Point~1} ($\eta=-4$, Type~I-W): double-well type;
    local maximum near $r\to 0$ (potential barrier), local minimum
    between the barrier and the repulsive core, and a deep global
    minimum near $r\approx 3$.  The Weyl correction acts as a barrier
    against collapse toward $r=0$.
  \item \textbf{Point~2} ($\eta=0$, Type~II-W): single-well type;
    the $r\to 0$ barrier disappears, leaving one clear minimum near
    $r\approx 1$.
  \item \textbf{Point~3} ($\eta=3$, Type~II-W): also single-well;
    formerly Type~III at $\varepsilon=0$, now acquiring a minimum at
    the repulsive-core edge.
\end{itemize}

The combination $\alpha_W<0$ and $\varepsilon>0$ produces the widest
Type~I-W region and most strongly supports the stable vacuum.

%- - - - - - - - - - - - - - - - - - - - - - - - - - - - - - - - - - - - - -
\subsubsection{\texorpdfstring{$\alpha_W=-0.100$, $\varepsilon=-0.150$}{αW=-0.100, ε=-0.150}}
\label{sec:appD-a-e-}

\begin{figure}[ht]
  \centering
  \includegraphics[width=\textwidth]{figures/fig10_potential_a-e-}
  \caption{Effective potential and phase diagram for
    $\alpha_W<0$, $\varepsilon<0$.  Compared with Fig.~\ref{fig:potential-a-e+},
    the Type~I-W region contracts and the phase boundary shifts to
    smaller~$\eta$, reflecting the asymmetric $\varepsilon$-dependence of
    the Weyl scalar.}
  \label{fig:potential-a-e-}
\end{figure}

\paragraph{Phase diagram.}
The same two-phase structure (Type~I-W blue, Type~II-W orange) as
Fig.~\ref{fig:potential-a-e+}, but the Type~I-W (double-well) region
contracts and the phase boundary shifts toward smaller~$\eta$ compared
with the $\varepsilon>0$ case.

\paragraph{Effective potential.}
All three representative points fall in Type~II-W (single-well).
The potential barrier seen at Point~1 in Fig.~\ref{fig:potential-a-e+}
disappears, and the potential descends smoothly to a single minimum.

The asymmetric reduction of the Type~I-W region when $\varepsilon$
changes sign corresponds to the fact that the Weyl scalar
$C^2\propto\varepsilon^2(\varepsilon{+}2)^2/r^4$ is not symmetric
under $\varepsilon\leftrightarrow-\varepsilon$.

%------------------------------------------------------------------------------
\subsection{Weyl Unstable Region: \texorpdfstring{$\alpha_W>0$}{αW>0}}
\label{sec:appD-weyl-unstable}

For $\alpha_W>0$, Theorem~2 guarantees
$\Veff(r,\varepsilon;\alpha_W)\to-\infty$ as $r\to 0^+$, so no global
stable minimum exists.  We define the phase exhibiting this behaviour
as \textbf{Type~G} (Ghost / Metastable).  Type~G may possess a
metastable local minimum at finite~$r$, but the true global minimum is
absent.

%- - - - - - - - - - - - - - - - - - - - - - - - - - - - - - - - - - - - - -
\subsubsection{\texorpdfstring{$\alpha_W=+0.100$, $\varepsilon=+0.150$}{αW=+0.100, ε=+0.150}}
\label{sec:appD-a+e+}

\begin{figure}[ht]
  \centering
  \includegraphics[width=\textwidth]{figures/fig11_potential_a+e+}
  \caption{Effective potential and phase diagram for
    $\alpha_W>0$, $\varepsilon>0$.  All stable-well phases
    (Types~I, II, I-W, II-W) are absent; $\alpha_W>0$ fundamentally
    destroys the stable vacuum.}
  \label{fig:potential-a+e+}
\end{figure}

\paragraph{Phase diagram.}
\begin{itemize}
  \item \textbf{Type~G (Metastable, red):} occupies the region
    $\eta\lesssim 2$.  No globally stable minimum exists.
  \item \textbf{Type~III-G (No Well, grey):} upper region
    ($\eta\gtrsim 2$).
\end{itemize}
Stable-well phases (Types~I, II, I-W, II-W) are entirely absent,
confirming visually that $\alpha_W>0$ fundamentally destroys the stable
vacuum.

\paragraph{Effective potential.}
\begin{itemize}
  \item \textbf{Point~1} ($\eta=-4$, Type~G): $\Veff\to-\infty$ as
    $r\to 0^+$.  A remnant of a metastable local minimum is visible at
    finite~$r$, but the true global minimum descends toward $r=0$
    without bound.  This is a direct numerical confirmation of
    Theorem~2.
  \item \textbf{Point~2} ($\eta=0$, Type~G): also $\Veff\to-\infty$
    as $r\to 0^+$.
  \item \textbf{Point~3} ($\eta=3$, Type~III-G): monotonically
    increasing in~$r$, but with $\Veff\to-\infty$ as $r\to 0^+$.
\end{itemize}

%- - - - - - - - - - - - - - - - - - - - - - - - - - - - - - - - - - - - - -
\subsubsection{\texorpdfstring{$\alpha_W=+0.100$, $\varepsilon=-0.150$}{αW=+0.100, ε=-0.150}}
\label{sec:appD-a+e-}

\begin{figure}[ht]
  \centering
  \includegraphics[width=\textwidth]{figures/fig12_potential_a+e-}
  \caption{Effective potential and phase diagram for
    $\alpha_W>0$, $\varepsilon<0$.  The two-phase structure
    (Type~G + Type~III-G) is essentially identical to
    Fig.~\ref{fig:potential-a+e+}; the sign of~$\varepsilon$ does not
    alter the destabilisation caused by $\alpha_W>0$.}
  \label{fig:potential-a+e-}
\end{figure}

\paragraph{Phase diagram.}
Essentially the same two-phase structure as
Fig.~\ref{fig:potential-a+e+} (Type~G + Type~III-G).  Although the
phase boundary shifts slightly relative to Fig.~\ref{fig:potential-a+e+},
the dominance of Type~G is maintained regardless of the sign
of~$\varepsilon$.

\paragraph{Effective potential.}
\begin{itemize}
  \item \textbf{Point~1} ($\eta=-4$, Type~G): $\Veff\to-\infty$ as
    $r\to 0^+$; same collapse behaviour as Fig.~\ref{fig:potential-a+e+}.
  \item \textbf{Point~2} ($\eta=0$, Type~III-G): classified as Type~G
    in Fig.~\ref{fig:potential-a+e+}, but transitions to Type~III-G
    under $\varepsilon<0$.
  \item \textbf{Point~3} ($\eta=3$, Type~III-G): monotonically
    increasing.
\end{itemize}

The uniform destabilisation by $\alpha_W>0$ regardless of the sign
of~$\varepsilon$ is consistent with Theorem~3 (universality of the
$\alpha=0$ stability boundary).
